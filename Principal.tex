\documentclass[rascunho, oneside]{fei}

\usepackage[utf8]{inputenc}
\usepackage{amsthm}
\usepackage{caption}
\usepackage{subcaption}

\graphicspath{{images/}}

\newtheorem{mydef}{Definição}

\hypersetup{
pdftitle={Framework para Comportamento Social Adaptativo do Robô na Aproximação Física do Ser Humano},
pdfauthor={Andrey Araujo Masiero},
pdfkeywords={\emph{Proxemics};}{Redes Bayesianas;}{Robótica Social;}{PCA;}{QGSIM;},
}

\author{Andrey Araujo Masiero}
\title{\emph{Framework} para Comportamento Social Adaptativo do Robô na Aproximação Física do Ser Humano}

\addbibresource{biblio.bib}

\begin{document}

%!TEX root=Principal.tex
\maketitle{}

\begin{folhaderosto}
Tese de Doutorado apresentada ao Centro Universitário da FEI para obtenção do título de Doutor em Engenharia Elétrica, orientado pelo Prof. Dr. Plinio Thomaz Aquino Junior e coorientado pelo Prof. Dr. Flavio Tonidandel.
\end{folhaderosto}

%\fichacatalografica
%\folhadeaprovacao

%%%%%%%%%%%%%%%%%%%%%%%%%%%%%%%%%%%%%%%%%%%%%%%%%%%%%%%%%

\dedicatoria{A Deus e a minha família que são o alicerce de minha vida.}

%%%%%%%%%%%%%%%%%%%%%%%%%%%%%%%%%%%%%%%%%%%%%%%%%%%%%%%%%
\begin{agradecimentos}
Em primeiro lugar gostaria de agradecer a Deus, que sempre me trouxe sabedoria e luz, mesmo nos momentos difíceis dessa jornada e de tantas outras.

À minha mãe Kathia, que desde o primeiro momento me apoiou e incentivou, mesmo quando tudo parecia impossível e eu não conseguia ver a luz no fim do túnel.

À minha irmã Andressa, que me suportou quando fiquei exaltado de felicidade ou tristeza perante as dificuldades.

Aos meus avós, Hélio e Rachel, que mesmo não presentes em carne, continuam iluminando minha vida e me guiam pelos caminhos que percorro deixando a sensação de sempre estar seguro.

Ao professor e orientador Plinio Thomaz Aquino Junior, que me auxilia a direcionar nos caminhos ao longo da jornada acadêmica e pessoal, com seus sábios conselhos e cumplicidade, fortalecendo a parceira a cada momento nesses últimos anos.

Ao professor e coorientador Flavio Tonidandel, que ajudou a tornar esse trabalho possível, com seus conselhos e ensinamentos, além de sempre puxar a minha orelha quando algo estava estranho ou elogiar sempre que eu conseguia um bom resultado. Tudo isso faz com que nossa parceria seja majestosa, desde a época do mestrado.

Aos professores da FEI, que compartilharam ao longo desse período seus conhecimentos e amizade, ajudando na evolução desse trabalho e também a minha como pessoa.

Aos meus amigos, que sem esse laço seria impossível avançar mais um passo neste caminho cheio de curvas. Os momentos de descontração, de discussão, almoços e principalmente cafés foram e são de extrema importância para nos ajudar a andar no caminho chamado vida.

E por fim a todos que de alguma maneira contribuíram para mais essa conquista.

\end{agradecimentos}

%%%%%%%%%%%%%%%%%%%%%%%%%%%%%%%%%%%%%%%%%%%%%%%%%%%%%%%%%
\epigrafe{In life, unlike chess, the game continues after checkmate.}{Isaac Asimov, 1988}

%%%%%%%%%%%%%%%%%%%%%%%%%%%%%%%%%%%%%%%%%%%%%%%%%%%%%%%%%
\begin{resumo}
A evolução da tecnologia torna-se cada vez mais evidente com o passar dos anos. As pessoas possuem computadores portáteis menores e com melhor configuração, \emph{tablets}, aparelhos de telefonia móvel inteligentes interligados com relógios e também robôs que possuem tarefas específicas como aspirar o pó da casa ou monitorar o ambiente a partir de um determinado ponto. Contudo, o robô inserido no ambiente doméstico ou pessoal atual, é apenas mais um dispositivo tecnológico que a pessoa possui. Caso um robô autônomo capaz de realizar diversas tarefas domésticas e de cuidados pessoais médicos seja inserido nesse ambiente e ainda ele realize interações através de voz, gestos e toque com o ser humano, o sentimento a partir desse momento não seria mais de um dispositivo tecnológico no ambiente. Existe uma possibilidade do ser humano ficar de uma certa maneira desconfortável com a presença do robô. Considerando a situação de desconforto do ser humano com o robô, essa tese propõem uma metodologia que mapeia o conjunto de ações que o robô é capaz de executar visando a maximização da probabilidade de uma interação humano-robô com maior qualidade, baseando-se no comportamento e características do indivíduo. A partir do mapeamento de comportamento da pessoa é possível determinar o comportamento que o robô deve ter para proporcionar uma situação confortável para a interação com o ser humano. Como resultado espera-se um \emph{framework} que possa aprender e analisar o comportamento do ser humano e que também seja capaz de transferir esse conhecimento com o robô inserido no ambiente, aumentando a eficácia da interação entre humanos e robôs.

\palavraschave{Robótica Social, Proxemics, Aprendizado de Máquina, Interação Humano-Robô}
\end{resumo}
%%%%%%%%%%%%%%%%%%%%%%%%%%%%%%%%%%%%%%%%%%%%%%%%%%%%%%%%%
\begin{abstract}
The technology's evolution has increased over the years. People have smaller laptops with better set up, tablets, smartphones interconnected with watches and also robots, which have specific tasks such as vacuuming or monitoring the environment from a certain point. However, the robot inserted into the current household or staff, is just another technological device that the person has. If an autonomous robot, able to perform various household chores and personal care doctors to be entered in this environment and still perform it interactions via voice, gestures and touch with the human being, the feeling would be no more than a technological device into the environment. There is a possibility of human beings in a way become uncomfortable with the presence of the robot. Considering the uncomfortable situation of the human being with the robot, this thesis proposes a methodology that maps the set of actions that the robot is able to perform in order to maximize the likelihood of human-robot interaction with higher quality, based on behavior and characteristics of the individual. From the behavior of the person mapping you can determine the behavior that the robot should have to provide a comfortable situation for interaction with humans. As a result we expect a framework that can learn and analyze the human behavior and also be able to transfer this knowledge to the robot inserted in the environment, increasing the effectiveness of the interaction between humans and robots.

\keywords{Social Robotic, Proxemics, Machine Learning, Human-Robot Interaction}
\end{abstract}



%%%%%%%%%%%%%%%%%%%%%%%%%%%%%%%%%%%%%%%%%%%%%%%%%%%%%%%%%
\listoffigures
% \listoftables
% \listofalgorithms
% \printglossaries

\tableofcontents

%%%%%%%%%%%%%%%%%%%%%%%%%%%%%%%%%%%%%%%%%%%%%%%%%%%%%%%%%

%!TEX root=Principal.tex
\chapter{INTRODUÇÃO}
\label{cap:introducao}
Com o passar dos anos é possível acompanhar a evolução dos sistemas computacionais, como por exemplo os telefones móveis, os computadores pessoais e portáteis, as televisões, e também os robôs pessoais, como o aspirador de pó iRobot Roomba\footnote{http://www.irobot.com/For-the-Home/Vacuum-Cleaning/Roomba.aspx} e o assistente pessoal JIBO\footnote{https://www.jibo.com/}. A evolução dos telefones móveis inteligentes mostra uma alta capacidade na realização de processamento de informações para executar diversas tarefas no dia a dia. Os componentes eletrônicos que compõem os aparelhos também diminuiram o tamanho. Isso permite que os aparelhos sejam mais finos, leves e com maior capacidade de processamento. Há também a inserção de robôs móveis em ambientes sociais, como as casas, hospitais e hotéis, unidos ao cenário da \emph{internet} das coisas~\cite{heenan:2014}. Esses robôs são chamados de robôs de serviços. Segundo o \textit{International Federation of Robotics (IFR)}~\footnote{https://ifr.org/}, o ápice de robótica de serviço ocorrerá em 2020. Em 2016, a indústria de robótica de serviço movimentou \$40 bilhões.

A robótica de serviços abrange todos os ramos que não envolvem a indústria propriamente dita. A robótica de serviços possui desde robôs pessoais e domésticos até robôs de aplicações profissionais. Alguns exemplos de aplicações profissionais abrangem áreas como defesa, medicina, agricultura, pecuária, logísticas, entre outros, ou seja, qualquer robô que utilizado em ambiente não industrial. Segundo a IFR, de 2018 a 2020 é esperado um crescimento de 400 mil unidades, ou seja, entre 20 a 25\% ao ano.

A área de maior crescimento é a de logística com o crescimento por ano de 189.700 unidades. Quando fala-se de valores monetários o crescimento é de \$26,8 bilhões no mesmo período. Ao falar de investimentos a área médica é a que possui um valor de projetos de \$7,8 milhões. De uma certa forma, isso auxilia na popularização dos projetos em robótica de serviço.

A popularização da robótica tem crescido também pela depreciasão de componenentes comuns de tecnologia, como câmeras, computadores, sensores de distância, e \emph{tablets}. Esse fenômeno faz com que pesquisadores e fabricantes investiguem a necessidade de robôs inteligentes, que possuam a habilidade de interagir com as pessoas. Com a popularização do contato na interação humano-robô aumentará a necessidade de criar projetos com robôs que atendam as necessidades de cada usuário~\cite{looi:2012}. Isso torna a interação entre robôs e seres humanos importante, não apenas pela questão social, mas também porque uma boa interação passa a ser uma questão essencial para a convivência entre humanos e robôs. Ao considerar que robôs encontram-se em ambientes sociais inteligentes como casas, hospitais, escolas, hotéis, investigar o desenvolvimento de robôs sociais é fundamental~\cite{albo-canals:2013, brown:2013}.

Um robô móvel inteligente possui várias maneiras de interagir. É capaz de identificar alguns padrões e ainda ter um nível de autonomia para tomada de decisões. O robô realiza as tarefas de interação através de sensores e atuadores espalhados em sua estrutura. Alguns sensores utilizados na construção dele são câmeras, infravermelhos, \emph{laser}, de profundidade, térmicos, entre outros. Os atuadores são todos os dispositivos que possam gerar interação, externando algo para o indivíduo, seja através de um movimento, uma imagem ou até mesmo algum sinal sonoro. Alguns exemplos de atuadores são: \emph{tablets}, caixas de som, manipuladores e motores~\cite{looi:2012, choi:2014, dobra:2014}.

Apesar da popularização e de pesquisas voltadas para a robótica de serviços, a documentação e formalização dos passos do projeto de interação humano-robô são 
pouco aplicados. Projetos de interação humano-robô não possuem um nível de detalhe e preocupação para que o robô seja melhor aceito pela sociedade~\cite{alenljung:2017}. Técnicas de engenharia de \textit{software} e usabilidade, principalmente projetos centrados no usuário, disseminada na comunidade de pesquisadores de interação humano-computador, são pouco aplicadas em projetos voltados para robôs autônomos para interações sociais~\cite{alenljung:2017}.

Com o uso das técnicas adequadas é possível estabelecer passos para construção de robôs, com uma boa especificação, podendo reproduzi-lo e reaproveitar o projeto em diversos cenários. Além disso, a manutenção do robô também é favorecida com o método utilizado em sua construção é adequado, assim como sua evolução para versões mais robustas. Quando cenários de interação social são analisados, o ponto chave é manter as pessoas na interação confortáveis e sem medo de uma aproximação ou abordagem de qualquer agente envolvido, seja robô ou ser humano. No caso do ser humano, é necessário conhecer o perfil do usuário em questão, e estabelecer ações que auxiliem a manter o seu nível de conforto e assim manter uma interação de longa duração.

Para aumentar as possibilidades de reutilização, manutenção e evolução desses projetos, é necessário, criar de maneira sistêmica, toda a documentação com a especificação do robô (\emph{hardware e software}). Sendo assim, essa tese apresenta um método centrado no usuário para a construção de um projeto de interação humano-robô autônoma possibilitando uma melhor manutenção e melhorias do projeto sempre visando uma interação social de qualidade. Questões que são abordadas durante a construção do projeto são: \emph{hardware}, \emph{software}, contexto de uso, cenário de testes, participantes de testes, funcionalidades do robô, entre outros. Como o projeto é centrado no usuário, é importante traçar as características do perfil deste usuário para detectar melhor as suas necessidades e preferências.

A técnica utilizada para identificar o perfil do usuário neste trabalho é a de Personas. Personas são arquétipos hipotéticos que representam um grupo de usuários reais através de um personagem fictício~\cite{aquino:2005, masiero:2011}. A técnica de Persona é importante, pois o alcance em número de perfis é alto e a quantidade de personagens é bem menor. Isso torna a comunicação e as tomadas de decisões centradas no usuário mais fácil, do que olhando perfil a perfil. Assim, o método proposto aqui, utiliza a técnica de Persona para garantir as preferências do usuário durante a construção do projeto. 

Para demonstrar o uso do método proposto é realizado um estudo de caso para a construção de um robô autônomo, que realiza a classificação do perfil do usuário através de uma rede Bayesiana. O classificador apresentado nessa tese, é um classificador Bayesiano que utiliza informações sobre as ações do robô, comportamentais, cenário e de percepção sobre heurísticas de avaliação de usabilidade, para identificar o perfil do usuário definido como Personas. A partir da classificação é possível identificar ações que o robô deve realizar para melhorar a interação com o usuário. Essa evolução do estudo de caso é apresentada como um novo ciclo do método proposto para a construção do robô.

%%%%%%%%%%%%%%%%%%%%%
\section{OBJETIVO} % OK
Como objetivo esta tese propõem a construção de um método sistêmico para projetar e desenvolver um robô totalmente autônomo e centrado no usuário para interação social.

%%%%%%%%%%%%%%%%%%%%%
\section{MOTIVAÇÃO}
O crescente número de pesquisas em robótica aplicados em ambientes sociais como casas, hospitais e escolas fazem com que seja um tópico de atenção entre os pesquisadores. Esse é um tópico importante, pois os diferentes formatos existentes de robôs podem gerar problemas de confiabilidade. Esse é um ponto que pode determinar o conforto do usuário ao estar em mesmo ambiente que o robô. Por consequência, a questão da confiabilidade pode determinar a aceitação do robô.

Para mitigar esse problema, vários fatores devem ser analisados. Fatores como o perfil do usuário na interação social e também as características do projeto do robô. Todas essas informações são consideradas para que o robô possa predizer quais são as melhores ações de interação com um determinado indivíduo. Encontrar uma solução  para esse problema é uma tarefa complexa. Deve-se considerar a coleta e o processamento dessas informações para a tomada de decisão correta, o que em muitas vezes é necessário sensores dedicados a uma tarefa específica, como o sensor de profundidade.

O custo de processamento dessas informações pode ser alto para o robô pois, sua infraestrutura tem uma capacidade computacional e eletrônica que limita a tarefa. Sendo assim, é necessário que exista uma arquitetura de sistema capaz de considerar a manutenção e expansão dos equipamentos utilizados na construção do robô. Assim, é possível fazer com que o robô evolua ao longo do tempo.

%%%%%%%%%%%%%%%%%%%%%%%%%
\section{JUSTIFICATIVA}
Durante os estudos de trabalhos que realizam a análise de comportamento humano através de robôs aplicados principalmente em robótica social, notou-se que existem poucos estudos voltados ao projeto de interação humano-robô e aplicações que atendam as necessidades do usuário de maneira sistêmica. Além disso, alguns trabalhos~\cite{okita:2012, henkel:2012b, vazquez:2014} utilizam a técnica de \emph{Wizard of OZ} (WoZ) para realizar os testes com humanos. Essa técnica condiz com o controle do robô de maneira remota, como se este fosse totalmente autônomo. Esse tipo de técnica, não consegue transmitir de maneira adequada o comportamento do robô, uma vez que ela não consegue trabalhar com os ruídos dos sensores e problemas encontrados durante uma navegação autônoma, como combinação do mapa ou detecção de obstáculos e pessoas. Tudo isso é realizado pelo operador em posse do controle remoto, tirando a naturalidade e autenticidade da interação do robô.

Assim,  é necessário a criação de um método que seja capaz de construir um robô possa, de maneira autônoma, realizar diversas tarefas e tomadas de decisão durante interação, de maneira sistêmica. Essa pesquisa é importante para que haja uma evolução dos ambientes inteligentes, principalmente os que consideram o robô como um agente. Além da evolução dos ambientes inteligentes, manter o indivíduo com a melhor experiência de interação com o robô, e também faze-lo confortável com a presença do robô.

%%%%%%%%%%%%%%%%%%%%%%
\section{METODOLOGIA}
A fundamentação do trabalho é realizada em pesquisas de cada uma das áreas abrangentes, interação humano-robô (IHR), conceito de \emph{proxemics}, experiência de usuário aplicado a IHR, onde identificou-se a necessidade da criação de um projeto sistêmico para IHR. A partir desse projeto é possível determinar os passos para a construção do robô, especificação do contexto de uso, perfil de usuários para interação, e ferramentas de testes. O projeto foi submetido ao comitê de ética o projeto para aprovação dos testes com seres humanos.

A primeira bateria de testes foi realizada. A partir dos resultados dos testes piloto, é aplicado o algoritmo QG-SIM para construção dos grupos de perfis similares. Com cada grupo identificado, são criadas as Personas que devem ser classificadas pelo robô com base nas informações obtidas através das variáveis de observação identificadas na fase de concepção do método proposto. A partir desse ponto, é realizado um estudo de caso para viabilizar a aplicação do método na construção real de um projeto de interação humano-robô. As variáveis e observações feitas durante os testes pilotos são utilizadas para determinar as variáveis que compõem o classificador. Para o classificador é utilizado a técnica probabilística, rede Bayesiana. O objetivo é eliminar repetição das dependências condicionais apresentadas na construção da estrutura da rede.

Na sequência novos testes são realizados, para que seja possível a validação do classificador e das questões referentes ao perfil do usuário, demonstrando as fases de teste e análise do método proposto. O cenário de teste utilizado é um ambiente simulado de residência, onde o robô habita com mais uma pessoa. Realizados os testes, os resultados são analisados e discutidos, apresentando as estatísticas e observações obtidas durante o processo. Por fim, os próximos passos para o projeto são apresentados.

%%%%%%%%%%%%%%%%%%%%%%%%%%%%%%
\section{ESTRUTURA DO TRABALHO} % OK
Esta tese é composta por um total de 7 capítulos discriminados a seguir.

O capítulo \ref{cap:introducao} apresenta a \textbf{introdução} do trabalho conduzindo o leitor ao problema que a pesquisa desta tese deve contribuir.

O capítulo \ref{cap:ihr} introduz a área de \textbf{interação humano-robô}, contando um pouco da história e sua importância para o futuro.

O capítulo \ref{cap:ux} introduz a área de \textbf{interação humano-computador}, apresentando os principais conceitos para o desenvolvimento de um sistema centrado no usuário. Também é apresentado trabalhos que aplicam as técnicas em cenários de interação humano-robô.

O capítulo \ref{cap:projetoihr} apresenta a \textbf{especificação do método para construção de um projeto em interação humano-robô} apresentado para a construção de robô de serviço autônomo.

O capítulo \ref{cap:estudocaso} apresenta um \textbf{estudo de caso}  através da criação de um classificador para demonstra a aplicação do método proposto por esta tese.

O capítulo \ref{cap:resultados} apresenta os \textbf{resultados e discussões} desta tese.

O capítulo \ref{cap:conclusoes} apresenta as \textbf{conclusões e trabalho futuros} obtidos ao longo dos estudos e testes dessa tese.

%!TEX root=Principal.tex
\chapter{INTERAÇÃO HUMANO-ROBÔ}
\label{cap:ihr}
Interação Humano-Robô (IHR) é a área de estudo que procura compreender, avaliar e implementar os robôs para que possam trabalhar em conjunto ou para o ser humano onde a interação seja menos invasiva e mais colaborativa. O primeiro guia da IHR apareceu em um trabalho de ficção científica de Isaac Asimov, que apresentou as primeiras leis da robótica. A primeira lei fala que um robô não pode ferir um ser humano e também deve proteje-lo para que nenhum mal o seja causado. A segunda lei diz que um robô deve obedecer as ordens dadas por seres humanos exceto nos casos que as ordens entrem em conflito com a primeira lei. E por fim a terceira lei diz um robô deve proteger sua própria existência desde que não entre em conflito com a primeira e/ou segunda leis. Essas leis regem os trabalhos voltados a IHR até nos dias atuais~\cite{Goodrich:2007, Weiss:2010}. 

Qualquer tipo de robô necessita de interação, mesmo os completamente autônomos. A interação pode ocorrer de duas maneiras específicas: Interações Remotas (robôs e humanos estão em diferentes locais espaço-temporais), por exemplo, a operação do robô Curiosity\footnote{https://www.nasa.gov/mission\_pages/msl/index.html} em Marte e a NASA aqui no planeta Terra; Interações Próximas (robôs e humanos estão em um mesmo local, compartilhando o mesmo espaço), por exemplo, em indústrias ou residências como o robô Roomba~\cite{Goodrich:2007}. 

Robôs teleoperados devem ser guiados por controles, como \emph{joysticks}, por exemplo. Já os robôs completamente autônomos devem consistir o ambiente, o cenário de atuação, os seres humanos que existem no ambiente e os que estão direcionando-o para o seu objetivo final, além de atualizar constantemente informações sobre o ambiente e suas restrições. Muitos trabalhos são direcionados a interação através de um controle ou central de comando com a operação de um ser humano, mas a quantidade de trabalhos com robôs autônomos vêem crescendo principalmente em pesquisas de robótica assistiva e/ou robótica para resgate em catástrofes, onde existe riscos a vida de seres humanos que procuram por vitimas~\cite{Goodrich:2007, Weiss:2010}.

IHR é um estudo que necessita da participação de diversas outras áreas de pesquisa, como Ciências Cognitivas, Linguística, Psicologia, Engenharia, Ciências da Computação, Matemática, Engenharia dos Fatores Humanos e Design. Além disso, é importante o estudo de padrões de interação para que sejam adotados pequenas perspectivas sobre soluções de problemas com interação, tornando mais fácil encontrar uma solução a algum problema que seja recorrente~\cite{Goodrich:2007}.

A interação pode ser defina pela atividade de trabalhar em conjunto para atingir o mesmo objetivo. A IHR afetada cinco fatores de interação, que são: (I) Nível e comportamento de autonomia; (II) Troca natural de informação; (III) Estrutura do time; (IV) Adaptação, aprendizado e treinamento de pessoas e robôs; e (V) Definir as tarefas. Um robô que possui um grande grau de autonomia é aquele que consegue permanecer desatento por um longo período de tempo sem realizar nenhum tipo de interação. Contudo em IHR a autonomia não é considerada com um resultado final, mas é um meio que auxilia no processo de interação~\cite{Goodrich:2007, Weiss:2010}.

O nível de autonomia de um robô determina o quanto esse pode agir por sua própria conta. Existem diversas formas de medir e analisar esse nível. O mais utilizado é a escala de Sheridan~\cite{Sheridan:1978} que apresenta um intervalo continuo desde de um robô que não realiza nenhuma tarefa por conta própria, ou seja, um robô teleoperado até um robô totalmente independente e autônomo. Apesar do grande uso da escala de Sheridan, sua aplicabilidade ao cenário completo pode não ser muito eficiente sendo melhor aplicado em subtarefas~\cite{Goodrich:2007, Weiss:2010}.

Em IHR o nível de autonomia pode ser melhor determinado por uma combinação entre o nível de interação entre humano e robô e o quanto ambos conseguem realizar as tarefas de forma independente. O desenvolvimento de habilidades cognitivas é importante para o robô interagir com o humano de maneira natural e eficiente. Nos anos 80, Brooks apresentou um novo paradigma para autonomia de robôs, conhecida com robôs baseados em comportamento~\cite{Brooks:1986, Brooks:1991}. Outro modelo chamado de sinta-pense-aja também é apresentado na literatura como uma arquitetura híbrida que apresenta um problema de desenvolver comportamentos que sejam natural e atividades robustas para robôs humanoides. Devido a isso, as áreas que trabalham no modelo cognitivo de aprendizagem e tomada de decisão tem crescido cada vez mais~\cite{Goodrich:2007}. 

Contudo, o estudo de interação entre humanos e robôs não se limita apenas ao nível de autonomia do robô. Modelos cognitivos, aplicações em ambientes sociais e principalmente em ambientes de cuidados médicos pessoais, têm se tornado cada vez mais frequentes nos novos estudos. \citeonline{Giovannangeli:2007} apresentam um modelo de IHR onde o robô é capaz de aprender tarefas a partir de uma pessoa realizando o papel de treinador, onde o robô reproduz seus movimentos e consegue armazena-lo para situações futuras. A teoria da mente também é aplicada em trabalhos de IHR. Ela auxilia o robô na análise do comportamento de um indivíduo e possibilita a tomada de decisão para uma interação próxima a natural~\cite{Hiatt:2011}.

Outro fator importante para IHR é a aparência do robô em conjunto com a capacidade de execução de tarefas esperada para àquela aparência. Dessa maneira, \citeonline{Minato:2007} apresentam uma plataforma robótica em formata de uma criança, mais precisamente um bebê, para realizar estudos de interação e principalmente a capacidade da cognição do robô durante a interação. Um outro modelo é apresentado nos estudos de IHR com o objetivo principal voltado para o mapeamento e análise do comportamento humano. Este modelo tem como sua essência a teoria de \emph{Proxemics}, que serve de base para essa tese e é apresentada em detalhes na seção~\ref{sec:proxemics}, a seguir.


%!TEX root=Principal.tex
\chapter{\emph{PROXEMICS}}
\label{cap:proxemics}
As pessoas, quando convivem em sociedade, tendem a respeitar o espaço existente entre cada individuo. Esse fenômeno é determinado como espaço social, sendo este medido através da distância social que é um dos princípios fundamentais para uma interação social com qualidade~\cite{Hall:1969, Henkel:2014}. A análise do comportamento das pessoas e a relação da distância social entre os indivíduos foi definido por~\citeonline{Hall:1969} como \emph{Proxemics}. \emph{Proxemics} então pode ser definido com o processo dinâmico de interação entre dois agentes, onde eles fiquem posicionados frente a frente e/ou próximos entre si~\cite{Mead:2011b}.

Durante os estudos de~\citeonline{Hall:1969}, observou-se que a questão da distância social está diretamente ligada a cultura de cada individuo. Isso quer dizer que a percepção dessa distância entre pessoas que viveram em regiões distintas poderá ser diferente.

Com base na teoria de~\citeonline{Hall:1969}, pode-se ilustrar o seguinte cenário como exemplo. Uma pessoa que vive no Brasil pode não se importar com o indivíduos muito próximos a ele. Em contra partida, a pessoa que vive no Japão talvez tenha preferência por manter uma distância maior entre ele e as demais pessoas durante o processo interação. Além disso, variáveis como gênero e idade também influenciam na relação espacial de interação entre indivíduos.

Apesar da observação sobre as variáveis que podem influenciar na relação da distância social, \citeonline{Hall:1969} não formalizou nenhuma regra sobre a distância social para interações entre indivíduos. Assim, \citeonline{Argyle:1988} definiu quatro zonas de proximidades, formalizando regras para distâncial social nas interações entre indivíduos. São elas: (I) Zona de Proximidade Pública; (II) Zona de Proximidade Social; (III) Zona de Proximidade Pessoal; e (IV) Zona de Proximidade Íntima. A figura~\ref{fig:proximityzones} ilustra a definição de~\citeonline{Argyle:1988} na formalização do espaço social.

\begin{figure}[ht!]
	\centering
	\begin{minipage}{\textwidth}
		\caption{Zonas de Proximidades}
		\includegraphics[width=\textwidth]{proxemicszones.png}
		\smallcaption{Fonte:~\citeonline{Argyle:1988}.}
		\label{fig:proximityzones}
	\end{minipage}
\end{figure}

Cada uma das zonas de proximidades apresentadas na figura~\ref{fig:proximityzones} possui características particulares que pode guiar como ocorrerão as interações sociais. Na zona de proximidade social, o individuo pode emitir sons com maior volume do que a zona de proximidade íntima que, por estarem muito próximos os indivíduos acabam se comunicando com sons mais baixos ou até mesmo sussurros. Interações na zona íntima são esperadas normalmente entre amigos muito próximos ou entre casais~\cite{Hall:1969, Argyle:1988}. O comportamento aceitável em zonas de proximidades mais distantes, como a social e a pública, é a comunicação com maior intensidade, movimentos mais amplos e até com uma força física maior que nas regiões mais próximas, onde há a probabilidade maior do indivíduo se assustar com esse tipo de comportamento~\cite{Henkel:2014}.

Além dos comportamentos diferentes em cada zona de proximidade, existe um outro fator que pode atrapalhar a interação exclusiva entre duas pessoas nas regiões mais distantes. A existência de pessoas inseridas nas regiões mais distantes, pode dificultar o estabelecimento uma interação exclusiva devido ao excesso de ruído no cenário. O ruído para esse cenário pode ser considerado através do volume excessivo de pessoas no local, junto com a altura dos sons emitidos e além da quantidade de gestos que cada individuo realiza simultaneamente~\cite{Walters:2009, Henkel:2014}. Isso pode influenciar diretamente no estabelecimento do ponto focal da interação.

Algo que pode ser feito para trabalhar mais próximo do ponto focal da interação é a aproximação entre os agentes, fazendo com que essa interação possua menos ruídos. Para que essa aproximação ocorra com sucesso, alguns fatores são importantes, como velocidade de aproximação, gestos e ruídos emitidos, entre outros fatores. Sendo assim, não é apenas o espaço social que a teoria de \emph{Proxemics} se refere, mas também à análise comportamental dos agentes envolvidos na interação. Algumas variáveis que são utilizadas para a leitura corporal também são utilizadas na análise comportamental. \citeonline{Mead:2013} lista algumas variáveis consideradas em seu trabalho, além da distância social, são elas: (I) orientação da postura; (II) orientação do quadril; (III) orientação dos ombros; (IV) posicionamento e orientação da cabeça; e (V) fixação do olhar entre os indivíduos. Todas as variáveis apresentadas por~\citeonline{Mead:2013} auxiliam a determinar a qualidade da interação social entre dois indivíduos, agentes ou entre robôs e humanos.

\emph{Proxemics} tem sido explorado em trabalhos de interação humano-robô~(IHR) desde 1997, somando aproximadamente 25 trabalhos de acordo com~\citeonline{Henkel:2014} e este número continua em constante crescimento. Contudo, não é apenas em IHR que o tema de \emph{Proxemics} é abordado. Trabalhos relacionados a tecnologia móvel e realidade virtual, também utilizam o tema com o intuito de melhorar a interação dos sistemas de maneira geral. Assim, a próxima seção apresentará os trabalhos relacionados que abordam o tema da \emph{Proxemics} e tecnologias, seguido pela abordagem em IHR, sempre tentando demonstrar o vínculo dos trabalhos apresentados e a tese defendida neste texto.

\section{\emph{Proxemics} e Interação Humano-Computador}
\label{sec:proxemicstec}
\citeonline{Hemmert:2013} apresentam um trabalho que tem como objetivo a aplicação de \emph{Proxemics} em aparelhos de telefonia móvel. A ideia principal é fazer com que o telefone reaja de acordo com a aproximação do aparelho pela voz da pessoa. O foco principal dentre as oito variáveis de \emph{Proxemics} é a postura do usuário. Interações de \emph{Proxemics} tem sido um dos modelos gerais em interação humano-computador (IHC), onde a parte mais observada é o comportamento do usuário na interação com os diversos sistemas e dispositivos. Entretanto, é um tema pouco explorado em telefonia móvel~\cite{Hemmert:2013}.

O projeto utilizado no caso de estudo apresenta uma nova maneira de interagir com dispositivos móveis, em especial telefones, tendo como base variáveis de linguagem corporal e proximidade do indivíduo para com o aparelho. Como trabalho futuro \citeonline{Hemmert:2013} querem apresentar um modelo que faça a leitura de diversas variáveis com o intuito de entender por completo como elas funcionam no comportamento do ser humano.

Um ponto interessante abordado por \citeonline{Hemmert:2013} é quando ele faz refêrencia ao uso de \emph{Proxemics} em IHR. Geralmente os trabalhos de \emph{Proxemics} são voltados ao comportamento de ser humano, e nenhum possui foco na reação do robô, ou seja, como esse robô reagirá caso um ser humano invada o espaço íntimo do robô sem consentimento. Esse é um ponto interessante que deve ser explorado ao longo dessa tese.

No trabalho proposto por \citeonline{Kastanis:2012} é apresentado um modelo de aprendizado por reforço aplicado a aproximação de uma pessoa a um avatar em um mundo virtual. O conceito de proximidade é aplicado com o intuito de validar se a tarefa de aproximação foi atingida com sucesso. Apenas 1 participante não alcançou o objetivo de 10, em um dos grupos. Ao todo das 20 pessoas que interagiram com o avatar utilizando aprendizado por reforço, apenas 3 não chegaram a uma posição de zona intima. Não foi avaliado no trabalho de aprendizado por reforço é a melhor técnica para isso, porém o algoritmo Q-Lambda atendeu ao experimento proposto.

\section{\emph{Proxemics} e Interação Humano-Robô}
\label{sec:proxemicsihr}
Nessa seção são apresentados os trabalhos de \emph{Proxemics} aplicados a interação humano-robô~(IHR), como o trabalho de \citeonline{Walters:2009} que propõem um \emph{framework} empírico com o objetivo de auxiliar a detecção da distância real, ou seja, a distância considerando fatores diversos da IHR.

Alguns fatores de impacto na IHR foram apresentados por \citeonline{Walters:2009} na discussão de seu trabalho. Um dos fatores explorados foi o impacto dos sons emitidos pelo robô durante a interação, ou seja, a voz do robô. A voz não causa impacto apenas pelo volume que é emitida, mas também o estilo dela que pode influenciar uma vez que é possível inferir emoções a partir do estilo em que a voz é emitida. Além disso, a voz também pode influenciar no tempo de aproximação entre o robô e o individuo, pois dependendo de como o som é emitido pode gerar insegurança ao individuo que está interagindo com o robô~\cite{Walters:2009}.

Fatores como a aparência do robô e informações demográficas como idade, gênero, grau de instrução, personalidade, carisma, entre outros também podem interferir na IHR. Por exemplo, as pessoas preferem manter uma distância maior dos robôs que possuem uma aparência humanoide, pois ela causa um pouco de preocupação sobre as ações dele, quando comparado a um robô com aparência mais mecânica. Entretanto, a altura do robô não é um fator que apresenta grande relevância para IHR~\cite{Walters:2009}.

Outro trabalho, apresentado por \citeonline{Torta:2011}, tem como objetivo a apresentação de um arquitetura para robótica baseada em comportamento que permite ao robô navegar em segurança por um ambiente doméstico mutável e consiga codificar interações não verbais de maneira embarcada. Dessa maneira, é possível fazer com que o robô possa apresentar o comportamento de aproximação adequado ao seu objetivo, utilizando um modelo de espaço pessoal.

O modelo utilizado considera a relação entre a orientação do robô em conjunto com a distância do objetivo e a avaliação do indivíduo para orientar a aproximação do robô~\cite{Torta:2011}. Para alcançar esse objetivo \citeonline{Torta:2011} utilizaram um filtro Bayesiano para inferir a localização do objetivo a partir da posição do robô de maneira dinâmica. O filtro Bayesiano é utilizado como guia para o robô em seu algoritmo de navegação.

Nos testes utilizou-se o robô NAO e obteve a validação de que a inclusão do espaço pessoal no algoritmo de navegação trouxe resultados positivos ao modelo implementado. Em estudos futuros, \citeonline{Torta:2011} incluirão outros fatores ao cenário de IHR, como a altura do robô, a aparência e o propósito da interação, e a partir dessas novas variáveis identificar como é possível melhorar a interação de tal forma, que esse modelo obtenha uma qualidade maior em sua aplicação~\cite{Torta:2011}.

A aplicação de \emph{Proxemics} não é exclusiva a robótica social ou doméstica, \citeonline{Srinivasan:2012} aplicam o conceito para o cenário de resgate de vítimas. O trabalho apresentado tem como objetivo avaliar a utilização do olhar social com movimentos de cabeça e funções escalares de \emph{Proxemics} para auxiliar na aproximação e trabalho em regaste de vítimas em centros urbanos.

Nesse cenário o robô deve manter a vítima calma, tranquila, com pensamentos positivos e cuidar dela, na medida do possível, até que o resgate consiga acesso ao local para que o trabalho de extração seja realizado com sucesso. Dois cenários de simulação foram utilizados para validar o método proposto: no primeiro cenário observou-se como a vítima correspondia a medida que o robô gesticulava com a cabeça durante a interação comparado ao robô totalmente estático. O movimento da cabeça foi programado para ficar sincronizado com a fala do robô de tal forma, que seu comportamento ficasse próximo a um comportamento natural. Neste primeiro cenário, foi validado a hipótese de que o usuário prefere o robô que tem o movimento social (gesticulação da cabeça) ao invés do robô que permanece totalmente estático durante a interação~\cite{Srinivasan:2012}.

No segundo cenário de simulação, utilizou-se funções escalares para definir a aproximação do robô junto à vítima. Nessa aproximação são consideradas as quatro regiões de proximidades, apresentadas na figura~\ref{fig:proximityzones}, para determinar a interação com a vítima. Foram comparadas três tipos de funções: (I) Logarítmica; (II) Linear; (III) Não escalar. Nos testes os melhores resultados foram obtidos através da função logarítmica, seguida pela função linear e depois a função não escalar~\cite{Srinivasan:2012}. Dessa forma, \citeonline{Srinivasan:2012} esperam melhorar a abordagem com robôs à vítimas de desastres em cenários de centros urbanos.

\citeonline{Okita:2012} realizaram um estudo para identificar quais fatores mais auxiliam na redução da distância física entre o robô e o ser humano. Foram utilizados dois tipos de abordagem para os testes realizados: (I) Robô com a iniciativa de se aproximar do ser humano; e (II) Humano com a iniciativa de se aproximar do robô.

Para o teste de ambos cenários foram utilizados dois tipos de indivíduos, separados em dois grupos diferentes, crianças e adultos. Na execução do teste, \citeonline{Okita:2012} utilizaram o método chamado de \emph{Wizard of Oz} (WOZ) que permite operar o robô através de um controle remoto distante da vista do indivíduo em interação. Dessa forma, é possível passar a impressão de que o robô é autônomo e ao mesmo tempo ter o controle dele para que não ocorra nenhum acidente durante a interação.

O experimento foi executado de duas maneiras diferentes sendo uma o robô aproximar-se sem nenhum tipo de aviso prévio e a outra maneira era exatamente avisar sua aproximação através de saudação via voz. Observou-se que quando o robô solicitava a permissão para aproximar do individuo o resultado sempre era positivo para a interação, quando comparado à aproximação sem aviso ou com aviso posterior a ação do robô~\cite{Okita:2012}.

Muitos trabalhos apontam maneiras de aplicar o estudo de \emph{Proxemics} em interações sociais. Algumas variáveis que podem afetar a interação são funções interpessoais de relacionamento, fatores fisiológicos moldados pela cultura de origem de um indivíduo, perspectivas etnológicas, além de informações sobre o ambiente de interação, como a luz ambiente, localização e ocupação física do agente, tamanho, entre outros fatores~\cite{Mead:2011b}.

Com a facilidade de compra dos sensores de captura de movimento não invasivo como o Microsoft\textregistered\ Kinect\textregistered ou o PrimeSensor\textregistered, os pesquisadores \citeonline{Mead:2011b} apresentam em seu trabalho um conjunto de métricas que são capazes de auxiliar na automatização do processo de análise do comportamento para distância social. As métricas por eles estabelecidas são: postura, posição do quadril, do ombro, do torso, dos braços, distâncias entre os agentes, gênero, entre outros.

Com base nas métricas definidas foi realizado um estudo conceitual para verificar se o cenário com um Kinect\textregistered\ inserido no ambiente, fosse capaz de capturar todas essas informações para que a partir delas, torna-se possível a criação de um mecanismo de análise automática do comportamento de agentes em um ambiente de interação social. Os testes preliminares possibilitaram a validação do cenário apresentado por \citeonline{Mead:2011b}.

Um cenário e ferramenta para coleta de informações sobre indivíduos em interação social são apresentados também por \citeonline{Mead:2011}. O principal objetivo é utilizar as informações coletadas em estudos futuros. Essas informações serviram para a criação de um modelo oculto de Markov (\emph{Hidden Markov Model} (HMM), em inglês) com seis classes para auxiliar na predição das \emph{Proxemics} de interação face a face. Nesse estudo, o HMM demonstrou-se com um desempenho superior para a tarefa de predição, quando comparado com um classificador aleatório ponderado por pesos~\cite{Mead:2011}.

Em um trabalho posterior, \citeonline{Mead:2012} apresentam uma discussão sobre os tipos de representações de variáveis para \emph{Proxemics}. Essas representações são: física e psicológica. Além desses dois tipos é proposto uma representação psicofísica que apresenta uma abordagem permitindo unir melhor as qualidades dos outros dois tipos de representação. A representação física tem como objetivo analisar como o espaço social é ocupado por dois indivíduos e é a abordagem mais comum em estudos de \emph{Proxemics}, tanto para interações humano-humano quanto para interações humano-robô. A representação psicológica mantém o foco em fatores de relacionamento interpessoal de alto nível entre dois ou mais indivíduos. Esse fatores estão relacionados a teoria de conflito afiliativo~\cite{Argyle:1965} e também a teoria de adaptação interpessoal~\cite{Burgoon:2007}.

Porém, com as lacunas existentes nesses dois tipos de representação de \emph{Proxemics}, foi proposto o tipo psicofísico. Os tipos psicofísicos tem como objetivo principal analisar a percepção e a produção de estímulo social entre dois ou mais indivíduos interagindo. A abordagem psicofísica é discutida também por~\citeonline{Hall:1969}. Essa representação está diretamente ligada com a experiência sensorial do estímulo social até os parâmetros espaciais de maneira física. A partir da representação psicofísica é realizado um estudo para capturar informações que servirão de base para treinamento de dois HMM. Cada HMM é responsável por uma exclusiva tarefa, início da interação ou término da interação. Essa representação deve auxiliar nas pesquisas de interação humano-robô, no intuito de que seja possível realizar uma análise para a interação ocorrer com maior qualidade~\cite{Mead:2012}.

Outro trabalho de \citeonline{Mead:2012b} apresenta um mecanismo de análise comportamental através da \emph{Proxemics}. Utiliza-se modelos probabilísticos de tal forma, que seja possível determinar alguns comportamentos dos indivíduos durante uma interação. Como métrica de proximidade utilizou-se a estratégia do mundo de grades para predizer a distância aproximada entre o robô e o individuo. Esse trabalho é implementado através de uma rede Bayesiana dinâmica como uma evolução para o mecanismo apresentado em trabalhos anteriores~\cite{Mead:2012b}.

Conforme tem sido discutido ao longo dessa seção, para que a interação entre um humano e um robô possa ocorrer de maneira confortável e com qualidade, ou seja atendendo as normas sociais, é necessário que o robô entenda as variáveis de espaço social. Além disso, é necessário também que ele possua o controle sobre essas variáveis de tal forma, que ele consiga tomar decisões sobre as ações que executará~\cite{Mead:2013b}.

\citeonline{Mead:2013b} apresentam um estudo baseado principalmente com variáveis de voz e gestos, utilizando um método de amostragem que tem como entrada a postura do indivíduo ao interagir com o robô, em outras palavras, a orientação do corpo representando assim uma linguagem corporal. A maior contribuição esperada por \citeonline{Mead:2013b} é a apresentação do entendimento obtido através das interações pré culturais que estão inseridas junto ao estudo de \emph{Proxemics}. O resultado apresentado é apenas uma base de dados para investigar todos os aspectos da interação humano-robô apresentadas no trabalho (voz e gestos).

A partir da base de dados gerada por \citeonline{Mead:2013b}, é apresentado outro trabalho onde \citeonline{Mead:2013} discutem a utilização de um HMM para extração de características comportamentais espaciais do ser humano (apresentadas na tabela~\ref{tab:variaveiscomportamentaismead}), em outras palavras, \emph{Proxemics}. Alguns fabricantes de sensores de movimentos, como o Microsoft\textregistered\ Kinect\textregistered\ e o ASUS\textregistered\ Xtion, têm pesquisado técnicas para aprimorar o estudo das distâncias sociais e também seus significados.

\begin{table}[!ht]
	\caption{Variáveis Comportamentais apresentadas por \citeonline{Mead:2013b}}
	\label{tab:variaveiscomportamentaismead}
	\centering
	\begin{tabular}{c | p{8 cm}}
		\hline
		\multicolumn{2}{c}{Características Individuais} \\
		\hline
		\emph{Stance Pose} & Orientação da postura do corpo \\
		\hline
		\emph{Hip Pose} & Orientação do quadril \\
		\hline
		\emph{Torso Pose} & Orientação do tronco \\
		\hline
		\emph{Shoulder Pose} & Orientação dos ombros \\
		\hline
		\emph{Head Pose} & Orientação da cabeça \\
		\hline
		\emph{Hip Torque} & Torsão do quadril \\
		\hline
		\emph{Shoulder Torque} & Torsão dos ombros \\
		\hline
		\emph{Head Torque} & Torsão da cabeça \\
		\hline
		\multicolumn{2}{c}{Características Físicas} \\
		\hline
		Distância Total & Calculada com base na distância euclidiana entre os dois agentes \\
		\hline
		\emph{Straight-Ahead Distance} & Magnitude no eixo X \\
		\hline
		Distância Lateral & Magnitude no eixo Y \\
		\hline
		Orientação Relativa do Corpo & Diferença da distância entre dois agentes \\
		\hline
		\multicolumn{2}{c}{Características Psicofísicas} \\
		\hline
		\emph{Distance Code} & Variável baseada na distância total \\
		\hline
		\emph{Socialfugal-SocioPetal Axis Code} & Variável baseada na orientação do corpo \\
		\hline
		\emph{Visual Code} & Variável baseada na posição da cabeça \\
		\hline
		\emph{Voice Loudness Code} & Variável baseada na distância total \\
		\hline
		\emph{Kinesthetic Code} & Variável baseada na distância entre a posição do quadril, tronco, ombros e braços \\
		\hline
		\emph{Olfaction Code} & Variável baseada na distância total \\
		\hline
		\emph{Thermal Code} & Variável baseada na distância total\\
		\hline
		\emph{Touch Code} & Variável baseada na distância total\\
		\hline
	\end{tabular}
	\smallcaption{Fonte:Adaptado de~\citeonline{Mead:2013b}.}
\end{table}

Com base nesses estudos, \citeonline{Mead:2013} analisam a possibilidade de automatizar o processo de análise das \emph{Proxemics}. A intenção do trabalho é extrair variáveis (vide tabela~\ref{tab:variaveiscomportamentaismead}) para que seja, então, possível determinar o início da interação, dado pela aproximação em direção do indivíduo, e o fim da interação social, que é o chamado de fuga social ou o afastamento para uma zona social mais pública sem intenção de uma reaproximação, através de um HMM. Para realizar os experimentos foram necessários dois indivíduos e um robô aplicados a um cenário de interação, onde os indivíduos se aproximam do robô sendo que os indíviduos estão separados por uma parede. Os resultados são apresentados em relação ao ponto de vista físico e psicológico. Na detecção das variáveis que representam as \emph{Proxemics} de maneira dinâmica, \citeonline{Mead:2013} consideram os resultados satisfatórios e como sequência do trabalho é mantido o foco em interações com fatores psicológicos complexos, para aprimorar a precisão do pacote criado para o ROS (\emph{Robot Operating Systems}).

\citeonline{Mead:2014} direcionam o foco de seu trabalho para a análise de conversa social e gestos, tanto na questão de produção das conversas e gestos de maneira automática quanto para o reconhecimento, aplicados em interações humano-humano e humano-robô. Todo o trabalho realizado está relacionado com o estudo de \emph{Proxemics} na interações sociais, uma vez que essas tem o objetivo de não só identificar as variáveis (vide tabela~\ref{tab:variaveiscomportamentaismead}), mas também de interpretar, manipular e compreender a dinâmica do comportamento espacial dentro do cenário das interações sociais.

Os estudos e experimentos sociais realizados por \citeonline{Mead:2014} auxiliaram na coleta de informações sobre o volume da fala de acordo com a distância, medida em polegadas, além dos gestos que necessitam de espaços maiores para execução sem prejudicar a interação. Os resultados apresentados apontam que a distância de interações entre humanos é menor que a distância da interação entre um humano e um robô. Além disso, os resultados obtidos não são aplicados à múltiplas culturas (nesse caso origem dos indivíduos), e isso deve ser realizado em outros trabalhos segundo \citeonline{Mead:2014}. Um mecanismo para personalizar o tratamento que o robô terá com o indivíduo durante a interação também é algo de deve ser construído ao longo dos trabalhos futuros.

Duas bibliotecas que colaboram para a execução de trabalhos com interação humano e robô, focadas em robótica assistiva e social, são apresentadas por \citeonline{Mead:2014a}. A primeira biblioteca é chamada de \emph{Social Behavior Library} (SBL) que tem o objetivo de prover os controles adequados para o robô saber como executar uma tarefa. Ela trata as questões do comportamento social do robô, como por exemplo, saber respeitar o espaço de uma pessoa. A segunda é a \emph{Social Interaction Manager} (SIM) que está focada em tomadas de decisão no mundo real. As técnicas aplicadas são para resolver problemas de reconhecimento, tomada de decisão e aprendizado, ou seja, informa ao robô o que ele deve fazer em uma determinada situação.

Um framework para trabalhar com proximidade entre humano e robô, é apresentado por \citeonline{Mead:2016}. O processo proposto é dividido em três etapas, que compõe este framework. A primeira etapa é a estimativa das ações para se aproximar e ficar com o robô frente a frente com o ser humano, pois é afirmado pelos autores que a melhor maneira de promover uma interação é posicionando os agentes olhando de frente um para o outro. Para realizar essa estimativa foi criada uma rede bayesiana que considera a posição relativa entre humano e robô, altura do som emitido pelo ser humano e o percebido pelo robô, e também os gestos emitidos e percebidos em cada zona de proximidade. Depois de estimar a posição final o robô realiza cálculos para determinar a distância que será percorrida, e também as diferenças entre os ângulos referentes a orientação do robô para pessoa e da pessoa para o robô. Isso resulta na diferença entre orientação atual para orientação objetivo. Como passo final, o robô realiza o planejamento de trajetória global que é ajustado ao longo do caminho considerando não só o ambiente ao redor, mas também o rastreamento dos movimentos da pessoa que ele está em busca da interação, já que é considerado que a pessoa está em constante movimento. Esse procedimento melhorou o encontro entre seres humanos e robôs dados face a face.

Definir um conjunto de variáveis para estudos com interação humano-robô é o foco do trabalho apresentado por \citeonline{Joosse:2011}. O conjunto de variáveis apresentados tem o objetivo de avaliar as respostas humanas durante a interação. O primeiro conjunto de variáveis apresentados, correspondem a medidas para atitudes pessoais, veja abaixo:

\begin{enumerate}
	\item \textbf{Aparência do robô com o humano}: o quão próximo de um ser humano o robô é. Essa é uma medida feita através de uma escala Likert de sete pontos, com tópicos como ``\emph{Human-made - Humanlike}'' ou ``sem gênero - masculino ou feminino''.
	\item \textbf{Atitudes em relação ao robô}: refere-se a atitudes negativas que o robô pode gerar socialmente. Também é medido através de uma escala Likert de 7 pontos.
	\item \textbf{Confiança no robô}: refere-se a confiança da pessoa no robô que interage com ela. Escala Likert de 7 pontos com credibilidade de fonte.
	\item \textbf{Aparente habilidade social do robô}: para medir as habilidades é utilizado uma escala likert de 5 itens chamada Wish \& Kaplan. Esse método originalmente possui uma escala bipolar de 9 pontos, porém foi revertida em uma escala Likert de 7 pontos.
	\item \textbf{Atração física e social do robô}: inclui-se itens de escala de atração interpessoal, que originalmente possuem uma escala de Likert 7 pontos com 15 itens. Porém foi realizado uma adaptação com a exclusão de 5 itens, matendo apenas 10 itens que representam as atrações físicas e sociais.
\end{enumerate}

Além da lista sobre atitudes pessoais, \citeonline{Joosse:2011} apresentam medidas comportamentais que são preenchidas pelo especialista através da observação do video feito durante o teste. Nas medidas comportamentais ele deve marcar se houve repostas de linguagem corporal como se afastar do robô ou se houve um inclinação em direção ao robô. Também são inclusas expressões faciais como sorrindo ou se parecia assustado.

Para validar a proposta de \citeonline{Joosse:2011} foi utilizado um cenário, onde o ser humano fica parado em frente a alguns quadros, observando-os, e após um certo período de tempo o robô se aproxima com uma determinada velocidade. Um video é gravado com todo o processo do teste. Após o processo de teste concluído, os videos são analisados e o questionário criado é preenchido e depois quantificado para realizar a análise estátistica.

Durante as análises realizadas percebeu-se que as habilidades sociais não interferiam nas reações do ser humano durante a interação. Dessa maneira, todos os pontos sobre habilidades sociais foram removidos do questionário. Apesar de terem removido os pontos sobre habilidades sociais do robô, \citeonline{Joosse:2011} acreditam que é um ponto importante e deve ser investigado com mais calma, pois pode apresentar resultados importantes para o processo de interação.

A questão de reconhecimento de expressão facial e características faciais de movimento podem ser úteis no futuro para ajuda na avaliação das emoções das pessoas. \citeonline{Joosse:2011} esperam que este arquabouço de questões, seja utilizado em mais pesquisas para tentar fazer com que ele torne-se mais robusto.

Analisando a diferença de cultura para variáveis de \emph{Proxemics}, \citeonline{Eresha:2013} apresentam como objetivo do trabalho a avaliação do comportamento de indivíduos ao se encontrarem com dois robôs interagindo entre si e caminhando em direção ao individuo de tal forma que este possa também interagir ou não com os robôs conforme a aproximação do ser humano. Além de avaliar o comportamento dos indivíduos durante a interação com os robôs, \citeonline{Eresha:2013} adicionaram a variável de cultura ao estudo. O objetivo é identificar como é a diferença de comportamento entre culturas diferentes. Foram escolhidos participantes de origem árabe e alemã para o estudo.

Nos experimentos, \citeonline{Eresha:2013} utilizaram dois robôs NAO que se posicionavam a 40 cm de distância entre eles e caminhavam até ficarem a uma distância diagonal de 85 cm do indivíduo. Para o experimento houve a participação de 24 indivíduos, 12 árabes e 12 alemães, sendo metade do gênero feminino e a outra metade do gênero masculino. Os testes apresentaram resultados interessantes, pois alguns indivíduos não reagiram como o esperado para pessoas de sua origem e muitas vezes o comportamento social na interação era idêntico entre alemães e árabes. Outro ponto apresentado por \citeonline{Eresha:2013} é que durante os testes dois alemães apresentaram o sentimento de medo de serem atacados fisicamente pelos robôs.

O trabalho de \citeonline{Eresha:2013} apresenta indícios de que as variáveis de \emph{Proxemics} não estão ligadas a cultura do indivíduo, como origem, mas sim na experiência cultural que este teve ao longo de sua vida. Dessa maneira, pode-se dizer que \emph{Proxemics} são variáveis extraculturais, porém é necessário realizar um tratamento para esse tipo de condição verificando a origem do indivíduo, onde o robô possa interagir com mais qualidade com pessoas que possuem diferentes experiências culturais.

\citeonline{Henkel:2012} investigam características entre diversas plataformas de teste para interação humano-robô, e com base no resultado deste estudo é realizado a proposta de uma nova plataforma de testes. A nova plataforma foi desenvolvida, pois \citeonline{Henkel:2012} alegam que não existe nenhuma plataforma de teste capaz de atender aos seis atributos de dependência das \emph{Proxemics}. Os atributos são: (I) movimento afetivo; (II) leitura das \emph{Proxemics}; (III) interação de voz; (IV) manipulação do estilo de áudio; (V) controle do olhar; e (VI) apresentação de conteúdo através de mídia, por exemplo, monitor ou leds.

A plataforma é constituída por uma cabeça feita com um monitor de 7'', junto com um encaixe construído para ser acoplado em qualquer base de robôs já existentes no mercado. Alguns testes que foram realizados no cenário de resgate à vítimas, demonstram que as pessoas que tinham a zona de espaço social íntimo invadida por qualquer parte do robô, sem uma interação prévia, ficavam em situação de \emph{stress} elevado. Essa reação foi totalmente oposta quando o robô iniciava com qualquer tipo de interação antes de realizar a aproximação do indivíduo~\cite{Henkel:2012}. O primeiro contato antes da aproximação para uma interação maior é importante, pois esse comportamento pode definir o quão confortável a interação entre os agentes será e esse comportamento deve ser explorado durante a execução dessa tese.

Em outro trabalho, \citeonline{Henkel:2014} apresentam duas funções escalares para avaliar os valores de proximidade entre humanos e robôs. As funções escalares são comparadas com outras funções não-escalares e também entre si de tal forma, que seja possível uma tomada de decisão em tempo de execução da ação/interação. As duas funções escalares apresentadas são: (I) logarítmica; e (II) linear.

Os testes foram executados no cenário de regaste à vitimas. Quando a função logarítmica foi aplicada, os resultados apresentados foram melhores do que os obtidos com as demais funções. Como o principal objetivo de \citeonline{Henkel:2014} é generalizar o método para outros cenários, eles pretendem realizar testes do modelo em outras situações e também utilizando outros tipos de robôs para sustentar melhor a hipótese. Os estudos prévios realizados demonstram que a generalização do modelo é possível.

A integração social do robô com os ambientes que envolvem cenários de cuidados médicos, construção, educação, serviço públicos, entre outros pode ser a chave de sua aceitação por parte dos seres humanos. Um dos caminhos para conseguir esse objetivo é fazer com que o robô saiba ter um comportamento adequado de interação em cada um desses cenários, assim como o que já é demonstrado em filmes de ficção científica. Dessa maneira, é possível fazer com que os seres humanos utilizem o próprio senso social para identificar essas habilidades no robô, quebrando um pouco o medo de interagir com ele~\cite{Heenan:2014}.

Como primeiro passo para que a interação ocorra naturalmente entre o ser humano e o robô, \citeonline{Heenan:2014} acreditam que deve haver sempre uma saudação entre ambas partes logo ao primeiro contato. Esse tipo de comportamento pode ser fundamental para que haja uma aceitação social do robô entre as pessoas. Durante uma saudação existem diversos fatores que são analisados implicitamente pelo ser humano, como nuanças de comunicação não verbal, vocalização das palavras e a distância inter pessoal. Esses fatores devem ser considerados ao projetar uma saudação por parte do robô, fazendo com que seja possível o robô iniciar a interação.

Fazer com que um robô realize uma saudação natural não é uma tarefa muito fácil. Deve ser considerado que um robô não tem a mesma capacidade de identificar as nuanças sociais com a mesma velocidade de um ser humano. Outro ponto negativo é que o robô possui o lado mecânico limitado, quando comparado a musculatura do ser humano. Assim, o primeiro objetivo do trabalho de \citeonline{Heenan:2014} é definir um subconjunto exato de elementos de uma saudação social que possa ser articulado pelo robô durante a tarefa e ainda como implementar as sutilezas do comportamento da interação de saudação social.

Os testes executados demonstram que a saudação é um ponto importante para o resultado com sucesso da interação com o ser humano. O robô NAO utilizado nos testes foi capaz de implementar ações de comportamento como o contato visual, linguagem corporal e distância social para comunicação efetiva. Apesar de algumas restrições do modelo de saudação ocorrerem devido a limitação do NAO, é possível realizar a generalização do mesmo para outros robôs~\cite{Heenan:2014}.

Percebeu-se que o contato visual se apresentou como um elemento de interação social bem natural, contudo deve-se tomar cuidado para que o robô não fique encarando a pessoa constantemente, pois é gerado um desconforto para a pessoa durante o contato. \citeonline{Heenan:2014} dizem que é possível afirmar que utilizar a saudação é importante no primeiro contato de dois agentes, além de aumentar a capacidade da interação social entre o robô e o ser humano.

\citeonline{Vazquez:2014} apresentam um robô móvel no formato de mobília, chamado Chester, construído para realizar interações com crianças. Como o Chester é muito grande optou-se por usar um segundo robô não móvel, ao qual \citeonline{Vazquez:2014} denominam \emph{sidekick}. O \emph{sidekick} é como um parceiro ou personagem secundário que auxilia as pessoas em volta a prestarem atenção no personagem principal, como por exemplo o burro da animação Shrek. O \emph{sidekick} criado é um abajur chamado Blink. Ele fica acoplado em cima do Chester. A figura~\ref{fig:vazquez} apresenta a combinação dos robôs Chester e Blink.

\begin{figure}[ht!]
	\centering
	\begin{minipage}{0.4\textwidth}
		\caption{Os robôs Chester e Blink.}
		\includegraphics[width=\textwidth]{vazquez2014.png}
		\smallcaption{Fonte:~\citeonline{Vazquez:2014}.}
		\label{fig:vazquez}
	\end{minipage}
\end{figure}

Blink tem uma linguagem própria e apenas o Chester é capaz de entender. É como o R2D2 em Star Wars que apenas alguns personagens são capazes de compreende-lo e falar com ele diretamente. Os resultados obtidos mostram que a inserção de um \emph{sidekick} não altera a questão de proximidade das crianças em relação ao robô, mas melhora a atenção com os elementos falantes do cenário~\cite{Vazquez:2014}.

Foi possível caracterizar alguns comportamentos das crianças ao interagir com os robôs. É afirmado por \citeonline{Vazquez:2014} que o formato de mobília para robôs é plausível para utilizar em robôs que interagem com crianças, pois elas se sentem mais empáticas aos robôs. Contudo, é questionável essa afirmação. Será que o que realmente influenciou esse resultado foi o formato do robô ou foi seu comportamento durante o contato com as crianças? Provavelmente, esse é um resultado que pode ser obtido com a mistura desses dois fatores, aparência e comportamento.

Por questões de segurança os testes foram executados utilizando o método \emph{Wizard of Oz} (WoZ), onde existe um especialista controlando o robô através de um controle de videogame, por exemplo, dando a sensação de robô completamente autônomo. Foram conduzidos duas variantes do teste, são elas: (I) com o \emph{sidekick} ativo; e (II) com o \emph{sidekick} inativo. O especialista que controla o robô encontrava-se na mesma sala de teste, mas algumas precauções foram consideradas para que não houvesse ruído nos resultados do teste. Uma dessas precauções foi inseri-lo na sala do teste antes do mesmo iniciar para que aparenta-se que ele estava apenas trabalhando normalmente. Além disso, o controle do robô foi posicionado embaixo da mesa para facilitar a oclusão do objeto e ainda fez com que nenhuma criança notasse que o robô era teleoperado por um especialista~\cite{Vazquez:2014}.

Para capturar as informações de distância foi acoplado ao teto um sensor Microsoft Kinect. Ele é responsável por capturar as informações de distância entre o robô e a criança interagindo com ele. Notou-se que na maioria das vezes a criança ficava sempre de frente a face do robô e não ao seu lado ou atrás dele. Variáveis como o tempo de resposta para se afastar enquanto o robô dizia ``recue'' também foi considerado para identificar os resultados~\cite{Vazquez:2014}.

Nos resultados finais, \citeonline{Vazquez:2014} encontraram algumas limitações do robô e também do experimento, como por exemplo, o pouco conteúdo de linguagem que o robô possui implementado para dar respostas aos participantes do teste. Outro problema encontrado foi no início e no final de interação onde outros pontos do cenário e tarefa atrapalharam a coleta de informações ou melhor o foco do caso de estudo. Devido a esse problema, a utilização de um \emph{sidekick} deverá ser estuda com mais detalhes e realizar os testes novamente para que possa ser comprovado o real benefício dele nos resultados da interação. Resultados preliminares confirmam que o \emph{sidekick} não atrapalha na interação entre o robô principal e as pessoas e ainda auxilia a aumentar a atenção das pessoas o que auxilia em um melhor comportamento reativo dos participantes~\cite{Vazquez:2014}.

Alguns estudos utilizando robôs para interagir com crianças com autismo apontam que pode apresentar reações positivas e negativas para o âmbito social. Especialistas são capazes de identificar esse tipo de avaliação através da análise dos vídeos gravados entre sessões. O objetivo do trabalho de \citeonline{Feil-Seifer:2010} é automatizar esse processo de análise através do uso de robôs. Para isso foi desenvolvido um classificador heurístico que utiliza um conhecimento prévio para auxiliar à discretizar as crianças que conseguem interagir com o robô daquelas que não conseguem.

O cenário de teste é composto de uma sala, um robô totalmente autônomo com o objetivo de incentivar a interação, uma criança diagnosticada com autismo e um familiar mais próximo. Para incentivar a interação o robô deve se aproximar apresentando vocalizações de sons felizes e também esboçar um sorriso para a criança, por exemplo. Caso alguma criança se afaste do robô, ele deve esboçar uma face triste e emitir sons que demonstre a sua não felicidade~\cite{Feil-Seifer:2010}.

Durante os testes foram gravados vídeos e algumas marcações foram realizadas no robô, e nos pais, com o intuito de auxiliar na medida das distâncias entre a criança e o robô ou seus pais. Para realizar uma avaliação sobre esse cenário foi utilizada a seguinte heurística: Para cada trecho de tempo se a criança encontrar-se a 0,85 m dos pais ela é considerada próxima à eles. Caso ela encontra-se a 0,5 m de uma parede ela é considerada próxima a parede. Para ser considerada atrás do robô ela deveria estar a qualquer distância, mas entre uma angulação maior que 135º e menor que -135º~\cite{Feil-Seifer:2010}.

A partir das informações capturadas é possível gerar o classificador onde ele análise se pelo menos 50\% do tempo gasto é com as informações de comportamento negativo (mapeado pelas heurísticas), então é considerado que a criança não deseja interagir com o robô. Caso contrário, menos de 50\% do tempo gasto, a criança deseja interagir com o robô. Apesar dos resultados positivos, esse classificador não deve ser considerado como regra para que haja uma maior escalabilidade do projeto e sua aplicação~\cite{Feil-Seifer:2010}. Esses tipos de parâmetros podem auxiliar na determinação de interação ou não interação. Dessa forma, pode-se fazer com que o robô recue ou tente uma nova abordagem, para quando a reação do indivíduo for negativa.

Outros estudos confirmam a existência de uma relação de distância social entre o robô e o ser humano, entretanto nenhum método foi proposto computacionalmente para que haja uma geração do comportamento em relação a essa distância~\cite{Henkel:2012b}. Assim, é apresentado um método escalar do comportamento do robô de tal forma, que esse comportamento baseado na distância social tenha como suporte uma lei física e duas psicológicas: \emph{inverse-square law}, \emph{Weber-Fechner law} e \emph{Steven's Power law}~\cite{Henkel:2012b}.

O cenário de teste é um ambiente de desastre no qual o robô deve localizar a vítima. A interação ocorre por meio de voz sintetizada, caminhos pre definidos e controle segundo o módulo de teste WoZ. Como meio de avaliação questionários pré e pós interação são aplicados aos usuários que participam do teste~\cite{Henkel:2012b}.

Atributos primários foram determinados para que possam ser identificados alguns níveis de consistências sociais: conforto, movimentos naturais, consideração do espaço pessoal, segurança e controle próprio. Atributos secundários também foram considerados nos estudos de \citeonline{Henkel:2012b}, são eles: atenciosidade, empatia, felicidade, similaridade, inteligência, sensibilidade, submissão e confiança. Os resultados demonstram que todos atributos primários e apenas três secundários provaram que apresentam melhor significância para o processo. O sistema de percepção escalar provou ser melhor do que o não escalar. O modelo escalar linear apresentou o mesmo resultado que o não escalar~\cite{Henkel:2012b}.

\citeonline{Brooks:2007} apresentam um modelo comportamental em camadas para auxiliar no processo de comunicação não verbal do robô com o ser humano. Um comportamento em camadas é definido como um processamento de saída a nível dos motores (atuadores) do robô, onde é realizado uma sequência de passos para deixar o movimento do robô mais natural ao do ser humano. Neste trabalho é discutido uma generalização do modelo para que seja possível aplicar em diversas plataformas robóticas e fontes de informação.

No trabalho é explorado a questão de linguagem corporal, pois expressões faciais já são amplamente utilizadas para estudos de interação humano e robô. \citeonline{Brooks:2007} apresenta a implementação funcional de seu modelo em um robô Sony QRIO. Alguns dos resultados apresentados são que o tamanho do robô pode influenciar no tipo de interação que ocorrerá, que a utilização da teoria de proximidade é essencial para que ocorra uma boa interação e que comunicações não verbais são a principal forma de comunicação em uma interação humano e robô.

Um dos pontos chave dos trabalhos apresentados ao longo dessa seção é que sempre utilizam sensores no ambiente para medir as variáveis de \emph{Proxemics} entre a pessoa e o robô. Contudo, acredita-se que esse tipo de abordagem não é natural ao robô móvel, pois os seres humanos não tem o auxílio sendo assim o robô também não deve utilizar desses recursos. Contudo, as variáveis de \emph{Proxemics} se mostram essenciais para determinar o sucesso de uma interação ou não, e devem ser consideradas ao longo da proposta desta tese de doutorado.

Dessa forma, todas as variáveis apresentadas nos trabalhos dessa seção são importantes para avaliar o comportamneto de um indivíduo durante uma interação com robôs e até outros dispositivos tecnológicos. Na seção~\ref{sec:extracaocaracteristicas} são apresentados as variáveis consideradas para o desenvolvimento desse trabalho. Além das variáveis, também são avaliados os meios de captura das informações visando a aplicação do trabalho desenvolvido na tese inserido em um ambiente inteligente.

% %!TEX root=Principal.tex
\chapter{APRENDIZADO EM ÁRVORE DE DECISÃO}
\label{cap:dectree}

Nesse capítulo será abordado o conceito de árvore de decisão e seu aprendizado para auxiliar na descoberta das variáveis mais relavantes ao problema de aproximação.

%!TEX root=Principal.tex
\chapter{APRENDIZADO}
\label{cap:ai}

Técnicas previtas:

- Redes Bayesianas
- Redes Neurais
- Neuro-Fuzzy

% %!TEX root=Principal.tex
\chapter{RACIOCÍNIO BASEADO EM CASOS}
\label{cap:rbc}

Raciocínio Baseado em Casos (RBC) é uma metodologia utilizada em Inteligência Artificial que constrói e utiliza um sistema de base de conhecimento criado a partir de experiências passadas. Esse tipo de comportamento é basicamente o comportamento que os seres humanos têm ao solucionar problemas similares com diferentes experiências obtidas ao longo de suas vida~\cite{Lopez:2013}.

O RBC tenta encontrar sempre uma solução, igual ou parecida com a situação atual, em sua base de conhecimento. Ao encontrar essa solução ou possível solução, o RBC tenta adapta-la da melhor maneira à tarefa atual. Caso não seja identificado nenhum caso similar ao atual sendo a situação totalmente nova é possível inferir uma sequência de ações para o caso e armazena-la para que possa ser consultado ao longo do seu ciclo de vida~\cite{Lopez:2013}.

A metodologia do RBC apresenta quatro fases principais. Essas fases são apresentadas na figura~\ref{fig:rbcciclo}~\cite{Lopez:2013}.

\begin{figure}[h!]
	\centering
	\includegraphics[scale=0.5]{images/cbr-cycle.png}
	\caption{Ciclo da Metodologia de um sistema de RBC.}
	\label{fig:rbcciclo}
\end{figure}

A seguir são apresentadas as fases da figura~\ref{fig:rbcciclo} em detalhes:

\begin{enumerate}
	\item \textbf{Resgatar}: Nessa fase são resgatados casos passados similares ao caso procurado. Métodos indutivos são aplicados normalmente para encontrar os casos armazenados em memória.
	\item \textbf{Reutilizar}: A partir de um conjunto de soluções similares resgatadas formam a base para a construção de uma solução ao problema procurado. A transformação que pode ser obtida nessa fase, ocorre através de generalização e especialização dos casos resgatados.
	\item \textbf{Revisar}: Verificar se a solução apresentada para o problema procurado provem ou não a saída desejada.
	\item \textbf{Reter}: Quando a solução apresenta ao problema gera uma nova experiência, esta pode ser ou não armazenada na base de casos. Para que o armazenamento seja realizado, é confrontado o valor obtido como resposta no processo de revisão do caso, junto com a política de retenção implementada no sistema.
\end{enumerate}

O modelo apresentado na figura~\ref{fig:rbcciclo} é baseado no nível de conhecimento que sistema pode representar. Ele identifica todo o ciclo da metodologia de RBC, do começo ao fim. Esse modelo também é conhecido como 4R’s, pois os nomes dados para cada uma das fases são iniciados com a letra R~\cite{Lopez:2013}.

%%%%%%%%%%%%%%%%%%%%%%%%%%%%%%%%%%%%%%%
\section{Classificação de sistemas de RBC}
\label{sec:classificacaorbc}

Para que sejam construídos, os sistemas de RBC consideram quatro critérios básicos que definem sua topologia~\cite{Lopez:2013}: Fonte de Conhecimento, Função, Organização e Maneira de Distribuição. A tabela~\ref{tab:classificaorbc} apresenta os possíveis tipos para cada um dos critérios.

\begin{table}[h!]
	\centering
	\caption{Classificação de Sistemas de RBC de acordo com sua topologia}
	\begin{tabular}{c | c | c | c} \hline
		Fonte de & Função & Organização & Maneira de \\ 
		Conhecimento & & & Distribuição \\ \hline
		- Textual & - Classificação & - Exclusivo & - Memória Única \\ 
		- Estrutural & - Recomendação & - Níveis Múltiplos & - Memória Múltipla \\
		- Conversacional & - Tutoria & - RBC Híbrido & - Agente Único \\
		- Temporal & - Planejamento & - Meta RBC & - Agente Múltiplo \\
		- Imagens & - Monitoramento & & \\
		& - Gerenciamento de & & \\ 
		& Conhecimento & & \\ \hline
	\end{tabular}
	\label{tab:classificaorbc}
\end{table}

O primeiro atributo apresentado na tabela~\ref{tab:classificaorbc}, Fonte de Conhecimento, refere-se em como é estrutura ou armazenamento do conhecimento passado no sistema. São eles~\cite{Lopez:2013}:

\begin{enumerate}
	\item \textbf{Textual}: É uma coleção de informações em formato de texto, que devem ser lidas pelo sistema afim de identificar a solução do problema. Um exemplo são sistemas do tipo FAQ, que podem ser encontrados em diversas ferramentas e aplicativos.
	\item \textbf{Estrutural}: É definido através de um vocabulário pré definido do problema. Cada caso pode englobar uma quantidade n de variáveis, como por exemplo, um sistema médico que possui informações como idade, histórico de atendimento, índice de massa corpórea, entre outros.
	\item \textbf{Conversacional}: Nesse tipo de RBC o caso é definido maneira iterativa através de um sistema de conversa entre usuários e/ou usuário-sistema, como por exemplo, um sistema de help-desk.
	\item \textbf{Temporal}: Quando os casos possuem uma relação temporal entre si, seja essa relação implícita ou explicita. Um exemplo desse tipo de sistema é o histórico de jogo de um usuário.
	\item \textbf{Imagens}: Cada caso é obtido através do conhecimento extraído na análise das imagens armazenadas no banco de dados. Essa análise ocorre de acordo com alguns fatores relevantes ao domínio do problema e encontrados nas imagens.
\end{enumerate}

O atributo Função refere-se ao objetivo de aplicação do RBC, ou seja, qual o tipo de problema que ele procura resolver. Abaixo a descrição detalhada de cada uma das funções que um sistema de RBC pode apresentar~\cite{Lopez:2013}:

\begin{enumerate}
	\item \textbf{Classificação}: Utilizado geralmente em aplicações onde existe a necessidade de predizer uma classe ou rótulo. Pode ocorrer de duas maneiras: (I) Através da prognosis sendo feita por duas classes, uma positiva e outra negativa; ou (II) Através do diagnóstico sendo feita por um número discreto de classes para predição.
	\item \textbf{Recomendação}: Utilizado para recomendar algum tipo de produto baseado na similaridade das escolhas passadas do usuário.
	\item \textbf{Tutoria}: Utilizado para buscar uma coleção de exercícios de uma determinada disciplina e assim auxiliar o usuário.
	\item \textbf{Planejamento}: Utilizado em duas tarefas normalmente.  A primeira é deixar o planejamento de um processo ou sistema mais eficiente e a segunda tarefa é realizar o raciocínio e aprendizado a nível de planejamento e não a nível da ação de execução como os demais RBC's.
	\item \textbf{Monitoramento}: Utilizado para predizer anomalias de comportamento durante a supervisão de sistemas, principalmente.
	\item \textbf{Gerenciamento do Conhecimento}: Utilizado para fontes de informação e avaliação do conhecimento, não só lembrando da experiência obtida, mas também aplicando esse conhecimento na solução de tarefas.
\end{enumerate}

O atributo Organização refere-se a combinação de RBC para solução de problemas em um mesmo domínio. A Organização também pode ser encontrada como arquitetura do sistema de RBC. As quatro principais são~\cite{Lopez:2013}:

\begin{enumerate}
	\item \textbf{Exclusivo}: Nessa configuração um único RBC é considerado para resolver o problema.
	\item \textbf{Níveis Múltiplos}: Nessa configuração é considerado um determinado número de RBC para solucionar o problema. Uma aplicação desse tipo de arquitetura é a interpretação de imagens.
	\item \textbf{RBC Híbrido}: Essa configuração é aplicada em problemas que necessitam de soluções híbridas ou complementares. Um exemplo de aplicação é o sistema que realiza o diagnóstico de uma doença combinado com outro sistema que realiza o planejamento para o tratamento dessa doença.
	\item \textbf{Meta RBC}: Essa configuração é utilizada quando existe vários sistemas de RBC que necessitam ser aplicados à um único domínio e é utilizado um segundo sistema RBC que verifica qual é o melhor sistema para ser aplicado dado um determinado caso.
\end{enumerate}

É importante ressaltar que o RBC é uma metodologia e não uma tecnologia. Portanto, cada fase dele pode ser resolvida por diversas técnicas de Inteligência Artificial~\cite{Lopez:2013}.

O último atributo que é apresentado na tabela~\ref{tab:classificaorbc} chama-se Maneira de Distribuição. Ele é caracterizado pela maneira como será feita o processamento dos casos. São classificados em dois critérios apenas: (I) Memória ou o número de casos existentes na base de conhecimento; e (II) Forma de distribuição do processamento, sendo possível em um único sistema ou em múltiplos sistemas. Dessa maneira, é obtido quatro possíveis combinações, são elas~\cite{Lopez:2013}:

\begin{enumerate}
	\item Memória Única, Agente Único;
	\item Memória Única, Agente Múltiplo;
	\item Memória Múltipla, Agente Único;
	\item Memória Múltipla, Agente Múltiplo.
\end{enumerate}

%%%%%%%%%%%%%%%%%%%%%%%%%%%%%%%%%%%%%%%
\section{A Base de Casos ou Conhecimento}
\label{sec:basecasos}

A base de casos é formada por quatro fatores principais: (I) Vocabulário; (II) Casos; (III) Medida de Similaridade; e (IV) Adaptação da Solução. Outros fatores podem ser incluídos na base de conhecimento de acordo com a necessidade do projeto ou aplicação. Ao longo dessa seção será discutido como modelar e organizar o caso na base de conhecimento~\cite{Lopez:2013}.

%%%%%%%%%%%%%%%%%%%%%%%%%%%%%%%%%%%%%%%
\subsection{Vocabulário}
\label{sec:vocabulario}

O vocabulário é um conjunto de termos que servem como base na formulação do caso que será inserido na base de conhecimento. Cada termo do vocabulário mantém o mesmo significado, independente da maneira como foi mapeado, para o domínio da aplicação. Em trabalhos mais recentes, verifica-se que não existe uma maneira de compreender um vocabulário sem a utilização de ontologias. Uma ontologia nada mais é que o relacionamento entre as palavras do vocabulário. Esse recurso auxilia a metodologia de RBC de tal forma, que seja possível identificar o relacionamentos entre os elementos de um caso a nível semântico~\cite{Lopez:2013}.

%%%%%%%%%%%%%%%%%%%%%%%%%%%%%%%%%%%%%%%
\subsection{Modelando um Caso}
\label{sec:casemodeling}

Um caso é a instância da descrição do processo que envolve o problema e a solução. Ele precisa descrever de maneira clara o problema e a sua solução de maneira separada e detalhada. Um maneira de representar um caso é através de uma tupla formada por $<p, s, o>$, onde $p$ é a descrição do problema, $s$ representa a descrição da solução e $o$ é a saída espera da aplicação do processo. Existem diversas maneiras de descrever um problema, as principais são através de um modelo atributo-valor ou através do relacionamento entre os objetos. O modelo atributo-valor é mais simples e permite, de acordo com o peso especificado, ignorar alguns atributos na hora do cálculo da similaridade~\cite{Lopez:2013}.

No modelo de relacionamento entre os objetos, os casos são visualizados como um grafo ou uma árvore mostrando a similaridade entre os casos de acordo com o grau do relacionamento. Modelos mais complexos também podem ser utilizados para determinar o caminho do relacionamento. Além disso, séries e sequências também podem ser utilizadas fazendo com que o modelo fique dependente da variável tempo. Quanto aos mapeamentos referentes a solução, geralmente são utilizados algoritmos de predição ou classificação para essa tarefa~\cite{Lopez:2013}.

%%%%%%%%%%%%%%%%%%%%%%%%%%%%%%%%%%%%%%%
\subsection{Medida de Similaridade}
\label{sec:medidasimilaridade}

Geralmente, utiliza-se um algoritmo que efetua a comparação de características de tal forma, que seja possível gerar um grau de similaridade, onde os valores se mantenham entre 0.0 e 1.0. Cada caso armazenado deve possuir um grau de similaridade quando comparado a outro caso da base. Para que o algoritmo chegue a esse valor de similaridade é necessário utilizar duas medidas de similaridade. A primeira medida de similaridade, conhecida como Similaridade Local, refere-se à comparação de uma única característica dos casos. Alguns dos métodos que podem ser usados aqui são~\cite{Gresse:2003}:

\begin{enumerate}
	\item \textbf{Correspondência exata}: Caso as strings sejam escritas da mesma forma. Por exemplo, ``\emph{e-commerce}'' e ``\emph{e-commerce}'' retorna 1.0, enquanto ``\emph{e-business}'' e ``\emph{e-commerce}'' retorna 0.0.
	\item \textbf{Correção ortográfica}: Ponderação entre o número de caracteres que são iguais pelo número total de caracteres. Por exemplo, ``menino'' e ``menina'', a quantidade de caracteres idênticos: 5. Quantidade total de caracteres: 6. Portanto, a similaridade é dada por: 5/6 = 0,83.
	\item \textbf{Contagem de palavras}: Utilizado em textos maiores. Faz um contagem por palavras idênticas nos casos comparados.
	\item \textbf{Taxa do maior \emph{substring} comum}: Taxa entre a maior sequência de caracteres entre os dois casos pelo número total da consulta.
\end{enumerate}

A outra medida de similaridade é conhecida como Similaridade Global. Essa similaridade gera um valor único referente à comparação de todas as características dos casos. Essas características, ou atributos, podem ter pesos, classificando-o como maior importância ou significância no domínio aplicado. Para ter maior eficiência, a busca pode ser feita em uma base de casos indexada o que auxilia os algoritmos de busca dos casos. A equação~\ref{eq:simglobal} apresenta o cálculo do algoritmo \emph{Nearest Neighbour} (vizinho mais próximo)~\cite{Gresse:2003}:

\begin{equation}
	Sim (X, Y) = \cfrac{\sum W_i * sim(q_i, c_i)}{\sum W_i}
	\label{eq:simglobal}
\end{equation}

\begin{flushleft}
	Aonde:
	\begin{enumerate}
		\item $W_i$ é o peso do atributo.
		\item $Sim (X, Y)$ é a similaridade global entre os casos $X$ e $Y$.
		\item $sim (q_i, c_i)$ é a similaridade local entre o atributo $q_i$ de $X$ e o atributo $c_i$ de $Y$.
	\end{enumerate}
\end{flushleft}

Além das medidas de similaridade apresentadas por \citeonline{Gresse:2003}, pode ser utilizados diversos outros métodos para calcular a similaridade local e global entre os casos. Uma medida de similaridade utilizada entre os algoritmos é a distância euclidiana~\cite{Masiero:2013}. Outras medidas de distância também são utilizadas para essa tarefa como a distância de Mahalanobis~\cite{Mahalanobis:1936}. Contudo, outras técnicas também são capazes de determinar o valor da similaridade entre os casos, como no caso da Lógica Nebulosa ou Lógica Fuzzy~\cite{Lopez:2013}.


%%%%%%%%%%%%%%%%%%%%%%%%%%%%%%%%%%%%%%%
\subsection{Adaptação da Solução}
\label{sec:adaptacaosolucao}

Após a definição do caso com a melhor solução de acordo com a similaridade entre os problemas apresentados, é necessário realizar a adaptação da solução do caso. Existem diversas técnicas de adaptação para a metodologia do RBC. \citeonline{Gresse:2003} apresentam a lista a seguir como as principais existentes:

\begin{enumerate}
	\item \textbf{Adaptação Nula}: quando a solução do caso selecionado é a solução do caso buscado.
 	\item \textbf{Adaptação Transformacional Substitutiva}: Quando o caso possui a mesma estrutura, esta adaptação substitui informações baseadas em regras pré-definidas.
	\item \textbf{Adaptação Derivacional}: Obtém informações sobre como um caso foi construído para que outro seja elaborado. Ou seja, nesta adaptação não se utiliza sua solução.
\end{enumerate}

%%%%%%%%%%%%%%%%%%%%%%%%%%%%%%%%%%%%%%%
\section{Revisão dos Casos}
\label{sec:revisaocasos}

Uma solução resgatada não reflete o sucesso do caso, então inicia-se o processo de revisão desta para que seja decidido se essa irá permanecer na base de casos ou se deverá ser descartada. Para que o processo de revisão ocorra são necessárias duas tarefas. A primeira tarefa consiste na avaliação criteriosa da solução dada a partir do processo de reuso. Nessa etapa, caso seja considerada correta, a solução é armazenada na base de casos através do processo de retenção, que será discutido mais a frente~\cite{Gresse:2003}.

A segunda tarefa da fase de revisão é a tentativa de aprimorar a solução para que ela seja correta para o caso em questão. Para que isso seja possível, o sistema ou até mesmo o especialista deve utilizar o conhecimento específico sobre aquele domínio para alterar a solução. Dessa forma, é possível manter os casos dentro da base com todas as possíveis soluções~\cite{Gresse:2003}.

%%%%%%%%%%%%%%%%%%%%%%%%%%%%%%%%%%%%%%%
\subsection{Avaliando a Solução}
\label{sec:avaliandosolucao}

O processo de avaliação de uma determinada solução é dada a partir do uso parcial ou total dela no processo de aplicação em um novo caso similar. Esse processo pode ser realizado pela monitoração dos resultados da aplicação da solução no caso do mundo real, sendo a monitoração automática (através de sensores) ou através da interação do usuário a partir de um retorno sobre o quão válida é a solução para àquele caso~\cite{Gresse:2003}.

Como a avaliação depende da observação da solução aplicada ao novo caso, ela geralmente ocorre em um processo apartado do sistema de RBC como um todo. Isso ocorre, pois, dependendo da situação, a avaliação pode levar alguns meses para que seja totalmente concluída. Um dos motivos é que a avaliação pode depender de um parecer especializado para ser concluída. Nessa situação, o sistema pode manter a solução na base de casos sendo que este deva permanecer com um indicador dizendo que ainda não foi avaliado~\cite{Gresse:2003}.

Um exemplo para a situação onde a solução demora para ser realizada é quando existe um sistema de tratamento médico onde é necessário esperar que a terapia seja concluída para dizer se a solução foi ou não adequada. Esse período pode levar alguns meses para que o resultado final seja alcançado, impactando diretamente no processo de avaliação da solução~\cite{Gresse:2003}.

%%%%%%%%%%%%%%%%%%%%%%%%%%%%%%%%%%%%%%%
\subsection{Corrigindo as Falhas}
\label{sec:corrigindofalhas}

Após o processo de avaliação é possível saber quais pontos da solução que falharam na aplicação ao mundo real ou em uma simulação. Sendo assim, torna-se factível a correção dessas falhas na solução para o caso em questão. A correção das falhas é dada através da explicação extraída no processo de avaliação da solução~\cite{Gresse:2003}.

As explicações das falhas ainda auxiliam na maneira como o sistema irá corrigi-las de acordo com o conhecimento sobre o domínio. No caso das falhas serem corrigidas por um especialista de maneira manual, é recomendado que este realize uma análise de todos os casos com soluções similares para que a falha não ocorra novamente no sistema. Se a correção for realizada automaticamente pelo sistema, esse deve estar programado para revisar todos os casos armazenados na base de conhecimento~\cite{Gresse:2003}.

%%%%%%%%%%%%%%%%%%%%%%%%%%%%%%%%%%%%%%%
\section{Retenção dos Casos}
\label{sec:retencaocasos} 
O processo de retenção de casos nada mais é do que a inclusão de um caso novo na base de conhecimento. A ideia por trás do processo de retenção é sempre armazenar a solução de um novo problema fazendo com que a base de dados esteja sempre atualizada e em constante crescimento. Quanto maior o número de casos armazenados dentro da base, maior é considerado o poder do sistema de RBC na solução de problemas. Essa etapa é dada através de algoritmos de aprendizado de máquina. Um dos algoritmos mais utilizados são os relacionados ao aprendizado baseado em instâncias~\cite{Gresse:2003}. 

O processo de retenção pode ser implementados em um sistema de RBC de três principais maneiras, são elas~\cite{Gresse:2003}:

\begin{enumerate}
	\item \textbf{Sem retenção de casos}: Esse é a implementação mais simples para um sistema de RBC, pois ela não possui a inclusão automática de novos casos na base de dados. Para que seja possível aplicar essa implementação é necessário possuir um conhecimento sólido sobre o domínio da aplicação do sistema. Dessa forma, o sistema não apresentará nenhum problema que já não esteja bem mapeado pelos especialistas. Um exemplo para esse tipo de sistema é uma linha de produção de um determinado modelo de veículo.
	\item \textbf{Retenção de soluções de problemas}: O tipo de aprendizado gerado por essa maneira de retenção é a mais clássica entre os sistemas de RBC. Nesse tipo de implementação toda vez que um problema é solucionado com sucesso, este é transformado em um caso da base de conhecimento. Dessa maneira, é possível expandir o conhecimento sobre um determinado domínio de aplicação do sistema de RBC. A retenção de soluções de problemas pode também armazenar as soluções que não foram satisfatórias para que essas não sejam utilizadas novamente para os problemas similares.
	\item \textbf{Retenção de documentos}: Nesse tipo de retenção o processo ocorre independentemente de um problema estar ou não solucionado. Na retenção de documentos o aprendizado e, consequentemente, a inclusão de novos casos da base ocorre sempre que um novo conhecimento sobre o domínio torna-se disponível ao sistema de alguma maneira. Por exemplo, toda vez que a descrição sobre um novo produto é disponibilizada pelo fabricante ou quando uma agência de viagens recebe informações de novos pacotes para viagens.
\end{enumerate}

O aprendizado gerado através da fase de retenção de casos  é considerada efetiva para o RBC, caso haja um conjunto de métodos bem trabalhado. Esse conjunto de métodos deve ser capaz de extrair conhecimento relevante sobre as experiências passadas, indexar o conhecimento para que seja possível utiliza-lo em um momento futuro e ainda integrar o conhecimento em uma estrutura existente~\cite{Gresse:2003}. Pode-se considerar os principais aspectos da fase de retenção os seguintes tópicos~\cite{Gresse:2003}:

\begin{enumerate}
	\item seleção adequada da informação armazenada em conjunto com os casos;
	\item seleção da estrutura da informação;
	\item seleção da estrutura de índices para buscas futuras na base;
	\item seleção do tipo de integração a ser realizado na estrutura de conhecimento.
\end{enumerate}

Com os principais conceitos de RBC apresentados, serão discutidos na seção~\ref{sec:rbcaplicado} as aplicações de RBC em diversas áreas de pesquisa. Contudo, o principal foco das aplicações estão em Interação Humano-Robô e Robótica em Geral.

%%%%%%%%%%%%%%%%%%%%%%%%%%%%%%%%%%%%%%%
\section{Raciocínio Baseado em Casos aplicado em Robótica}
\label{sec:rbcaplicado}

O desenvolvimento de aplicações para mundos reais são complexas, não sendo possíveis de uma observação completa do cenário e do ambiente explorado, e ainda as decisões, que devem ser tomadas em tempo real pelo agente em tarefa. Além disso, as tarefas executadas pelos agentes podem mudar a qualquer momento fazendo com que este deva mudar e ainda aprender com os novos comportamentos~\cite{Floyd:2011}.

Assim é proposto um \emph{framework} chamado jLOAF (Java Learning by ObservAtion Framework) permitindo que os agentes possam aprender as tarefas no mundo real através da observação do comportamento dos especialistas. Esse \emph{framework} permite a evolução dos agentes em diversos tipos de ambiente e aplicações. Ele faz com que os agentes aprendam o comportamento na execução de uma tarefa sem que se diga qual tarefa deve ser executada~\cite{Floyd:2011}.

Todas as observações realizadas pelo agente durante a interação do especialista com o ambiente são armazenadas com um caso dentro de uma base de conhecimento. Cada caso é representado por um conjunto de ações e reações entre o ambiente e o especialista que será recuperado pelo agente de acordo com a situação que ele irá encontrar no ambiente real~\cite{Floyd:2011}.

Para testar o \emph{framework} foram realizados testes em quatro casos de estudo diferentes aplicados em tarefas como controle de um braço robótico, jogo de futebol simulado e o jogo de Tetris. Os resultados apresentados são interessantes, pois com o \emph{framework} foi possível fazer com que o mesmo agente fosse capaz de aprender os diferentes domínios sem realizar nenhuma alteração no módulo de raciocínio do agente. Entretanto, existem algumas limitações no \emph{framework}, principalmente quando a tarefa é simples e não existe mudança de domínio. Esse tipo de situação inviabiliza o uso dele, que foi preparado para ambientes e tarefas mais complexas~\cite{Floyd:2011}.

Em outro trabalho um algoritmo para jogos baseado em Raciocínio Baseado em Casos para auxiliar no tratamento de fisioterapia para idosos é apresentado por \citeonline{Hansen:2012}. O jogo utiliza um robô móvel que deve desviar-se de uma bolinha que é jogada por uma pessoa. O robô adapta seu comportamento de acordo com as habilidades apresentadas pela pessoa durante o jogo. As habilidades da pessoa são medidas através do comportamento espaço-temporal dela, ou seja, a dificuldade de deslocamento até a bolinha ou o robô.

Os resultados apresentados durante os testes foram robustos e fazem com que o robô seja uma ótima ferramenta para esse tipo de tarefa. A adaptação do algoritmo utilizando RBC demonstrou boas soluções para ajustar os desafios do jogo. Entretanto, quando a pessoa não mantém um padrão de locomoção faz com que o algoritmo falhe na tomada de decisão. Esse é o um ponto que dever ser melhorado para trabalhos futuros~\cite{Hansen:2012}.

\citeonline{Srinivasan:2006} propõem um sistema para gerar estratégias de comportamentos ofensivos e defensivos em um time de futebol de robôs. O sistema é desenvolvido utilizando a metodologia de RBC unido a um classificador Bayesiano. Cada caso da base de conhecimento é descrito como um conjunto de movimentos que representam uma jogada do time. A jogada que será realizada é escolhida com base na similaridade entre a situação de jogo atual com as armazenadas na base de casos.

No trabalho de \citeonline{Srinivasan:2006} a composição dos casos é feita através de um vetor de características composto pela quantidade de jogadores da equipe, e também do oponente, dentro do quadrante onde a bola é localizada, além do fator que determina distância da bola até o gol. O vetor de características é comparado com os casos armazenados na base através do classificador Bayesiano que irá maximizar a escolha da melhor estratégia para o time. O classificador Naïve Bayes assume que existe uma independência condicional entre as características consideradas no vetor de características para realizar a tarefa de escolha de estratégia.

Quando a análise é feita sobre o cenário de serviço domésticos executados por robôs, pode ser obtido um intervalo amplo de tarefas no ambiente. Assim, para que os robôs sejam capazes de realizar as tarefas domésticas com destreza é necessário um mecanismo capaz de planejar com precisão cada passo a ser executado. Com esse objetivo em foco foi construído um sistema utilizando a abordagem da metodologia de RBC para um robô de serviço doméstico que torna capaz o aprendizado e ainda possui um mecanismo cognitivo de Interação Humano-Robô. O mecanismo cognitivo inclui quatro modelos para adaptação dos casos de acordo com a situação da tarefa. Os quatro modelos são: necessidades, tarefas, interações e modelo de usuários~\cite{Jung:2007}.

Entretanto, para que fosse possível reutilizar e flexibilizar o uso dos casos em diversas tarefas foi necessário o desenvolvimento de uma linguagem de descrição de tarefas para o robô. O objetivo de linguagem é chegar a um nível atômico de decisão de tal forma, que seja mais robusta o caminho escolhido pelo robô. Contudo, a aplicação de serviços domésticos é uma pequena parte na gerência de uma residência de maneira completa, mas \citeonline{Jung:2007} acreditam que o mecanismo desenvolvido será capaz de adaptar-se de maneira precisa para as demais tarefas a serem executadas em um residência.

Tratamentos realizados para reabilitação de pacientes e também para auxiliar em programas de educação, têm utilizado cada vez mais \emph{tablets} como ferramenta de apoio através dos aplicativos disponíveis. Entretanto as pessoas próximas dos principais usuários desses aplicativos, como pais de crianças, muitas vezes não possuem uma experiência e também não dispõem de paciência para auxiliar e encorajar o uso do aplicativo. Terapeutas especializados, que são mais indicados a acompanhar os pacientes, fazem com que seja necessário um alto nível de investimento que acaba sendo um pouco desvantajoso para os pais do paciente. Dessa forma, \citeonline{Park:2013} apresentam um robô que é capaz de realizar a tarefa de encorajar e auxiliar o uso de aplicativos voltados para terapia em \emph{tablets} disponíveis no cenário mercadológico atual.

Para que isso se torne possível é utilizado a metodologia de RBC, fazendo com que o robô aprenda interagir com o aplicativo e auxilie a pessoa através de uma tela compartilhada. Os casos do RBC são armazenados e criados através de um modelo estatístico que leva em consideração as ações tomadas para realizar a tarefa. Essas ações são contabilizadas através da quantidade de eventos que são acionados na aplicação~\cite{Park:2013}.

Ao longo dos testes percebeu-se que a quantidade de casos armazenados fazia com que fosse uma atividade complexa recuperar esses casos. Assim, eles foram agrupados de acordo com a tarefa que seria executada pelo robô. O desempenho do robô em relação ao tempo de execução da tarefa não foi considerado. O objetivo era fazer apenas com que o robô fosse possível de aprender as tarefas e auxiliar as pessoas. Agora como trabalho futuro deve ser melhorado o desempenho na interação social do robô e ainda transformar essa implementação em uma ferramenta que possa ser utilizada em diversos outros lugares para aplicação~\cite{Park:2013}.

O trabalho de \citeonline{Park:2013} apresenta uma possibilidade de trabalhar com diversos meios de entrada para um caso de interação, principalmente quando considera-se que o robô é apenas mais um agente dentro de ambientes inteligentes. Isso torna-se interessante para essa tese, pois é dado como uma premissa que o robô irá consumir informações providas de sensores espalhados em uma residência para ajudá-lo a entender o comportamento do indivíduo.

Ao realizar o planejamento de um caminho que o robô deve seguir até o objetivo evitando os obstáculos é uma tarefa complexa. Esse tópico ainda é uma das principais discussões existentes em robótica móvel. Entretanto, a complexidade dessa tarefa pode ser minimizada caso exista um conhecimento a priori sobre o ambiente ou pelo menos algumas dicas sobre como proceder em situações similares~\cite{Wang:2013}.

As abordagens mais clássicas para o problema de planejamento de trajetória não consideram nenhum conhecimento pré existente sobre o caminho a seguir. Dessa maneira, \citeonline{Wang:2013} resolveram unir o algoritmo \emph{modified artificial potential field} (MAPF) com a metodologia RBC para que assim fosse possível utilizar o conhecimento anterior em tarefas de exploração futuras.

Os resultados apresentados nos mostram que a técnica é capaz de encontrar o caminho ótimo, evitando os obstáculos, e o processamento computacional é simples e rápido. Contudo, ainda é necessário verificar se o mecanismo com RBC funciona corretamente em tempo real. Além disso, os mecanismos de busca dos casos devem ser aprimorados, pois estão com uma defasagem de processamento podendo gerar uma falha do RBC~\cite{Wang:2013}.

Imaginando a necessidade de ferramentas que implementam a metodologia de RBC pode-se citar uma linha de produção industrial. Nessa linha de produção existem diversos tipos de robôs que necessitam de manutenção em horas críticas. Caso isso não ocorra há um grande risco de acarretar um prejuízo caso exista a paralisação da indústria. Para que as manutenções sejam realizadas adequadamente existem diversos procedimentos, que devem ser executados, descritos nos manuais de cada equipamento. Existem milhares de manuais em cada linha de produção e para os mais diversos tipos de robô. Identificar o procedimento adequado de acordo com o problema apresentado muitas vezes é uma tarefa difícil, ainda mais levando em consideração o tempo para a tomada de decisão~\cite{Crowder:2000}.

Dessa forma, \citeonline{Crowder:2000} construíram um sistema de hipermídia unido a metodologia de RBC para que a partir de uma falha do robô ou até mesmo de sintomas prévios apresentados por ele, seja possível a execução da devida manutenção em tempo hábil para não existir prejuízo à empresa.

Em outro trabalho, \cite{CelibertoJr:2012} propõem a combinação do uso de RBC e Aprendizado por Reforço com Heurísticas para que possa otimizar o processo de aprendizagem durante a transferência de conhecimento entre robô em domínios diferentes porém similares. Para isso, é realizado um processo de armazenagem da política aprendida em forma de casos para alimentar a base de conhecimento. Os casos armazenados são como heurística, dependendo a situação da tarefa, para acelerar o aprendizado por reforço dos robôs em uma liga de simulação de futebol. Os testes foram executados em um jogo de futebol chamado Littman e na sequência, os casos extraídos foram utilizados no RoboCup Soccer Keepway para verificar o quão acelerado seria o aprendizado a partir do transferido. O novo algoritmo mesclando RBC e Aprendizado por Reforço Heurístico acelerou de forma significativa o aprendizado dos agentes entre os diferentes jogos.

Pode-se perceber ao longo dos projetos apresentados na seção~\ref{sec:rbcaplicado} que RBC é uma metodologia que tem a possibilidade de ser aplicado tanto áreas de pequisa e quanto na indústria. Nessa tese, o RBC tem o objetivo de armazenar características das interações entre os seres humanos e o robô, formando os casos de conhecimento. A partir dos casos é possível realizar um processo de aprendizagem para que o robô interaja de tal forma, que o ser humano fique confortável com sua aproximação e também com a execução de tarefas em conjunto com o robô. Além disso, o RBC pode ser utilizado como uma ferramenta para planejar os movimentos e navegação do robô e realizar a transferência de conhecimento entre robôs, o que será muito útil para o domínio de interação humano-robô apresentado como aplicação dessa tese.

%!TEX root=Principal.tex
\chapter{CLASSIFICADOR BAYESIANO DO USUÁRIO COM BASE NO COMPORTAMENTO DO ROBÔ E HEURÍSTICAS}
\label{cap:proposta}

Nos capítulos anteriores é possível identificar que a existência de robôs em ambientes sociais torna-se cada vez mais comum no dia-a-dia das pessoas. Como qualquer produto existente no mercado, é necessário que os robôs proporcionem uma boa experiência aos usuários durante o uso, que nesse caso é o convívio em ambiente doméstico ou industrial. Entretanto, as técnicas difundidas em interação humano-computador, que tem o objetivo de aumentar a experiência positiva do usuário em produtos tecnológicos, são pouco aplicadas por pesquisadores na área de robótica social, de serviço e assistiva~\cite{alenljung:2017}.

Dado o cenário onde a melhora da interação pode ser alcançada através do uso das técnicas de interação humano-computador, como as voltadas para usabilidade, essa tese apresenta um classificador bayesiano do perfil de usuário. A classificação é feita através das Personas que representam o usuário com base nas ações do robô, informações sobre conforto, desconforto e medo do usuário, técnica de \emph{Proxemics} para compreensão do espaço social e também de informações observadas através das heurísticas de avaliação de interface. O intuito é possibilitar que o robô seja capaz de identificar o perfil do usuário que está interagindo e a partir desse conhecimento consigar tomar a decisão de ações que promovam a melhora do relacionamento entre eles. No futuro, é esperado que o robô possa deixar o usuário a vontade enquanto convivem no mesmo ambiente, como se ele fosse um membro da família.

O trabalho apresentado nessa seção, serve como guia para classificação de usuário durante a interação e aproximação do robô. O uso desse classificador auxilia a identificar o usuário com base na sua experiência de interação com o robô, sendo ela positiva ou negativa. As técnicas utilizadas para compor esse classificador são: (I) Personas, técnica utilizada para representar um grupo de usuários através de um personagem fictício que possui o perfil médio do grupo; (II) Heurísticas de Interação utilizadas para avaliação da interface de um sistema, foram adaptadas para a interação humano-robô. Elas auxiliam o robô a medir variáveis sobre a experiência do usuário na interação; e (III) \emph{Proxemics}, teoria de proximidade (vide seção~\ref{cap:proxemics}), que estuda o comportamento de agentes sociais de acordo com a distância entre si, auxiliando o robô a se posicionar de maneira adequada a uma interação social.

Outro ponto importante apresentado na literatura, é sobre a cultura da pessoa. A cultura influência diretamente o comportamento de uma pessoa em ambientes sociais. Isso significa que algumas reações apresentadas por pessoas no ambiente social são influenciadas pelo seu local de nascimento, pelos locais onde viveu e, consequentemente, pela sua experiência de vida. Fatores como a experiência de vida e cultura são difíceis de avaliar apenas através da observação. Conseguir fazer com que uma pessoa tenha liberdade para contar esse tipo de informação necessita de interações de longo prazo, podendo levar meses até compreender toda história dessa pessoa. Algumas questões referentes a cultura do usuário são discutidas nos resultados apresentados no capítulo~\ref{cap:resultados}.

Sendo o robô um produto tecnológico, é importante o uso de técnicas que sejam capazes de aumentar a efetividade da interação, não havendo a necessidade de conhecer por completo a história de vida do indivíduo em questão. Essas técnicas tem o objetivo de deixar o sistema mais fácil de usar e acima de tudo aumentar a experiência, a satisfação do usuário ao utilizá-lo. Para que o robô possa realizar essa classificação e raciocínio sobre as informações que existem no ambiente alguns passos são apresentados ao longo deste capítulo.

Com base no robô disponível para os experimentos, as ações que são possíveis para ele executar são estabelecidas. A seção~\ref{sec:robo} apresenta em detalhes o robô utilizado para essa pesquisa. Na sequência dois questionários são apresentados, o primeiro aplicado antes do teste de interação e o segundo após. O primeiro questionário estabelece o perfil do usuário referente a questões de aderência a tecnologia, contato prévio com robôs, a cultura que ele se identifica e a expectativa em relação a robôs em ambientes domésticos e profissionais. O questionário pós-teste possui informações declaradas sobre o conforto, desconforto, medo e avaliação do comportamento do robô, durante a interação.

Após definição dos questionários, 39 pessoas foram selecionadas para participar do teste de interação com o robô, de acordo com o cenário descrito na seção~\ref{sec:cenario}. Os testes foram executados com o comportamento do robô de maneira aleatória, ou seja, com o mínimo de preocupação com a experiência do usuário neste momento, apenas zelando pela segurança. Coletados os perfis e informações dos usuários, é realizado uma preparação e normalização dos dados para que o algoritmo QG-SIM possa realizar o agrupamento dos perfis similares, conforme o processo apresentado por \citeonline{masiero:2013b}. A partir dos grupos gerados pelo algoritmo QG-SIM, é realizado o processo de criação de Personas~\cite{masiero:2013, masiero:2013b}.

Personas são criadas com base nas informações extraídas de cada grupo através dos questionários de pré e pós teste de interação. O próximo passo, é estabelecer as dependências condicionais entre cada variável aleatória incluida através da análise dos dados e observações dos testes de interação. As variáveis são compostas pelas Personas criadas, algumas heurísticas que se enquadram nas observações realizadas pelos usuários durante a interação com o robô, a proximidade entre robô e usuário que é importante para identificar os comportamentos e reações dado o espaço social ocupado, as ações e comportamentos do robô e efeitos de conforto, desconforto e medo da pessoa durante a interação. Essa dependência condicional entre as variáveis faz com que elas sejam organizadas no formato de uma rede bayesiana. O objetivo dessa rede é classificar qual o perfil do usuário dado que houve a leitura de conforto, desconforto e/ou medo durante a interação com o robô em um cenário residencial.

As variáveis aleatórias foram definidas de maneira abstrata e podem, no futuro, sugir outras que auxiliem na classificação do usuário de maneira que o resultado da rede bayesiana possa servir como alimentação de um mecanismo de decisão do robô para melhorar a interação com o usuário. Ao final, um conjunto de variáveis são propostos como pontos para futuras investigações e evolução do classificador entregue por essa tese.

\section{Ações e Comportamentos do Robô}
\label{sec:comportamento-robo}
O robô escolhido para os experimentos possui uma série de características que auxiliam a interação social e serviço domésticos, uma vez que ele foi construído para atender a competição de robótica RoboCup@Home~\cite{robocup:2015}. Com base nos atuadores de interação existentes (vide seção~\ref{sec:robo}), foram mapeados as variáveis de ações e comportamentos do robô. A tabela~\ref{tab:variaveisvalores} apresenta as variáveis, compostas por seus respectivos valores.

\begin{table}[!ht]
	\caption{Variáveis de Comportamento do Robô}
	\label{tab:variaveisvalores}
	\centering
	\begin{tabular}{c | c}
		\hline
		Variável & Valor \\
		\hline
		Gestos & curto \\
		& longo \\
		\hline
		Estilo da Fala & educada \\
		& autoritária \\
		\hline
		Expressão Facial & amigável \\
		& não amigável \\
		\hline
		Proximidade & longe - (Entre Pública e Social ) \\
		& perto - (Entre Pessoal e Íntima ) \\
		\hline
		Velocidade & rápida \\
		& devagar \\
		\hline
		Posição & sentado \\
		& em pé \\
		\hline
	\end{tabular}
	\smallcaption{Fonte: O autor.}
\end{table}

Além das variáveis direcionadas aos atuadores do robô, a tabela~\ref{tab:variaveisvalores} apresenta a variável de proximidade entre o robô e o ser humano e a variável posição que determina como o usuário estava durante a interação, sentado ou em pé. São variáveis importantes, pois auxiliam a determinar o comportamento de reação da pessoa para um mesmo tipo de gesto do robô em diferentes situações. Por exemplo, se o robô encontra-se próximo da pessoa, entre as zonas Pessoal e Íntima, um gesto amplo pode gerar um desconforto maior do que o mesmo gesto ocorrendo entre as zonas Pública e Social. Ou como é a reação do usuário quando o robô aproximar o manipulador próximo ao rosto do usuário.  A partir das variáveis definidas e comportamentos implementados no robô é importante agora trabalhar com os questionários pré e pós testes.

\section{Questionários para Coleta de Dados de Interação}
\label{sec:questionarios}
Para apoiar o processo de obtenção das informações e construção dos perfis dos usuários, dois questionários foram construídos. O primeiro, aplicado no momento anterior ao experimento de interação, tem como objetivo mapear as informações referentes às características físicas que tem a possibilidade do robô utilizar sensores para reconhecê-las, adesão a tecnologia, contatos prévios com robôs, questões culturais onde o usuário declara quais locais ele possui mais afinidade e quais ele já teve o privilégio de visitar, além da expectativa de possuir um robô em casa ou no trabalho. As questões serão agrupadas de acordo com seus objetivos e apresentadas em listas. Todas as questões a seguir fazem para do questionário pré-experimento.

Questões focadas na identificação e características físicas do usuário:

\begin{enumerate}
	\item Informe seu nome completo
	\item e-mail para contato
	\item Informe o número do seu celular
	\item Testes poderão ocorrer usando o Robô no Centro Universitário FEI. Você gostaria de realizar o teste com o robô físico? Opções de resposta: Sim, Não
	\item Qual a sua idade? (em anos)
	\item Qual a sua altura? (em metros)
	\item Informa seu gênero: Opções de resposta: Feminino, Masculino, Prefiro não dizer
	\item Na maior parte do tempo, você se considera uma pessoa com feição: Opções de reposta: Sorridente, Normal, Séria/Fechada
	\item Você se considera uma pessoa sociável? Opções de resposta: Sim, Não
	\item Você utiliza óculos de grau? Opções de resposta: Sim, Não (Obs: Pessoas com lente de contato, por favor, repondam não. A intenção é identificar a armação.)
	\item Você possui cabelo comprido? Opções de resposta: Sim, Não
	\item Qual etnia você se considera? Opções de resposta: Amarela, Branca, Indígena, Parda, Preta, Não declarada
\end{enumerate}

Questões para informações sobre adesão à tecnologia:

\begin{enumerate}
	\item Qual(is) dispositivo(s) tecnológico(s) você mais utiliza (marque 1 ou mais opções): Opções de resposta: Celular, Computador (de mesa ou notebook), Tablet, Smart TV, Relógio Smart, MP3 Player, Câmera Fotográfica Digital, Leitor de e-Book, outros
	\item Qual(is) dispositivo(s) tecnológico(s) você nunca utilizou (marque 1 ou mais opções): Opções de resposta: Celular, Computador (de mesa ou notebook), Tablet, Smart TV, Relógio Smart, MP3 Player, Câmera Fotográfica Digital, Leitor de e-Book, Já utilizei todas
	\item Você possui conta em banco digital (ex: Original, Neon, etc.) ? Opções de resposta: Sim, Não
	\item Você possui cartão de crédito digital (ex: Nubank, Digio, etc.) ? Opções de resposta: Sim, Não
	\item Qual o principal meio de pagamento de suas contas? Opções de resposta: Celular, Computador, Tablet, Autoatendimento, Caixa Físico
	\item Você utiliza redes sociais? Opções de resposta: Sim, Não
	\item Quais as redes sociais que você mais utiliza (marque 1 ou mais opções): Opções de resposta: Facebook, Instagram, Twitter, Google\+, Snapchat, outras
\end{enumerate}

Questões sobre informações culturais:

\begin{enumerate}
	\item Qual foi o local de nascimento? (Informe da seguinte maneira: Cidade; Estado; País)
	\item Em qual local, você viveu por mais tempo durante sua infância e adolescência? (Informe da seguinte maneira: Cidade; Estado; País)
	\item Qual o seu atual local de moradia? (Informe da seguinte maneira: Cidade; Estado; País)
	\item Qual o país que você melhor se identifica com a cultura? (Considere também a opção do seu país de nascimento.)
	\item Qual a cidade, na sua opinião, que melhor representa a cultura que você se identifica (resposta não dependente da questão acima)?
	\item Você visitou outros países, além do Brasil? Opções de resposta: Sim, Não
	\item Quais países você já visitou? (Responda separando os países por ponto e vírgula, ex: França; Estados Unidos; Itália; Japão;)
	\item Aproximadamente, quantas cidades na região nordeste do Brasil você visitou?
	\item Aproximadamente, quantas cidades na região norte do Brasil você visitou?
	\item Aproximadamente, quantas cidades na região centro-oeste do Brasil você visitou?
	\item Aproximadamente, quantas cidades na região sudeste do Brasil você visitou?
	\item Aproximadamente, quantas cidades na região sul do Brasil você visitou?
\end{enumerate}

Por fim, questões sobre contato prévio com robôs e expectativa de tê-los no trabalho e também em casa:

\begin{enumerate}
	\item Em algum momento de sua vida, você teve contato com robôs? Opções de resposta: Sim, Não
	\item Se sim para a questão anterior, quais tipos de robôs você teve contato (marque 1 ou mais opções): Opções de resposta: Parecido com animais, Parecido com pessoas, Robôs de linha de produção/fábrica, Robôs Móveis (que contém rodas), outros
	\item O que você espera do comportamento do Robô ao tê-lo em sua casa?
	\item O que você espera do comportamento do Robô ao tê-lo em seu trabalho?
	\item Dadas as questões anteriores, gostaria de fazer mais algum comentário sobre você?
\end{enumerate}

O segundo questionário mantém o foco na interação do usuário durante o experimento e quais pontos do robô mais agradaram ou não na opinião dele. Além disso, um detalhe sobre a posição do usuário durante o experimento (sentado ou em pé) é coleta, pois está informação pode influenciar na interação com o robô. As questões apresentas a seguir compõe o questionário pós-teste:

\begin{enumerate}
	\item Informe o número de amostra (Identificador dos documentos referentes ao comitê de ética)
	\item Você se sentiu confortável durante a aproximação do robô? Escala de 1 à 10
	\item Você se sentiu com medo em algum momento durante a aproximação do robô? Escala de 1 à 10
	\item Você estava \_\_\_\_\_\_\_\_\_ durante a aproximação do robô. Opções de resposta: Sentado, em Pé
	\item Você voltaria a interagir com esse robô novamente? Opções de resposta: Sim, Não
	\item Justifique a resposta anterior.
	\item O que você mais gostou no robô?
	\item O que você menos gostou no robô?
	\item Depois dessa experiência, você interagiria com outros robôs? Opções de resposta: Sim, Não
	\item Você estaria confortável com um robô convivendo em sua casa? Opções de resposta: Sim, Não
	\item Justifique a resposta anterior.
	\item Em algum momento da interação, você se sentiu desconfortável com o comportamento do robô? Opções de resposta: Sim, Não
	\item Descreva o desconforto em caso de sim, na resposta anterior.
	\item Você alteraria algum comportamento apresentado pelo robô durante o teste? Qual?
	\item Observações e comentários:
\end{enumerate}

As informações coletadas através dos questionários apresentados são utilizadas no processo de definição das probabilidades referentes as tabelas condicionais das variáveis aleatórias que compõem a rede bayesiana. Todo o processo e questionários apresentados nesse capítulo são aprovados pelo comitê ética sobre identificador de processo CAAE: 70057117.0.0000.5508.

\section{Preparando os dados para o algoritmo QG-SIM}
\label{sec:preparacao}
Com as informações obtidas através dos testes e observações das interações, é necessário trabalhar na normalização dos dados para que seja possível executar o algoritmo que agrupará os perfis semelhantes. Nesse momento, é necessário uma análise manual para separar as informações. O primeiro passo é deixá-las todas em uma única base de dados, pois encontram-se em bases distintas.

Base de dados unificada, o próximo passo é remover as informações de texto livre, uma vez que o algoritmo não possui um interpretador semântico e dessa maneira não é possível criar um modelo quantitativo para essas respostas, de maneira que exista uma significância comparativa entre as respostas.

As informações quantitativas existentes, como por exemplo a idade do usuário, pode-se aplicar medidas como a distância euclidiana para realizar a comparação entre um perfil e outro. Outras medidas também podem ser aplicadas, isso fica a critério da necessidade do projeto~\cite{masiero:2013}. No caso do agrupamento de perfis desta tese, a distância euclidiana é adotada já que atende a necessidade do algoritmo e do processo para agrupamento dos perfis.

Em relação as variáveis categóricas, ou seja, as variáveis que possuem um valor textual que podem ser separadas por categoria, existem duas opções para converte-las em valores quantitativos. A primeira opção é inserir um código númerico para cada valor, por exemplo, os valores ``Celular, Computador, Tablet, Autoatendimento, Caixa Físico'' recebem um valor representado por um número inteiro cada ficando ``Celular = 1, Computador = 2, Tablet = 3, Autoatendimento = 4, Caixa Físico = 5'', conforme~\citeonline{masiero:2013}. A segunda opção é transformá-las em variáveis \emph{dummies}~\footnote{http://pandas.pydata.org/}. O método \emph{dummies} transforma cada opção de resposta ou cada categoria em uma nova variável binária onde o valor 1 é para quando a opção for verdadeira e 0 para o oposto.

Nesse ponto a base de dados está com todas as variáveis quantificadas, porém existe um segundo problema que pode afetar o resultado do algoritmo. Cada variável possui uma escala diferente. Essa diferença na escala das variáveis pode gerar tendências no resultado do algoritmo, assim é necessário padronizar os valores númericos existentes na base. Para realizar a padronização dos dados o processo de normalização é executado. A normalização mais comum a ser feita é manter os valores das variáveis entre 0 e 1~\cite{lattin:2011}. A equação~\ref{eq:normalizacao1} apresenta a forma mais simples de realizar o processo de normalização dos dados. É feita a divisão do valor da característica pelo valor máximo encontrado entre a característica analisada.

\begin{equation}
	X_{i_{normalizado}} = \frac{X_i}{\max_{X_i}}
	\label{eq:normalizacao1}
\end{equation}

Entretanto, o uso da equação \ref{eq:normalizacao1} para normalizar os dados, pode gerar também uma tendência ou generalização da normalização. Pode existir uma concentração dos dados em um determinado intervalo generalizando a informação coletada~\cite{masiero:2013}. Para evitar o problema da concentração dos dados, utiliza-se a equação \ref{eq:normalizacao2} como método mais efetivo na normalização dos dados.

\begin{equation}
	X_{i_{normalizado}} = \frac{X_i - \min_{X_i}}{\max_{X_i} - \min_{X_i}}
	\label{eq:normalizacao2}
\end{equation}

Após o processo de normalização, as escalas da base estão com uma distribuição uniforme e prontas para serem consumidas pelo algoritmo.

\section{Executando o QG-SIM e criando as Personas}
\label{sec:criarpersonas}
Com as informações normalizadas, o próximo passo é executar o algoritmo de agrupamento QG-SIM. A implementação do algoritmo pode ser encontrada no endereço \url{https://github.com/amasiero/qgsim}. O algoritmo solicita um valor de similaridade mínimo para manter a qualidade entre os elementos de um mesmo grupo. Esse valor é chamado de Q. A partir desse valor, o QG-SIM agrupará os perfis de acordo com a similaridade desejada e o número de grupos é obtido automaticamente através do valor Q~\cite{masiero:2013}.

Assim que os grupos são definidos, é necessário encontrar a medida de dispersão central para cada uma das variáveis do perfil. As medidas de dispersão mais comuns são: média, mediana e moda. Elas auxiliam no processo de construção da Persona, como definição de idade, o uso de óculos, tipo do cabelo, gênero, e outras informações. Como os grupos encontrados não possuiam ruídos em sua distribuição, a medida de disperção adota foi a média (vide capítulo~\ref{cap:resultados}). Nesse momento, as informações de texto livres preenchidas são utilizadas para análise que é base para preencher a descrição e história da Persona. É verificado cada resposta dos indivíduos que compõe o grupo e na sequência é criado toda uma história de vida e experiências para a Persona.

Nesse processo, cinco Personas foram encontradas e são apresentadas a seguir:
\begin{table}[!ht]
	\caption{Persona Joaquim}
	\label{tab:joaquim}
	\centering
	\begin{tabular}{ m{2 cm} | m{13cm} }
		\hline
		Foto: & \rule{0cm}{2.7cm} \includegraphics[scale=0.8]{joaquim.png} \\
		\hline
		Nome: & Joaquim \\
		\hline
		Descrição: & Tem 21 anos, 1,71 m de altura, em geral não é uma pessoa séria ou carrancuda, mas também não é sorridente. É um homem sociável, cheio de amigos a sua volta e adora ir ao barzinho com eles. Mora na capital paulista, centro econômico brasileiro, local perfeito para um homem que gosta de variedade cultural. Não fica longe de seu smartphone e também sempre que pode, está com seu laptop no colo navegando pelo Facebook e postando fotos no Instagram. Tudo que pode ser resolvido pelo seu smartphone ele faz, seja por chamada de voz ou qualquer aplicativo. Mas, ainda não conseguiu se habituar aos serviços financeiros digitais, prefere o método clássico para guardar seu dinheiro, o colchão. Nunca viajou para fora do Brasil, inclusive seu mapa de viagens nacionais também não é extenso. Ao todo, visitou apenas 9 cidades do Brasil com o passar do tempo.

		Na universidade acompanhou os times de robótica nas competições e teve contato com diversos tipos de robôs, como os parecidos com humanos e animais, com mobilidade através de rodas e também os de linha de produção. Quando perguntam sua expectativa sobre robôs convivendo em sua casa, ele diz que tudo bem, desde que ele execute as tarefas domésticas sempre com obediência e de certa maneira, também espera que o robô seja afetivo na interação. Um comportamento próximo ao de uma diarista na família. Já no ambiente industrial, Joaquim acredita que os robôs são apenas ferramentas de trabalho e não devem fazer nada além de executar o que lhe foi programado. \\
		\hline
	\end{tabular}
	\smallcaption{Fonte: O autor.}
\end{table}

\begin{table}[!ht]
	\caption{Persona Maria Eduarda}
	\label{tab:mariaeduarda}
	\centering
	\begin{tabular}{ m{2 cm} | m{13cm} }
		\hline
		Foto: & \rule{0cm}{2.7cm} \includegraphics[scale=0.8]{maria_eduarda.png} \\
		\hline
		Nome: & Maria Eduarda \\
		\hline
		Descrição: & Aos 36 anos, com 1,71 m de altura, é uma garota reservada que adora sorrir em diversas ocasiões. É bem sociável, e mantém os amigos por perto. É uma mulher moderna e gosta de manter seu corte de cabelo mais curto que o convencional. Mora em São Bernardo do Campo, cidade da grande São Paulo e gosta muito de visitar o interior de São Paulo para passar seus feriados prolongados. Não vive sem seu celular, e no trabalho o computador é sua principal ferramenta. Quando está em casa utiliza sua Smart TV para assistir suas séries e filmes favoritos. Gostario muito de ter um leitor de e-book para evitar carregar livros pesados durante seu trajeto pelo transporte público. Mesmo com essa adoção a tecnologia, empresas digitais, principalmente do mercado financeiro, não a atraem. Sempre conectada através do celular, ela posta tudo no Facebook, tanto de trabalho quanto de lazer.

		Já viajou algumas vezes para os EUA, sempre a passeio com o principal destino a Disney. Pelo Brasil, já viajou para algumas cidades fora de São Paulo e deixou sua marca por todas as regiões do país. Como ela trabalha em uma universidade de engenharia, já viu diversos tipos de robôs, que são utilizados nas aulas. Porém, nunca teve um contato direto com eles, a não ser seu aspirador de pó. Tanto em casa quanto no trabalho, ela espera que robôs sejam capazes de realizar tarefas com eficácia, como dirigir um carro, digitar planilhas, mas que ao mesmo tempo não seja capaz de substituí-la.\\
		\hline
	\end{tabular}
	\smallcaption{Fonte: O autor.}
\end{table}

\begin{table}[!ht]
	\caption{Persona Alfredo}
	\label{tab:alfredo}
	\centering
	\begin{tabular}{ m{2 cm} | m{13cm} }
		\hline
		Foto: & \rule{0cm}{2.7cm} \includegraphics[scale=0.8]{alfredo.png} \\
		\hline
		Nome: & Alfredo \\
		\hline
		Descrição: & Aos 24 anos, rapaz de estatura normal, por volta de 1,75m, está sempre com um belo sorriso no rosto, faça chuva ou faça sol. Sempre tem pessoas a sua volta, gosta de contar piadas e fazer todos sorrirem. Morador da cidade de São Bernardo do Campo, mas sempre que pode vai para o litoral paulista visitar os pais e curtir uma praia. Usa computador para fazer os trabalhos da faculdade e passa grande parte do seu tempo no celular. Não possui serviços financeiros digitas, pois ainda não conseguiu a aprovação do cadastro. Quando se trata de internet banking, acredita que o seu computador é mais seguro que o uso de celular.

		Alfredo vive antenado nas redes sociais, como Twitter, Instagram e Facebook. Ajudam ele a ficar conectado com as últimas notícias e eventos a sua volta. Tem um sonho de viajar para o exterior, mas isso ainda não foi possível, em compensação pelo Brasil já visitou mais de 30 cidades, a maioria na região Sudeste. Na universidade, através do curso de engenharia de automação, teve contato com robôs de fábrica e móveis conforme os laboratórios das disciplinas ocorriam. Qunado perguntam a Alfredo o que ele espera de um robô doméstico e também um robô no trabalho, ele diz que robôs devem executar as tarefas propostas de maneira eficiente e que seu interação seja toda por comando de voz.\\
		\hline
	\end{tabular}
	\smallcaption{Fonte: O autor.}
\end{table}

\begin{table}[!ht]
	\caption{Persona Danielo}
	\label{tab:danielo}
	\centering
	\begin{tabular}{ m{2 cm} | m{13cm} }
		\hline
		Foto: & \rule{0cm}{2.7cm} \includegraphics[scale=0.8]{danielo.png} \\
		\hline
		Nome: & Danielo \\
		\hline
		Descrição: & Com 27 anos de idade, 1,83m, Danielo está sempre na academia para treinar com seus amigos. Mora em São Bernardo do Campo, e utiliza seu computador para fazer seu trabalho e o celular para manter contato com seus amigos. Nunca quis saber de leitores de e-book, pois acha sua tecnologia sem utilidade nos dias atuais. A sua única rede social é o Facebook. Ele acha que já toma tempo o suficiente e não precisa de outras para ver a mesma coisa. Danielo é um rapaz que já viajou bastante. Já visitou 3 países latinos e no Brasil visitou mais de 90 cidades, concetradas em sua grande parte, na região Sudeste. O contato com robôs é limitado e restrito a robôs de fábrica. Em casa ele acredita que o robô será parecido com seres humanos para fazer as atividades domésticas, e no trabalho substituirão seres humanos em trabalhos repetitivos, como nas fábricas e linha de produção.\\
		\hline
	\end{tabular}
	\smallcaption{Fonte: O autor.}
\end{table}

\begin{table}[!ht]
	\caption{Persona Manuel}
	\label{tab:manuel}
	\centering
	\begin{tabular}{ m{2 cm} | m{13cm} }
		\hline
		Foto: & \rule{0cm}{2.7cm} \includegraphics[scale=0.8]{manuel.png} \\
		\hline
		Nome: & Manuel \\
		\hline
		Descrição: & Aos 33 anos, 1,85 m,  Manuel um professor universitário sempre sorridente. Seus alunos sempre o procuram para esclarecer dúvidas e pedir conselhos. Mora em São Bernardo do Campo, próximo ao seu local de trabalho, por que adora o conforto de ir em sua casa poder almoçar uma comida fresca. Acredita que tem uma melhor qualidade de vida assim. Não é muito fã de tecnologia de ponta, então fica contente em ter seu computador, onde resolve tudo que pode. Digitalmente, considera-se antisocial e não mantém cadastro em nenhuma rede social.

		Já visitou paises pela Europa, África, América do Norte e do Sul. No Brasil, seu foco de visitar está na região Sudeste, principalmente o estado de Minas Gerais. No total já percorreu mais de 62 cidades pelo país. Como professor, sua linha linha de pesquisa principal de estudos é a robótica, fazendo com que tenha contato com todos os tipos de robôs. Em casa, pensa em ter um robô para atender suas necessidades, assim como no trabalho. Porém, o robô no trabalho deve atender também as necessidades e expectativas da empresa.\\
		\hline
	\end{tabular}
	\smallcaption{Fonte: O autor.}
\end{table}

As Personas apresentadas nas tabelas~\ref{tab:joaquim}, \ref{tab:mariaeduarda}, \ref{tab:alfredo}, \ref{tab:danielo}, \ref{tab:manuel} foram criadas com base nas informações coletadas nos questionários e ajudaram na definição das independências condicionais da rede bayesiana. Mais informações das análises feitas com base nas Personas, e também sobre sua criação, em relação a interação com o robô são apresentadas no capítulo~\ref{cap:resultados}.

\section{Heurísticas de Interação Humano-Robô}
\label{sec:heuristicas}
Avaliação heurística é um método utilizado por especialistas em usabilidade para verificar problemas de interação em interfaces de sistemas e produtos. \citeonline{nielsen:1994} apresenta 10 heurísticas para avaliações de interfaces em sistemas web (vide seção~\ref{sec:avaliacao}). As heurísticas de Nielsen tem sido amplamente utilizadas ao longo dos tempos para sites e sistemas desktop. Alguns trabalhos apresentados ao longo da seção~\ref{sec:ihrux} apresentam modificações das heurísticas de Nielsen para o cenário de interação humano-robô. Para os fins dessa tese, as heurísticas adaptadas que apresentam uma maior aplicabilidade em robótica social e interação humano-robô são as heurísticas de \citeonline{clarkson:2007}. As heurísticas são apresentadas na tabela~\ref{tab:heuristicasihr}.

Heurísticas de interação e interface são sempre utilizadas com o propósito de avaliar o produto para que o usuário fique mais satisfeito. Porém, durante os primeiros testes de interação entre o robô e o usuário, observou-se que muitas das pontuações sobre conforto, desconforto e até medo estavam dentro das diretrizes das heurísticas. Um ponto importante é que alguns perfis se atentaram a algumas características enquanto outros perfis nem mencionaram isso. Dessa maneira, essa tese propõem o uso das heurísticas como um conjunto de variáveis que complementa a classificação do perfil do usuário. O uso das heurísticas dessa maneria, está alinhada com o conceito de utilidade apresentado na seção~\ref{sec:raciocinio-probabilistico}, onde uma heurística pode ser mais útil para um determinado perfil de usuário do que para um terceiro.

Com base nas observações e informações dos testes de interação foi possível identificar as seguintes heurísticas para uso como variável de classificação do usuário:

\begin{itemize}
	\item Heurística 02: Visibilidade do estado do sistema;
	\item Heurística 04: Uso de sugestões naturais;
	\item Heurística 05: Síntese do sistema e interface;
	\item Heurística 06: Ajudar o usuário a reconhecer, diagnosticar, e recuperar de erros.
\end{itemize}

Cada uma das heurísticas apresentadas na lista acima, foram observadas entre os testes e possuem dependência condicional com os perfis e/ou com as ações do robô. Mais detalhes são apresentados na seção~\ref{sec:rede-bayesiana}, a seguir.

\section{Classificação do Usuário através da Rede Bayesiana}
\label{sec:rede-bayesiana}

Com todos os passos necessários para identificar o usuário e suas necessidades com o projeto, o próximo passo é a criação de um mecanismo que possa fazer com que o robô realize a classificação do perfil do usuário a partir das observações de conforto, desconforto e medo. Tais informações geram incertezas, que podem ocorrer baseadas em falhas de sensores, ruído da percepção do robô perante a ação/reação do usuário, cenário de atuação, estado emocional interno do usuário, entre outros. Dado esse cenário de incerteza, é importante que a técnica utilizada não afirme sobre uma classificação, mas sim possa inferir a probabilidade de uma resposta coerente a sua leitura. Como apresentado na seção~\ref{sec:cbrw}, a técnica de rede bayesiana tem sido utilizada em muitos trabalhos com o mesmo objetivo. Sendo assim, essa seção apresenta uma rede bayesiana que seja capaz de classificar o perfil do usuário com base na informação de conforto, desconforto e medo declarado pelo usuário. Na sequência será apresentado cada conjunto de nós e suas dependências condicionais para a criação da estrutura da rede bayesiana de classificação.

Os nós raizes da rede bayesiana são compostos pelas 5 Personas apresentadas na seção~\ref{sec:criarpersonas}. Elas são escolhidas como raiz por que representam os perfis de usuários que devem ser classificados durante a aproximação do robô para interação. Cada vez que novas interações ocorrem, é possível que novos perfis sejam incluídos para a classificação. A probabilidade do usuário em interação ser ou não aquela Persona é determinada, a princípio, pela quantidade de pessoas que ajudaram a compor ela. Como são nós raizes, não existem nenhuma observação que auxilie na composição do seu valor de probabilidade. As equações~\ref{eq:joaquim}, \ref{eq:mariaeduarda}, \ref{eq:alfredo}, \ref{eq:danielo} e \ref{eq:manuel} representam a probabilidade de cada uma das Personas obtidas.

\begin{equation}
	\label{eq:joaquim}
	P(joaquim)
\end{equation}

\begin{equation}
	\label{eq:mariaeduarda}
	P(maria\_eduarda)
\end{equation}

\begin{equation}
	\label{eq:alfredo}
	P(alfredo)
\end{equation}

\begin{equation}
	\label{eq:danielo}
	P(danielo)
\end{equation}

\begin{equation}
	\label{eq:manuel}
	P(manuel)
\end{equation}

As variáveis são nomeadas com letras minúsculas, pois são variáveis com apenas dois valores representando ser ou não ser. Essa notação segue a convenção apresentada por \citeonline{russell:2002}.

Seguindo com a construção da rede, cada nó interno foi considerado com base nas observações feitas pelos usuários dos testes iniciais. Do mesmo modo que as independências condicionais entre cada nó. A seguir o processo de criação dos nós será detalhado. Os primeiros nós inseridos na camada interna da rede bayesiana foram os comportamentos e ações do robô e também a questão da proximidade e posição do usuário no ambiente, como apresentado na tabela~\ref{tab:variaveisvalores}.

O primeiro nó é o nó Proximidade, que leva em consideração os espaços sociais definidos por \citeonline{hall:1969}. O domínio foi simplificado para \{perto, longe\}, pois durante os testes pilotos a reação do usuário era a mesma entre as regiões íntima e pessoal (perto) e as regiões social e pública (longe). A dependência condicional foi aplicada de acordo com a declaração explícita entre os perfis que sentiram algum desconforto com a aproximação do robô. A equação~\ref{eq:proximidade} define o cálculo de probabilidade condicional para a variável aleatória Proximidade.

\begin{equation}
	\label{eq:proximidade}
	P(Proximidade | joaquim, alfredo, danielo)
\end{equation}

A próxima variável aleatória inserida é a Posição da pessoa no ambiente. O domínio dessa variável é determinado por \{sentado, em pé\}. Ela foi observada durante a prova de reconhecimento de pessoas e aproximação do robô na RoboCup de 2016. Nesse cenário, as pessoas que estavam sentadas ficavam bem desconfortáveis com a aproximação do robô, principalmente com relação ao seu manipulador. Nos teste pilotos, a situação demonstrou-se a mesma. As pessoas que estavam sentadas demonstravam um comportamento mais apreensivo do que as em pé. A equação~\ref{eq:posicao} apresenta o cálculo de probabilidade condicional para a variável Posicao.

\begin{equation}
	\label{eq:posicao}
	P(Posicao | joaquim, maria\_eduarda, alfredo, danielo, manuel)
\end{equation}

As 4 próximas variáveis aleatórias descritas são referentes a ações do robô. Todas os 4 conjuntos são importantes na interação social e geram diferentes reações aos perfis de usuários. Um ponto interessante a ser resaltado é que cada Persona mapeada, ficou atenta durante a interação em apenas algumas das variáveis. As 4 variáveis são Expressão Facial (equação~\ref{eq:face}), Gestos (equação~\ref{eq:gestos}), Estilo da Fala (equação~\ref{eq:fala}) e Velocidade (equação~\ref{eq:velocidade}). Seus respectivos domínios estão descritos na tabela~\ref{tab:variaveisvalores}.

\begin{equation}
	\label{eq:face}
	P(Face | joaquim, maria\_eduarda, alfredo, danielo, manuel)
\end{equation}

\begin{equation}
	\label{eq:gestos}
	P(Gestos | maria\_eduarda, alfredo, danielo, manuel)
\end{equation}

\begin{equation}
	\label{eq:fala}
	P(Fala | joaquim, alfredo, danielo, manuel)
\end{equation}

\begin{equation}
	\label{eq:velocidade}
	P(Velocidade | joaquim, maria\_eduarda)
\end{equation}

A partir da variável Gestos, observou-se que quando ocorreu o toque do robô na pessoa, gerou uma situação de medo. A variável toque é mapeada com a dependência condicional da variável Gestos (equação~\ref{eq:toque}). Seu domínio é binário, $\{toque, \neg toque\}$.

\begin{equation}
	\label{eq:toque}
	P(toque | Gestos)
\end{equation}

Nos testes de interação, observou-se que as pessoas que interagiram com o robô apontavam situações na interação e comportamento do robô que condizem com heurísticas de avaliação de usabilidade. Quando as heurísticas eram atendidas, geravam mais conforto ao usuário. E de uma certa maneira, cada heurística apontada de maneira implicita, como a visibilidade do estado do robô, tinha uma relação diretamente com o perfil do usuário ou com algumas ações do robô. Visto que isso pode auxiliar na classificação do perfil para tomadas de decisões e adaptações do comportamento do robô no futuro, elas foram incluídas na camada interna da rede bayesiana. As heuristícas utilizadas foram apresentadas na seção~\ref{sec:heuristicas}. Nem todas as heurísticas foram utilizadas, pois nem todas tiveram relações com os comportamentos do robô durante os testes de interação e também não foram mencionadas pelos usuários. A lista a seguir mostra as heurísticas e as nomeações como variáveis aleatórias da rede bayesiana.

\begin{itemize}
	\item Visibilidade do estado do sistema - estado\_robo (equação~\ref{eq:estadorobo});
	\item Uso de sugestões naturais - natural (equação~\ref{eq:natural});
	\item Síntese do sistema e interface - sintese (equação~\ref{eq:sintese});
	\item Ajudar o usuário a reconhecer, diagnosticar, e recuperar de erros - ajudar (equação~\ref{eq:ajudar}).
\end{itemize}

\begin{equation}
	\label{eq:estadorobo}
	P(estado\_robo | joaquim, alfredo, manuel)
\end{equation}

\begin{equation}
	\label{eq:natural}
	P(natural | Fala, Gestos)
\end{equation}

\begin{equation}
	\label{eq:sintese}
	P(sintese | Fala)
\end{equation}

\begin{equation}
	\label{eq:ajudar}
	P(ajudar | maria\_eduarda, alfredo)
\end{equation}

Por fim, é definido as variáveis aleatórias chamadas de nós folhas da rede bayesiana. Esses nós correpondem ao sentimento das pessoas durante a interação com o robô. Esses sentimentos são declarados pelas pessoas durante a interação de acordo com o comportamento do robô. A composição das relações com esses nós foi dada pela observação dos testes e também das declarações realizadas através do questionário pós interação. As três variáveis aleatórias são conforto~\ref{eq:conforto}, desconforto~\ref{eq:desconforto} e medo~\ref{eq:medo}.

\begin{equation}
	\label{eq:conforto}
	P(conforto | Proximidade, Face, estado\_robo, natural, sintese)
\end{equation}

\begin{equation}
	\label{eq:desconforto}
	P(desconforto | Posicao, Face, estado\_robo, ajudar, natural)
\end{equation}

\begin{equation}
	\label{eq:medo}
	P(medo | Velocidade, Face, toque)
\end{equation}

As variáveis conforto e desconforto estão separadas, pois em alguns casos existiram pequenas diferenças que em uma mesma ação alguns usuários sentiram conforto e desconforto ao mesmo tempo. Por exemplo, ao se aproximar o robô chegou muito perto o que gerou o desconforto já que o usuário estava sentado, mas a expressão facial apresentada pelo robô no momento deixou ele tranquilo e confortável, mesmo não tendo como escapar da frente do robô.

A figura~\ref{fig:rb} apresenta a estrutura completa da rede bayesiana criada ao longo dessa seção.

\begin{figure}[ht!]
	\centering
	\begin{minipage}{\textwidth}
		\caption{Rede bayesiana construída para auxiliar no diagnóstico e avaliação da experiência do usuário na interação com o robô.}
		\includegraphics[width=\textwidth]{rb-thesis.png}
		\smallcaption{Fonte: Autor.}
		\label{fig:rb}
	\end{minipage}
\end{figure}

É importante ressaltar que a classificação do usuário é feita com base em sua experiência, pois é o principal interessado na interação com o robô. A grande preocupação em manter o foco no ser humano é por que ele é o mais interassado na interação com o sistema, já que irá conviver com o robô em sua casa brevemente. A tomada de decisão para melhorar a experiência do usuário a partir da classificação do perfil do usuário pelo robô não faz parte do escopo desta tese. Entretanto, uma lista com possíveis características a serem consideradas é apresentada na seção~\ref{sec:extracaocaracteristicas}. As variáveis para auxiliar melhor a identificação de características da interação podem complementar a camada interna da rede bayesiana e por consequência aprimorar os resultados do classificador bayesiano.

%!TEX root=Principal.tex
\chapter{CENÁRIOS DE TESTE}
\label{cap:testes}
Esse capítulo apresentará os cenários de teste considerados para auxiliar a validação do processo de aprendizado de interação entre humanos e robôs proposto por essa tese. São apresentados dois cenários, o primeiro representa a primeira abordagem para interagir com pessoas desconhecidas e o segundo cenário é uma continuação do primeiro, onde a partir de interações passadas o robô auxiliará uma pessoa durante alguma tarefa no ambiente doméstico.

\section{Cenário 1 - Primeira Interação}
\label{sec:cenario1}
O primeiro cenário de teste ocorre com o pressuposto de que o robô nunca interagiu com a pessoa em questão previamente. Para isso o robô PeopleBot é posicionado frente a frente com a pessoa. As informações sobre distância social(vide capítulo~\ref{cap:proxemics}) serão utilizadas para validar o cenário. A figura~\ref{fig:cenario1} apresenta uma ilustração do cenário e suas possibilidades.

\begin{figure}[ht!]
	\centering
	\includegraphics[width=0.8\textwidth]{images/cenario01.png}
	\caption{Cenário de teste para primeira interação entre humano e robô.}
	\label{fig:cenario1}
\end{figure}

Na figura~\ref{fig:cenario1}a o robô permanece parado e inicia a interação com a pessoa de tal forma, que essa sinta-se confortável e aproxime-se do robô. O cenário será considerado como concluído e também como sucesso quando a pessoa entrar pelo menos na zona pessoal do robô, demarcada pela cor azul, e permanecer pelo menos alguns segundos nela. A figura~\ref{fig:cenario1}b representa o cenário inverso. Nesse a pessoa fica parada e durante a interação o robô vai se aproximando dele. A velocidade que o robô irá se aproximar dependerá das reações positivas e negativas da pessoa para com as ações do robô. Como o cenário figura~\ref{fig:cenario1}a, o sucesso será determinado na entrada e permanência do robô à zona pessoal do indivíduo.

\section{Cenário 2 - Cenário Doméstico}
\label{sec:cenario2}
Nesse segundo cenário de teste o robô e a pessoa já interagiram previamente, sendo assim o desconforto inicial de interação já deve não existir mais. Contudo, variáveis como humor e tarefa a ser executada podem contribuir com a variação no estilo de interação. Dessa forma, o robô deve continuar com a preocupação de manter o ser humano confortável durante a execução da tarefa de ajuda. Para validar essa segunda interação é proposto um cenário de atividade doméstica, representado através da figura~\ref{fig:cenario2}.

\begin{figure}[ht!]
	\centering
	\includegraphics[width=0.8\textwidth]{images/cenario02.png}
	\caption{Cenário de teste para segunda interação entre humano e robô.}
	\label{fig:cenario2}
\end{figure}

O cenário apresentado na figura~\ref{fig:cenario2} é composto por quatro etapas, são elas: (a) existem pessoas em um ambiente doméstico e elas necessitam de alguma ajuda, seja para pegar um copo d'água ou tomar um medicamento, solicitando ao robô através de um comando de voz; (b) o robô percebe através de um movimento brusco da pessoa utilizando os sensores do Microsoft\textregistered\ Kinect\textregistered\ ou o ASUS\textregistered\ Xtion\textregistered\ ou recebe a informação direta através de um comando de voz pela pessoa solicitando a ajuda. O robô se aproxima do indivíduo de acordo com as informações obtidas e processadas a partir do cenário~\ref{sec:cenario1}. Ele também se mantém atento para qualquer adaptação de comportamento que seja necessário durante a interação alimentando a base de dados; (c) o robô identifica a ajuda que lhe foi solicitada e realiza a tarefa de acordo com o solicitado e na sequência retorna até a pessoa que lhe solicitou a ajuda; e (d) por fim, a pessoa que recebeu ajuda esboça o seu contentamento para com o robô pelo serviço prestado através de um elogio verbal, uma expressão facial positiva (sorriso, por exemplo) ou algum gesto de saudação.

O sucesso do cenário será medido de forma acumulativa durante todos os processos de interação entre o robô e o ser humano. Nesse cenário é esperado que o robô não faça nada para retroceder o estado de conforto da pessoa para com ele. Também é esperado que o tratamento para com o robô seja feito de maneira natural de tal forma, que a pessoa comece a não entender o robô como uma máquina, mas sim como qualquer outro agente que possa auxilia-lo em suas tarefas diárias.

\section{Seleção das Pessoas para o Teste}
\label{sec:perfistestes}
Para realizar os testes são priorizadas as pessoas que não tiveram nenhum contato prévio com robô ou com um contato mínimo. As pessoas possuem idades diversificadas, porém as preferências são por idosos e crianças. Alguns candidatos ao teste possuem medo declarado de robôs e neste caso o especialista ficará acompanhando o teste com uma maior proximidade para evitar problemas com o robô e principalmente com a pessoa.

Os integrantes da equipe que constrói o robô não serão consideradas como público ou registro oficial dos testes realizados nos dois cenários. Também são evitados a repetição dos candidatos entre os dois cenários com a tentativa de maximizar o resultado dos testes.

%!TEX root=Principal.tex
\chapter{RESULTADOS ESPERADOS}
\label{cap:resultados}
Como resultados desta tese é esperado um \emph{framework} para adaptação do comportamento do robô durante a interação com uma pessoa. Esse \emph{framework} deverá ser capaz de identificar algumas variáveis apresentadas na seção~\ref{sec:extracaocaracteristicas} e a partir dessas informações extrair o perfil comportamental das pessoas com quem o robô interagir. Dessa maneira, o robô irá conseguir fazer com que o indivíduo fique confortável e mantenha uma determinada qualidade de interação com ele. 

Através da qualidade de interação espera-se comprovar que o robô, independente da sua configuração, consegue manter o ser humano confortável com sua presença. Ainda é possível aproximar o comportamento do robô para uma forma mais natural e adequada ao ser humano. E também, existe a possibilidade de minimizar fatores regionais com relação a questão cultural, fazendo com que o comportamento do robô gere o mesmo resultado independente do local de origem do indivíduo.

Além do \emph{framework}, é esperado que possa encontrar um conjunto mínimo de variáveis de comportamento para fazer todo o trabalho de inferência na interação, sobre qual são as melhores ações para o robô de acordo com o perfil. Análises estatísticas devem indicar o quanto todo o processo mapeado auxiliou para manter o indivíduo confortável em meio a interação com o robô. E por fim, a continuidade no aprendizado de interação do robô, indiferentemente do seu formato e aparência, junto a qualquer perfil comportamental apresentado pelas pessoas.
%!TEX root=Principal.tex
\chapter{CONCLUSÕES PARCIAIS}
\label{cap:conclusoes}
De acordo com os estudos realizados na literatura existente, é possível perceber que a criação de um \emph{framework} para interação humano-robô capaz de aprender e se adaptar ao comportamento de uma pessoa torna-se viável e essencial a partir do momento que a popularização da robótica está cada vez maior, principalmente em ambientes domésticos para fins de ajuda ao ser humano.

Para a interação ocorrer de maneira efetiva é necessário que o robô saiba respeitar os limites espaciais do ser humano e também ao realizar uma aproximação ou movimento em direção a pessoa, estes devem ser delicado o suficiente para que não gere nenhum desconforto ou medo. Por exemplo, durante uma apresentação do robô PeopleBot para alunos e professores do ensino médio, percebeu-se que a aproximação do robô pode causar um certo desconforto e medo dependendo, em especial quando a pessoa não estava esperando essa aproximação e não era avisada sobre a ação. 

Quando o robô se locomovia em direção a pessoa sem nenhum anúncio prévio, essa pessoa por muitas vezes ficava com medo. O medo em algumas situações observadas era tão evidente que a pessoa deixa o mesmo ambiente que o robô estava. Porém, quando o robô se aproximava e era anunciado pelo apresentador, as pessoas ficavam paradas deixando o robô chegar a alguns poucos centímetros dela. 

As observações a partir desse experimento reforçam a importância de ter um componente de interação adaptativo para que o robô possa identificar o perfil comportamental e personalidade do indivíduo de tal forma que eles possam conviver no mesmo ambiente em uma maneira confortável e sem medo por parte do ser humano.

Esse componente deve ainda ser capaz de transferir o conhecimento adquirido a partir de um robô para outros robôs, levando em consideração não só as características da pessoa, mas também as características do robô, pois esses fatores podem influenciar no comportamento das pessoas e robôs durante a interação. As características do robô são muito importantes para determinar a forma de interagir, já que existem robôs em diversos formatos como quadrutores, direção diferencial, bípede, quadrupede, com ou sem manipuladores, com tamanhos diferentes e também o nível de ruído de cada robô. Todas essas variáveis devem ser consideradas em estudos futuros, mas já devem estar contempladas pelo \emph{framework} que é um dos produtos finais dessa tese.
% %!TEX root=Principal.tex
\chapter{CRONOGRAMA}
\label{cap:cronograma}
Nesse capitulo é apresentado o cronograma definido para a conclusão da tese. O início do cronograma é demarcado a partir da apresentação do exame de qualificação, conforme apresentado na figura~\ref{fig:cronograma}.

\begin{figure}[ht!]
	\centering
	\includegraphics[width=\textwidth]{images/cronograma.png}
	\caption{Cronograma para Conclusão da Tese.}
	\label{fig:cronograma}
\end{figure}

A lista a seguir apresenta com mais detalhes as tarefas apresentadas no cronograma da figura~\ref{fig:cronograma}.

\begin{enumerate}
	\item \textbf{A1}: Apresentação do Exame de Qualificação para o Doutorado;
	\item \textbf{A2}: Definição da Arquitetura que irá auxiliar o desenvolvimento e execução do Sistema que poderá ser consumido por diferentes tipos de robô ao mesmo tempo;
	\item \textbf{A3}: Desenvolvimento dos algoritmos que serão utilizados para fazer com que o robô possa extrair as informações comportamentais das pessoas, de acordo com o apresentado na seção~\ref{sec:extracaocaracteristicas};
	\item \textbf{A4}: Desenvolvimento dos algoritmos que irão controlar os atuadores do robô, como por exemplo, cabeça, manipulador e motores;
	\item \textbf{A5}: Desenvolvimento dos algoritmos que compõem o mecanismo para Raciocínio Baseado em Casos, responsável pelo aprendizado de interação do robô;
	\item \textbf{A6}: Execução dos testes de interação de acordo com o descrito no capitulo~\ref{cap:testes};
	\item \textbf{A7}: Análise dos resultados utilizando métodos estatísticos e também as observações obtidas durante o acompanhamento dos testes de interação;
	\item \textbf{A8}: Submissão de pelo menos 2 artigos sobre a tese para revistas relevantes para a área de pesquisa, classificadas de acordo com o WebQualis da CAPES entre os níveis de A1 à B2;
	\item \textbf{A9}: Revisão bibliográfica contínua para certificar da originalidade e atualidade do trabalho de tal forma, que sua contribuição possa ajudar o avanço da área de pesquisa em robótica social e assistiva;
	\item \textbf{A10}: Consolidação do trabalho no texto do documento da tese para entrega à banca avaliadora;
	\item \textbf{A11}: Apresentação da Defesa para o título de Doutor.
\end{enumerate}

% \bibliography{biblio}
\printbibliography
\end{document}
