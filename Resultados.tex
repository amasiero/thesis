%!TEX root=Principal.tex
\chapter{RESULTADOS ESPERADOS}
\label{cap:resultados}
Como resultados desta tese é esperado um \emph{framework} para adaptação do comportamento do robô durante a interação com uma pessoa. Esse \emph{framework} deverá ser capaz de identificar algumas variáveis apresentadas na seção~\ref{sec:extracaocaracteristicas} e a partir dessas informações extrair o perfil comportamental das pessoas com quem o robô interagir. Dessa maneira, o robô irá conseguir fazer com que o indivíduo fique confortável e mantenha uma determinada qualidade de interação com ele. 

Através da qualidade de interação espera-se comprovar que o robô, independente da sua configuração, consegue manter o ser humano confortável com sua presença. Ainda é possível aproximar o comportamento do robô para uma forma mais natural e adequada ao ser humano. E também, existe a possibilidade de minimizar fatores regionais com relação a questão cultural, fazendo com que o comportamento do robô gere o mesmo resultado independente do local de origem do indivíduo.

Além do \emph{framework}, é esperado que possa encontrar um conjunto mínimo de variáveis de comportamento para fazer todo o trabalho de inferência na interação, sobre qual são as melhores ações para o robô de acordo com o perfil. Análises estatísticas devem indicar o quanto todo o processo mapeado auxiliou para manter o indivíduo confortável em meio a interação com o robô. E por fim, a continuidade no aprendizado de interação do robô, indiferentemente do seu formato e aparência, junto a qualquer perfil comportamental apresentado pelas pessoas.