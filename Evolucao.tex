%!TEX root=Principal.tex
\chapter{EVOLUINDO O PROJETO DE IHR E O CLASSIFICADOR}
\label{cap:evolucao}
Com o classificador bayesiano de perfil do usuário pronto, o projeto está encontra-se completo para uso. Contudo, o projeto apresentado por essa tese é iterativo e pode evoluir a cada ciclo de interação e análises. Cada passo do projeto, apresentado nos capítulos~\ref{cap:projetoihr} e \ref{cap:proposta}, produzem conhecimentos que alimentam o projeto, fazendo com que haja sua evolução. Nesse capítulo são apresentados os pontos de conhecimento do projeto de interação humano-robô. Cada ponto desse é debatido e apresenta-se o que gera de contribuição para o projeto. Isso deve ocorrer tanto em novos projetos, quanto em projetos já em andamento de maneira que possam ser aprimorados para outras aplicações. A sequência da análise manterá a ordem apresentada entre os capítulos~\ref{cap:projetoihr} e~\ref{cap:proposta}.

\section{Contexto de Uso}
\label{sec:contextouso2}
O contexto de uso é um dos pilares do projeto de interação humano-robô apresentado nessa tese. Ele delimita as tomadas de decisão do projeto, pois é o ponto chave que determina o objetivo da interação. Dessa maneira, esse é o passo do projeto que pode sofrer maiores alterações e motivar mudanças consideráveis nos demais passos do ciclo.

Para que seja feita uma alteração no contexto de uso, é necessário identificar os pontos chaves da tarefa de interação e do contexto da aplicação. A partir da identificação desses pontos, é realizado a descrição do cenário de interação, estipular o escopo e seguir o processo de implementação, alterando os pontos necessários.

Outra maneira de evoluir o contexto de uso, é adicionar pequenos passos a cada interação. Por exemplo, no contexto de uso da figura~\ref{fig:contextouso} o robô apenas procura pelo objeto e não encontra. Em uma nova iteração de testes, ele pode encontrar o objeto e levá-lo até a pessoa para que ela fique com o ele. É uma pequena mudança no contexto de uso, porém pode gerar novas possibilidades de interação entre humano e robô. Novas possibilidades podem gerar novos comportamentos e percepções que irão auxiliar o robô a identificar melhor o perfil do usuário que interage com ele.

O contexto de uso apresentação na seção~\ref{sec:contextouso} pode ser aplicado em alguns cenários de tarefas domésticas e que envolvam um interação social em residência. Um outro contexto de uso que pode ser explorado é o de fábricas ou hospitais.

\section{Variáveis do Projeto}
\label{sec:variaveisprojeto2}
Dado o contexto de uso, é necessário verificar se todas as variáveis apresentadas na seção~\ref{sec:variáveis} atendem adequadamente. Caso a resposta seja negativa, o primeiro passo e identificar qual classe de variável não está adequada, ou falta informação. Dentre as classes mapeadas nesta tese temos as variáveis etnográficas (vide seção~\ref{sec:etnograficas}), variáveis comportamentais (vide seção~\ref{sec:reacoes}), varíaveis do robô (vide seção~\ref{sec:variaveisrobo}) e ação do robô (vide seção~\ref{sec:acoes}). Ao identificar as variáveis que não atendem, recomenda-se adicionar na lista para outros projetos e contextos de uso.

A lista de variáveis encontradas no projeto, podem ser utilizadas para capturar novas informações a fim de identificar melhor os usuários durante a classificação. Dessa maneira, a tomada de decisão poderá ter maior acurácia durante o processo. As variáveis do robô são as mais suscetíveis a mudança, pois dependerá muito do \emph{hardware} utilizado na construção do robô. As variáveis também podem auxiliar no mecanismo de tomada de decisão, para auxiliar na adaptação das ações do robô. A cada iteração do ciclo de vida do projeto de interação humano-robô, uma nova variável pode ser adicionada com o intuíto de melhorar a percepção que o robô tem do ambiente, e sua capacidade de executar as tarefas e interagir.

\section{Projeto Interação Humano-Robô}
\label{sec:projetoihr2}
Como todo projeto de \emph{software}, um robô também tem funcionalidades, requisitos de tarefas, novos \emph{hardwares}, entre outras coisas. Dessa maneira, o gerenciamento de mudanças do projeto é necessário. Para que o gerenciamento seja realizado, cada iteração de evolução do projeto deve ser feita seguindo algumas práticas de gerenciamento de projeto de \emph{software}. O primeiro passo, é analisar quais são as novas funcionalidades que devem ser inseridas no robô. Cada nova funcionalidade, deve estar de acordo com as demais e não pode gerar nenhum problema nas existentes. Os problemas tratados nesse ponto são referentes a deixar de executar as funcionalidades já existentes. Não pode ocorrer uma quebra de funcionamento do robô ao adicionar uma nova função. 

Por exemplo, adicionar um controle de impedância no manipulador do robô, para que não exerça uma força que machuque o ser humano. A adição dos controle de impedância, não pode afetar o controle de movimento do manipulador do robô. Nesse caso, uma análise de risco e impacto deve ser utilizada, validando a viabilidade de execução da inclusão dessa nova funcionalidade. Após isso, as implementações e testes funcionais devem ocorrer para validar não sá a nova, mas todas as funcionalidades já existentes.

Caso a equipe do projeto ache necessário, testes unitários podem ser implementados. Uma sequência de testes funcionais a nível de \emph{hardware} para verificar o robô toda vez que uma nova versão é entregue em seu sistema. Além dos testes, metodologias ágeis podem ser utilizadas para uma construção incremental de novos pontos e funcionalidades do projeto.

\section{Robô}
\label{sec:robo2}
Conforme é realizada a mudança do contexto de uso, novas funcionalidades no projeto e novos \emph{hardware} são inseridos, o robô precisa de ajustes. Os ajustes devem ser feitos de acordo com as novas inclusões do projeto. Caso seja apenas uma nova tarefa, uma nova implementação no \emph{software} pode contemplar essa demanda. Para novos \emph{hardware} alguns cuidados devem ser tomados, pois projetos mecânicos, eletrônicos e de \emph{software} pode ser necessários para atingir o objetivo. Em casos mais extremos, um novo projeto de robô inteiro pode ser considerado. Contudo, para o contexto de uso em cenários domésticos um robô com as mesmas características que o apresentado nessa tese atende grande parte dos objetivos.

Um outro ponto relevante é a manutenção corretiva do robô. Manter todo o projeto como componentes independentes que conversam entre si, auxiliou na instalação de componentes substitutivos. Então, como novas versões do \emph{hardware} podem aparecer no mercado, a evolução do robô também é considerada na atualização do equipamento.

\section{Personas}
\label{sec:personasnovas}
Personas são capazes de considerar um grande número de usuários. Dessa maneira, a criação de novas Personas não deve ser frequente. Contudo, assim como o ser humano, uma Persona também pode nascer, evoluir e morrer, de acordo com a evolução do público alvo do projeto. Caso haja a necessidade, pois existem muitos novos perfis que não se enquadram na classificação, o processo de criação deve ser realizado. Para isso, novos questionários com os perfis não reconhecidos pelo classificador e uma nova sequência de testes devem ser feitos. Depois o algoritmo de agrupamento é executado para encontrar novos grupos. Esse novos grupos passam pelo processo de definição da Personas apresentado na seção~\ref{sec:criacaopersonas}. A partir desse momento, a Persona está pronta para ser inserida no classificador como um novo nó raiz. Ao inseri-la, deve-se trabalhar com todos os passos para definir as probabilidades e dependências condicionais, dentro da rede bayesiana.

\section{Rede Bayesiana}
\label{sec:novarb}
Em caso de novas variáveis adicionadas na rede bayesiana, é necessário incluí-las na estrutura de maneira que minimize o número de probabilidades a serem calculadas. E ao mesmo tempo, maximize a classificação da Persona. Após a inserção, com as conexões devidamente realizadas, são calculadas as probabilidades para aquela nova variável. As probabilidades devem ser ajustadas nesse momento, de acordo com a experiência da equipe através dos testes e anotações feitas durante a interação entre o robô e o ser humano. O ajuste das probabilidades também pode ser feito a partir de um algoritmo de aprendizado. O algoritmo de aprendizado não foi abordado na tese, pois exige uma pesquisa dedicada a criação de um modelo bayesiano para o problema abordado, porém é um dos pontos de evolução entre os ciclos de iteração do projeto de interação humano-robô.

\section{Identificação das variáveis e adaptação das ações do robô}
\label{sec:ivaar}
Por fim, cada variável existente na rede bayesiana deve tornar-se um componente que torne o seu reconhecimento mais flexivel e melhor aplicado no contexto de uso. Assim, a equipe que aplicará essa rede bayesiana como classificador de perfil do usuário, pode melhor adaptar a captura de informação de acordo com o projeto e objetivo.

Do mesmo jeito que as variáveis, o mecanismo de tomada de decisão para adaptação das ações do robô, também precisam ser mantidos como componentes independentes. Assim, é possível compor pequenas tarefas em uma tarefa mais complexa, por exemplo, controle do manipulador, fala, expressão facial do robô e navegação compostos no cenário do contexto de uso da figura~\ref{fig:contextouso}.
