%!TEX root=Principal.tex
\chapter{INTRODUÇÃO}
\label{cap:introducao}
Com o passar dos anos é possível acompanhar a evolução dos sistemas computacionais, como por exemplo os telefones móveis, os computadores pessoais e portáteis, as televisões, e também os robôs pessoais, como o aspirador de pó iRobot Roomba\footnote{http://www.irobot.com/For-the-Home/Vacuum-Cleaning/Roomba.aspx} e o assistente pessoal JIBO\footnote{https://www.jibo.com/}. A evolução dos telefones móveis inteligentes mostra uma alta capacidade na realização de processamento de informações para executar diversas tarefas no dia a dia. Os componentes eletrônicos que compõem os aparelhos também diminuiram o tamanho. Isso permite que os aparelhos sejam mais finos, leves e com maior capacidade de processamento. Há também a inserção de robôs móveis em ambientes sociais, como as casas, hospitais e hotéis, unidos ao cenário da \emph{internet} das coisas. Entretanto, os robôs Roomba e JIBO possuem tarefas específicas e o nível de interação com as pessoas é limitado. Não existe uma diferença significativa entre esses robôs e outros \emph{gadgets} no mercado~\cite{heenan:2014}.

A popularização da robótica tem crescido e isso ocorre principalmente a depreciasão de componenentes comuns de tecnologia, como câmeras, computadores, sensores de distância, e \emph{tablets}. Esse fenômeno faz com que pesquisadores e fabricantes investiguem a necessidade de robôs inteligentes, que possuam a habilidade de interagir com as pessoas. Com a popularização do contato na interação humano-robô aumentará a necessidade de criar projetos com robôs que atendam as necessidades de cada usuário~\cite{looi:2012}. Isso torna a interação entre robôs e seres humanos importante, não apenas pela questão social, mas também porque uma boa interação passa a ser uma questão essencial para a convivência entre eles. Ao considerar que robôs encontram-se em ambientes sociais inteligentes como casas, hospitais, escolas, hotéis, investigar o desenvolvimento de robôs sociais é fundamental~\cite{albo-canals:2013, brown:2013}.

Um robô móvel inteligente possui várias maneiras de interagir. É capaz de identificar alguns padrões e ainda ter um nível de autonomia para tomada de decisões. O robô realiza as tarefas de interação através de sensores e atuadores espalhados em sua estrutura. Alguns sensores são câmeras, infravermelhos, \emph{laser}, de profundidade, térmicos, entre outros. Os atuadores são todos os dispositivos que possam gerar interação, externando algo para o indivíduo, seja através de um movimento, uma imagem ou até mesmo algum sinal sonoro. Alguns exemplos de atuadores são: \emph{tablets}, caixas de som, manipuladores e motores. O robô deve ser capaz de realizar a leitura de padrões do indivíduo para auxiliá-lo nas tarefas que seja necessárias~\cite{looi:2012, choi:2014, dobra:2014}.

Para isso, é necessário que o robô tenha um comportamento que atenda as necessidades de cada usuário durante a interação. Diversos fatores podem influenciar em um projeto de interação humano-robô. Alguns desses fatores são: cultura do indivíduo, o quão próximo ocorre a interação, o estado emocional, o cenário da interação, entre outros~\cite{hall:1969, argyle:1988, jung:1991}. Seres humanos conseguem tratar a questão da interação social de maneira natural e intuitiva. Todavia, as pessoas possuem diferentes perfis e podem reagir ainda de maneira diferente de acordo com a tarefa que estão executando ou o ambiente em que estão inseridos~\cite{jung:1991}. Dessa forma, há a necessidade de, em muitos casos, adaptar a forma de interação para conseguir ganhar a confiança do indivíduo e conseguir se aproximar para manter a interação por um período de tempo maior.

O primeiro passo para uma boa interação é estabelecer um nível de confiança com um indivíduo onde a aproximação dele chegue a um nível pessoal. E a partir desse ponto é possível realizar novas tarefas em colaboração ou até em benefício para o próprio indivíduo, como no caso de cuidados pessoais. Porém, deve-se fazer com que o robô consiga interagir de forma intuitiva e natural como a apresentada na interação entre os seres humanos. Essa naturalidade na interação não ocorre de maneira imediata entre os seres humanos, ela é aprendida ao longo de sua vida~\cite{hall:1969, argyle:1988}.

Porém, construir um robô que possua a capacidade de interagir com o ser humano, nas mais diversas tarefas, é complexo. Essa tarefa exige uma demanda no projeto que, por muitas vezes, não são consideradas de maneira adequada~\cite{alenljung:2017}. Em muitos projetos, o foco é a construção do comportamento do robô para uma determinada tarefa, sem preocupcação com a experiência que o usuário irá ter. Em trabalhos que a preocupação com a experiência é tratado com uma certa importância, não existe uma especificação sistêmica sobre o projeto. Esses pontos tornam o projeto difícil de ser replicado, pois exigem o mesmo equipamento e muitas vezes o mesmo cenário de aplicação~\cite{meerbeek:2009, ruckert:2013, alenljung:2017}.

Técnicas de especifícação de sistemas e experiência do usuário, podem auxiliar na construção de um robô que possa atender as necessidades do usuário. Com o uso das técnicas adequadas é possível estabelecer passos para construção de robôs, com uma boa especifícação, podendo reproduzi-lo e reaproveitar o projeto em diversos cenários. Quando é analisado cenários de interação social, o ponto chave é manter as pessoas na interação confortáveis e sem medo de uma aproximação de qualquer um envolvido. Neste caso, é necessário conhecer o perfil do usuário em questão, e estabelecer ações que auxiliem a manter o seu nível de conforto. Projetos de interação humano-robô não possuem um nível de detalhe e preocupação para que o robô seja melhor aceito pela sociedade~\cite{alenljung:2017}.

Para aumentar a aceitação desses projetos é necessário, criar de maneira sistemica, toda a especificação do robô (\emph{hardware e software}) deve ser documentada. Sendo assim, essa tese apresenta a criação de um projeto de interação humano-robô com uma documentação sistemica que pode auxiliar outros projetos e a melhoria de interações com robô. Questões que são abordadas durante a construção do projeto são: \emph{hardware}, \emph{software}, contexto de uso e cenário de interação, participantes de testes, funcionalidades do robô, entre outros. É apresentado também, um classificador do perfil do usuário para auxiliar na tomada de decisão da interação.

Identificar o perfil do usuário é importante para saber como o robô deve se comportar na interação. Ao se aproximar de uma pessoa, o robô deve realizar uma classificação de seu perfil baseando-se em seu comportamento e algumas informações de linguagem corporal, expressão facial, esteriótipos e vai ajustando essa classificação de acordo com as reações da outra pessoa. Em inteligência artificial, existem muitos algoritmos que são capazes de realizar a classificação de pessoas e de diversas maneiras. Porém, o uso de técnicas determinísticas não são aconselhadas, pois tratando de seres humanos existem muitas variáveis internas a ele que geram muita incerteza. Assim, classificadores probabilísticos são mais adequados para utilizar na tarefa com variáveis humanas~\cite{faceli:2011, hartson:2012}.

O classificador utilizado nessa tese, é um classificador bayesiano que utiliza informações sobre as ações do robô, comportamentais, cenário e de percepção sobre heurísticas de avaliação de usabilidade, para identificar o perfil do usuário definido como Personas. A partir da classificação é possível identificar ações que o robô deverá realizar para melhorar a interação com o usuário. Outro ponto importante é o uso do perfil como Persona, que auxilia a atender um público maior do que apenas os mapeados como Personas. Por fim, é discutido as tomadas de decisões que são feitas a partir da classificação do usuário e como é possível expandir o projeto de interação humano-robô apresentado.

%%%%%%%%%%%%%%%%%%%%%
\section{Objetivos}
Nessa seção são apresentados o objetivo principal e os objetivos secundários defendidos por essa tese.

%%%%%%%%%%%%%%%%%%%%%%%%%%%%%%
\subsection{Objetivo Principal}
Como objetivo principal, esta tese propõem um classificador bayesiano construído com base nas ações do robô e informações de comportamento e percepção do usuário para identificar grupos de perfis de usuários, no formato de Personas, durante a aproximação do robô.

%%%%%%%%%%%%%%%%%%%%%%%%%%%%%%
\subsection{Objetivos Secundários}
Os objetivos secundários almejados nessa tese são:

\begin{itemize}
    \item Entregar de um pacote funcional do ROS~\footnote{www.ros.org}, para utilização em qualquer robô que possua o conjunto de sensores e atuadores utilizados durante o processo.
    \item Apresentar uma metodologia de desenvolvimento de sistemas para construir um projeto de interação humano-robô, de maneira que fique escalável e de fácil manutenção.
\end{itemize}

%%%%%%%%%%%%%%%%%%%%%
\section{Hipóteses}
Como hipóteses de comprovação essa tese apresenta:

\begin{itemize}
    \item Durante a interação social, informações sobre a percepção do usuário e ações do robô são importantes para determinar o perfil do indivíduo;
    \item Padrões de comportamento de interação social são definidos com base na cultura, e a experiência de vida do indivíduo aumenta as possibilidades de interação humano-robô, porém não sobrepõe sua cultura.
\end{itemize}

%%%%%%%%%%%%%%%%%%%%%
\section{Motivação}
O crescente número de pesquisas em robótica aplicados em ambiente sociais como casas, hospitais e escolas fazem com que seja um tópico de atenção entre os pesquisadores. Esse é um tópico importante, pois os diferentes formatos existentes de robôs podem gerar problemas de confiabilidade. Esse é um ponto que pode determinar o conforto do usuário ao estar em mesmo ambiente que o robô. Por consequência, a questão da confiabilidade pode determinar a aceitação do robô.

Para mitigar esse problema, vários fatores devem ser analisados. Fatores como o perfil do usuário na interação social e também as características do projeto do robô. Todas essas informações são consideradas para que o robô possa predizer quais são as melhores ações de interação com um determinado indivíduo. Encontrar uma solução  para esse problema é uma tarefa complexa. Deve-se considerar a coleta e o processamento dessas informações para a tomada de decisão correta, o que em muitas vezes é necessário de sensores dedicados a uma tarefa especifíca, como sensor de profundidade.

O custo de processamento dessas informações pode ser alto para o robô pois, sua infraestrutura tem uma capacidade computacional e eletrônica que limita a tarefa. Sendo assim, é necessário que exista uma arquitetura de sistema capaz de considerar a complexibilidade e expansão dos equipamentos utilizados na construção do robô. Assim, é possível fazer com que o robô evolua ao longo do tempo.

%%%%%%%%%%%%%%%%%%%%%%%%%
\section{Justificativa}
Durante os estudos de trabalhos que realizam a análise de comportamento humano através de robôs aplicados principalmente em robótica social, notou-se que existem poucos estudos voltados ao projeto de interação humano-robô e aplicações que atendam as necessidades do usuário de maneira sistemica. Além disso, alguns trabalhos utilizam a técnica de \emph{Wizard of OZ} (WoZ) para realizar os testes com humnaos. Essa técnica condiz com o controle do robô de maneira remota, como se este fosse totalmente autonomo. Esse tipo de técnica, não consegue transmitir de maneira adequada o comportamento do robô.

Assim, a criação de um processo que seja capaz de fazer com que o robô possa, de maneira autônoma, classificar o perfil do usuário durante interação, e tomar a decisão sobre como interagir é necessário. Essa pesquisa é importante para que haja uma evolução dos ambientes inteligentes, principalmente os que consideram o robô como um agente. Além da evolução dos ambientes inteligentes, manter o indivíduo com a melhor experiência de interação com o robô, e também faze-lo confortável com a presença do robô.

%%%%%%%%%%%%%%%%%%%%%%
\section{Metodologia}
A fundamentação do trabalho é realizada em pesquisas de cada uma das áreas abrangentes, Interação Humano-Robô (IHR), conceito de \emph{proxemics}, experiência de usuário aplicado a IHR, agrupamento de dados e redes bayesianas para classificação, onde identificou-se a necessidade da criação de um projeto sistêmico para IHR. A partir desse projeto é possível determinar os passos para a construção do robô, especificação do contexto de uso, perfil de usuários para interação, e ferramentas de testes. Com tudo isso em mãos, foi submetido ao comitê de ética um projeto para aprovação dos testes.

A primeira bateria de testes foi realizada. A partir dos resultados dos testes iniciais, é aplicado o algoritmo QG-SIM para construção dos grupos de perfis similares. Com cada grupo identificado, são criadas as Personas que auxiliaram na tomada de decisão que, o robô deve ter durante a interação, para manter o usuário confortável. A partir desse ponto, as variáveis e observações dos testes são utilizadas para determinar as variáveis que compõem o classificador. Para o classificador é utilizado a técnica probabilística, rede bayesiana. O objetivo é eliminar repetição das dependências condicionais apresentadas na construção da estrutura da rede.

Na sequência novos testes são realizados, para que seja possível a validação do classificador e das questões referentes ao perfil do usuário. O cenário de teste utilizado é uma residência, onde o robô habita com mais uma pessoa. Realizados os testes, os resultados são analisados e discutidos, apresentando as estatísticas e observações obtidas durante o processo. Por fim, os próximos passos para o projeto são apresentados.

%%%%%%%%%%%%%%%%%%%%%%%%%%%%%%
\section{Estrutura do Trabalho}
Esta tese é composta por um total de 10 capítulos discriminados a seguir.

O capítulo \ref{cap:introducao} apresenta a \textbf{introdução} do trabalho conduzindo o leitor ao problema que a pesquisa desta tese deve contribuir.

O capítulo \ref{cap:ihr} introduz a área de \textbf{Interação Humano-Robô}, contando um pouco da história e sua importância para o futuro.

O capítulo \ref{cap:ux} introduz a área de \textbf{Experiência do Usuário}, apresentando os principais conceitos para o desenvolvimento de um sistema centrado no usuário. Também é apresentado trabalhos que aplicam as técnicas em cenários de interação humano-robô.

O capítulo \ref{cap:proxemics} apresenta o conceito de análise comportamental chamado \emph{\textbf{Proxemics}}, que tem como objetivo o estudo do espaço social durante a interação.

O capítulo \ref{cap:ai} apresenta os \textbf{conceitos de inteligência artificial} utilizados nessas tese. Duas técnicas são apresentadas em detalhes, algoritmos de agrupamento de dados para construção do perfil do usuário. E redes bayesianas que é utilizada como classificador do perfil do usuário.

O capítulo \ref{cap:projetoihr} apresenta a \textbf{especificação do projeto de interação humano-robô} apresentado para a construção do robô e cenário contemplado por esta tese.

O capítulo \ref{cap:proposta} apresenta as \textbf{Personas} e o \textbf{classificador} construídos para auxiliar no problema apresentado por esta tese.

O capítulo \ref{cap:evolucao} apresenta como realizar \textbf{expansões} em cada parte do projeto contemplado por esta tese.

O capítulo \ref{cap:resultados} apresenta os \textbf{resultados e discussões} desta tese.

O capítulo \ref{cap:conclusoes} apresenta as \textbf{conclusões e trabalho futuros} obtivdos ao longo dos estudos e testes dessa tese.
