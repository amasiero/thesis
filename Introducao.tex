%!TEX root=Principal.tex
\chapter{INTRODUÇÃO}
\label{cap:introducao}
Com o passar dos anos é possível acompanhar a evolução dos sistemas computacionais, como por exemplo os telefones móveis, os computadores pessoais e portáteis, as televisões, e também os robôs pessoais, como o aspirador de pó iRobot Roomba\footnote{http://www.irobot.com/For-the-Home/Vacuum-Cleaning/Roomba.aspx} e o assistente pessoal JIBO\footnote{https://www.jibo.com/}. A evolução dos telefones móveis inteligentes mostra uma alta capacidade na realização de processamento de informações para executar diversas tarefas no dia a dia. Os componentes eletrônicos que compõem os aparelhos também diminuiram o tamanho. Isso permite que os aparelhos sejam mais finos, leves e com maior capacidade de processamento. Há também a inserção de robôs móveis em ambientes sociais, como as casas, hospitais e hotéis, unidos ao cenário da \emph{internet} das coisas~\cite{heenan:2014}. Esses robôs são chamados de robôs de serviços. Segundo o \textit{International Federation of Robotics (IFR)}~\footnote{https://ifr.org/}, o ápice de robótica de serviço ocorrerá em 2020. Em 2016, a indústria de robótica de serviço movimentou \$40 bilhões.

A robótica de serviços abrange todos os ramos que não envolvem a indústria propriamente dita. A robótica de serviços possui desde robôs pessoais e domésticos até robôs de aplicações profissionais. Alguns exemplos de aplicações profissionais abrangem áreas como defesa, medicina, agricultura, pecuária, logísticas, entre outros, ou seja, qualquer robô que utilizado em ambiente não industrial. Segundo a IFR, de 2018 a 2020 é esperado um crescimento de 400 mil unidades, ou seja, entre 20 a 25\% ao ano.

A área de maior crescimento é a de logística com o crescimento por ano de 189.700 unidades. Quando fala-se de valores monetários o crescimento é de \$26,8 bilhões no mesmo período. Ao falar de investimentos a área médica é a que possui um valor de projetos de \$7,8 milhões. De uma certa forma, isso auxilia na popularização dos projetos em robótica de serviço.

A popularização da robótica tem crescido também pela depreciasão de componenentes comuns de tecnologia, como câmeras, computadores, sensores de distância, e \emph{tablets}. Esse fenômeno faz com que pesquisadores e fabricantes investiguem a necessidade de robôs inteligentes, que possuam a habilidade de interagir com as pessoas. Com a popularização do contato na interação humano-robô aumentará a necessidade de criar projetos com robôs que atendam as necessidades de cada usuário~\cite{looi:2012}. Isso torna a interação entre robôs e seres humanos importante, não apenas pela questão social, mas também porque uma boa interação passa a ser uma questão essencial para a convivência entre humanos e robôs. Ao considerar que robôs encontram-se em ambientes sociais inteligentes como casas, hospitais, escolas, hotéis, investigar o desenvolvimento de robôs sociais é fundamental~\cite{albo-canals:2013, brown:2013}.

Um robô móvel inteligente possui várias maneiras de interagir. É capaz de identificar alguns padrões e ainda ter um nível de autonomia para tomada de decisões. O robô realiza as tarefas de interação através de sensores e atuadores espalhados em sua estrutura. Alguns sensores utilizados na construção dele são câmeras, infravermelhos, \emph{laser}, de profundidade, térmicos, entre outros. Os atuadores são todos os dispositivos que possam gerar interação, externando algo para o indivíduo, seja através de um movimento, uma imagem ou até mesmo algum sinal sonoro. Alguns exemplos de atuadores são: \emph{tablets}, caixas de som, manipuladores e motores~\cite{looi:2012, choi:2014, dobra:2014}.

Apesar da popularização e de pesquisas voltadas para a robótica de serviços, a documentação e formalização dos passos do projeto de interação humano-robô são 
pouco aplicados. Projetos de interação humano-robô não possuem um nível de detalhe e preocupação para que o robô seja melhor aceito pela sociedade~\cite{alenljung:2017}. Técnicas de engenharia de \textit{software} e usabilidade, principalmente projetos centrados no usuário, disseminada na comunidade de pesquisadores de interação humano-computador, são pouco aplicadas em projetos voltados para robôs autônomos para interações sociais~\cite{alenljung:2017}.

Com o uso das técnicas adequadas é possível estabelecer passos para construção de robôs, com uma boa especificação, podendo reproduzi-lo e reaproveitar o projeto em diversos cenários. Além disso, a manutenção do robô também é favorecida com o método utilizado em sua construção é adequado, assim como sua evolução para versões mais robustas. Quando cenários de interação social são analisados, o ponto chave é manter as pessoas na interação confortáveis e sem medo de uma aproximação ou abordagem de qualquer agente envolvido, seja robô ou ser humano. No caso do ser humano, é necessário conhecer o perfil do usuário em questão, e estabelecer ações que auxiliem a manter o seu nível de conforto e assim manter uma interação de longa duração.

Para aumentar as possibilidades de reutilização, manutenção e evolução desses projetos, é necessário, criar de maneira sistêmica, toda a documentação com a especificação do robô (\emph{hardware e software}). Sendo assim, essa tese apresenta um método centrado no usuário para a construção de um projeto de interação humano-robô autônoma possibilitando uma melhor manutenção e melhorias do projeto sempre visando uma interação social de qualidade. Questões que são abordadas durante a construção do projeto são: \emph{hardware}, \emph{software}, contexto de uso, cenário de testes, participantes de testes, funcionalidades do robô, entre outros. Como o projeto é centrado no usuário, é importante traçar as características do perfil deste usuário para detectar melhor as suas necessidades e preferências.

A técnica utilizada para identificar o perfil do usuário neste trabalho é a de Personas. Personas são arquétipos hipotéticos que representam um grupo de usuários reais através de um personagem fictício~\cite{aquino:2005, masiero:2011}. A técnica de Persona é importante, pois o alcance em número de perfis é alto e a quantidade de personagens é bem menor. Isso torna a comunicação e as tomadas de decisões centradas no usuário mais fácil, do que olhando perfil a perfil. Assim, o método proposto aqui, utiliza a técnica de Persona para garantir as preferências do usuário durante a construção do projeto. 

Para demonstrar o uso do método proposto é realizado um estudo de caso para a construção de um robô autônomo, que realiza a classificação do perfil do usuário através de uma rede Bayesiana. O classificador apresentado nessa tese, é um classificador Bayesiano que utiliza informações sobre as ações do robô, comportamentais, cenário e de percepção sobre heurísticas de avaliação de usabilidade, para identificar o perfil do usuário definido como Personas. A partir da classificação é possível identificar ações que o robô deve realizar para melhorar a interação com o usuário. Essa evolução do estudo de caso é apresentada como um novo ciclo do método proposto para a construção do robô.

%%%%%%%%%%%%%%%%%%%%%
\section{OBJETIVO} % OK
Como objetivo esta tese propõem a construção de um método sistêmico para projetar e desenvolver um robô totalmente autônomo e centrado no usuário para interação social.

%%%%%%%%%%%%%%%%%%%%%
\section{MOTIVAÇÃO}
O crescente número de pesquisas em robótica aplicados em ambientes sociais como casas, hospitais e escolas fazem com que seja um tópico de atenção entre os pesquisadores. Esse é um tópico importante, pois os diferentes formatos existentes de robôs podem gerar problemas de confiabilidade. Esse é um ponto que pode determinar o conforto do usuário ao estar em mesmo ambiente que o robô. Por consequência, a questão da confiabilidade pode determinar a aceitação do robô.

Para mitigar esse problema, vários fatores devem ser analisados. Fatores como o perfil do usuário na interação social e também as características do projeto do robô. Todas essas informações são consideradas para que o robô possa predizer quais são as melhores ações de interação com um determinado indivíduo. Encontrar uma solução  para esse problema é uma tarefa complexa. Deve-se considerar a coleta e o processamento dessas informações para a tomada de decisão correta, o que em muitas vezes é necessário sensores dedicados a uma tarefa específica, como o sensor de profundidade.

O custo de processamento dessas informações pode ser alto para o robô pois, sua infraestrutura tem uma capacidade computacional e eletrônica que limita a tarefa. Sendo assim, é necessário que exista uma arquitetura de sistema capaz de considerar a manutenção e expansão dos equipamentos utilizados na construção do robô. Assim, é possível fazer com que o robô evolua ao longo do tempo.

%%%%%%%%%%%%%%%%%%%%%%%%%
\section{JUSTIFICATIVA}
Durante os estudos de trabalhos que realizam a análise de comportamento humano através de robôs aplicados principalmente em robótica social, notou-se que existem poucos estudos voltados ao projeto de interação humano-robô e aplicações que atendam as necessidades do usuário de maneira sistêmica. Além disso, alguns trabalhos~\cite{okita:2012, henkel:2012b, vazquez:2014} utilizam a técnica de \emph{Wizard of OZ} (WoZ) para realizar os testes com humanos. Essa técnica condiz com o controle do robô de maneira remota, como se este fosse totalmente autônomo. Esse tipo de técnica, não consegue transmitir de maneira adequada o comportamento do robô, uma vez que ela não consegue trabalhar com os ruídos dos sensores e problemas encontrados durante uma navegação autônoma, como combinação do mapa ou detecção de obstáculos e pessoas. Tudo isso é realizado pelo operador em posse do controle remoto, tirando a naturalidade e autenticidade da interação do robô.

Assim,  é necessário a criação de um método que seja capaz de construir um robô possa, de maneira autônoma, realizar diversas tarefas e tomadas de decisão durante interação, de maneira sistêmica. Essa pesquisa é importante para que haja uma evolução dos ambientes inteligentes, principalmente os que consideram o robô como um agente. Além da evolução dos ambientes inteligentes, manter o indivíduo com a melhor experiência de interação com o robô, e também faze-lo confortável com a presença do robô.

%%%%%%%%%%%%%%%%%%%%%%
\section{METODOLOGIA}
A fundamentação do trabalho é realizada em pesquisas de cada uma das áreas abrangentes, interação humano-robô (IHR), conceito de \emph{proxemics}, experiência de usuário aplicado a IHR, onde identificou-se a necessidade da criação de um projeto sistêmico para IHR. A partir desse projeto é possível determinar os passos para a construção do robô, especificação do contexto de uso, perfil de usuários para interação, e ferramentas de testes. O projeto foi submetido ao comitê de ética o projeto para aprovação dos testes com seres humanos.

A primeira bateria de testes foi realizada. A partir dos resultados dos testes piloto, é aplicado o algoritmo QG-SIM para construção dos grupos de perfis similares. Com cada grupo identificado, são criadas as Personas que devem ser classificadas pelo robô com base nas informações obtidas através das variáveis de observação identificadas na fase de concepção do método proposto. A partir desse ponto, é realizado um estudo de caso para viabilizar a aplicação do método na construção real de um projeto de interação humano-robô. As variáveis e observações feitas durante os testes pilotos são utilizadas para determinar as variáveis que compõem o classificador. Para o classificador é utilizado a técnica probabilística, rede Bayesiana. O objetivo é eliminar repetição das dependências condicionais apresentadas na construção da estrutura da rede.

Na sequência novos testes são realizados, para que seja possível a validação do classificador e das questões referentes ao perfil do usuário, demonstrando as fases de teste e análise do método proposto. O cenário de teste utilizado é um ambiente simulado de residência, onde o robô habita com mais uma pessoa. Realizados os testes, os resultados são analisados e discutidos, apresentando as estatísticas e observações obtidas durante o processo. Por fim, os próximos passos para o projeto são apresentados.

%%%%%%%%%%%%%%%%%%%%%%%%%%%%%%
\section{ESTRUTURA DO TRABALHO} % OK
Esta tese é composta por um total de 7 capítulos discriminados a seguir.

O capítulo \ref{cap:introducao} apresenta a \textbf{introdução} do trabalho conduzindo o leitor ao problema que a pesquisa desta tese deve contribuir.

O capítulo \ref{cap:ihr} introduz a área de \textbf{interação humano-robô}, contando um pouco da história e sua importância para o futuro.

O capítulo \ref{cap:ux} introduz a área de \textbf{interação humano-computador}, apresentando os principais conceitos para o desenvolvimento de um sistema centrado no usuário. Também é apresentado trabalhos que aplicam as técnicas em cenários de interação humano-robô.

O capítulo \ref{cap:projetoihr} apresenta a \textbf{especificação do método para construção de um projeto em interação humano-robô} apresentado para a construção de robô de serviço autônomo.

O capítulo \ref{cap:estudocaso} apresenta um \textbf{estudo de caso}  através da criação de um classificador para demonstra a aplicação do método proposto por esta tese.

O capítulo \ref{cap:resultados} apresenta os \textbf{resultados e discussões} desta tese.

O capítulo \ref{cap:conclusoes} apresenta as \textbf{conclusões e trabalho futuros} obtidos ao longo dos estudos e testes dessa tese.
