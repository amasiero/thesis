%!TEX root=Principal.tex
\chapter{CONCLUSÕES PARCIAIS}
\label{cap:conclusoes}
De acordo com os estudos realizados na literatura existente, é possível perceber que a criação de um \emph{framework} para interação humano-robô capaz de aprender e se adaptar ao comportamento de uma pessoa torna-se viável e essencial a partir do momento que a popularização da robótica está cada vez maior, principalmente em ambientes domésticos para fins de ajuda ao ser humano.

Para a interação ocorrer de maneira efetiva é necessário que o robô saiba respeitar os limites espaciais do ser humano e também ao realizar uma aproximação ou movimento em direção a pessoa, estes devem ser delicado o suficiente para que não gere nenhum desconforto ou medo. Por exemplo, durante uma apresentação do robô PeopleBot para alunos e professores do ensino médio, percebeu-se que a aproximação do robô pode causar um certo desconforto e medo dependendo, em especial quando a pessoa não estava esperando essa aproximação e não era avisada sobre a ação. 

Quando o robô se locomovia em direção a pessoa sem nenhum anúncio prévio, essa pessoa por muitas vezes ficava com medo. O medo em algumas situações observadas era tão evidente que a pessoa deixa o mesmo ambiente que o robô estava. Porém, quando o robô se aproximava e era anunciado pelo apresentador, as pessoas ficavam paradas deixando o robô chegar a alguns poucos centímetros dela. 

As observações a partir desse experimento reforçam a importância de ter um componente de interação adaptativo para que o robô possa identificar o perfil comportamental e personalidade do indivíduo de tal forma que eles possam conviver no mesmo ambiente em uma maneira confortável e sem medo por parte do ser humano.

Esse componente deve ainda ser capaz de transferir o conhecimento adquirido a partir de um robô para outros robôs, levando em consideração não só as características da pessoa, mas também as características do robô, pois esses fatores podem influenciar no comportamento das pessoas e robôs durante a interação. As características do robô são muito importantes para determinar a forma de interagir, já que existem robôs em diversos formatos como quadrutores, direção diferencial, bípede, quadrupede, com ou sem manipuladores, com tamanhos diferentes e também o nível de ruído de cada robô. Todas essas variáveis devem ser consideradas em estudos futuros, mas já devem estar contempladas pelo \emph{framework} que é um dos produtos finais dessa tese.