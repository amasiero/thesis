%!TEX root=Principal.tex
\chapter{CONCLUSÕES E TRABALHOS FUTUROS}
\label{cap:conclusoes}
De acordo com os estudos realizados na literatura existente, é possível perceber que a especificação de projetos focados em interação humano-robô, para criação de robôs que atendam as necessidades do usuário e proporcionem uma melhor experiência de usuário. Além disso, a adaptação do comportamento do robô dado o perfil de uma pessoa torna-se viável e essencial a partir do momento que a popularização da robótica está cada vez maior, principalmente em ambientes domésticos para fins de ajuda ao ser humano.

Para a interação ocorrer de maneira efetiva é necessário que o robô saiba respeitar os limites espaciais do ser humano e realizar movimentos naturais aos seres humanos. Todas as ações devem ser planejadas para que não gerem nenhum desconforto ou medo ao usuário. Por exemplo, durante a execução de uma tarefa de navegação do robô na casa simulada através da figura~\ref{fig:cenario}, percebeu-se que a aproximação do robô pode causar um certo desconforto. Esse desconforto é ocasionado, muitas vezes, pela invasão do espaço social do usuário, feita de maneira forçada e sem nenhum aviso prévio.

Além disso, cada pessoa tem uma reação de acordo com sua percepção sobre o robô. É importante a identificação do perfil de cada usuário que interage com o robô, para que a adaptação das ações seja possível. Essa tese apresenta um classificador de perfis de usuários utilizando as técnicas de rede bayesiana e Personas. O uso de Personas é fundamental, pois o alcance dos perfis dos usuários que são contemplados por cada Persona é maior do que o tratamento de perfil a perfil. Dessa maneira, o robô pode generalizar a adaptação do comportamento durante a interação.

A técnica de rede bayesiana é utiliza devido um perfil de usuário poder existir variações de comportamento durante a interação. Esse fenômeno gera incertezas na classificação que são tratadas através das probabilidades existentes na rede bayesiana. É a técnica que melhor trabalha com as incertezas de uma classificação de acordo com a literatura apresentada ao longo da tese.

A percepção do robô no ponto de vista de cada usuário leva ao agrupamento deles, pois existe uma similaridade entre seus comportamentos. Essa percepção apresenta sinais de conforto, desconforto e/ou medo durante a interação de maneira uniforme em um mesmo grupo. Além disso, variáveis criadas com base nas heurística de avaliação de usabilidade em interação humano-robô apresentaram tamanha significância durante a classificação das Personas.

Especificar de maneira sistêmica o projeto de interação humano-robô apresentou resultados importantes para a evolução e manutenção do robô. Substituição de partes do robô de maneira \emph{plug'n play} e adição de novos componentes de \emph{software} responsáveis pelo controle e inteligência do robô durante a execução da tarefa. É possível concluir que o classificador de Personas obteve um bom desempenho considerando as variáveis de interação utilizadas e ainda que elas são capazes de segmentas perfis de usuários.

Em relação a cultura, conclui-se que esta não pode ser sobreposta pela experiência do usuário, mesmo que ele afirme o contrário. Seu comportamento está alinhado de acordo com sua cultura origem ou de aprendizado. A partir deste ponto, é preciso criar mecanismos de aprendizado para auxiliar na distribuição dos valores de probabilidade condicional. A criação de um mecanismo de tomada de decisão, a partir da classificação da Persona em interação, para que seja adaptado o comportamento do robô é impressindivél para o avanço das interações sociais. Além disso, trabalhos referentes a aparência do robô devem ser trabalhados com uma certa urgência. Por fim, componentes para identificação de cada variável da rede bayesiana de maneira automática devem ser construídos e adicionados no robô.
