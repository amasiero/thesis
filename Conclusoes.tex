%!TEX root=Principal.tex
\chapter{CONCLUSÕES E TRABALHOS FUTUROS}
\label{cap:conclusoes}
De acordo com os estudos realizados na literatura existente, é possível perceber que a especificação de projetos focados em interação humano-robô, para criação de robôs que atendam as necessidades do usuário e proporcionem uma melhor experiência de usuário. Além disso, a adaptação do comportamento do robô dado o perfil de uma pessoa torna-se viável e essencial a partir do momento que a popularização da robótica está cada vez maior, principalmente em ambientes domésticos para fins de ajuda ao ser humano.

Para a interação ocorrer de maneira efetiva é necessário que o robô saiba respeitar os limites espaciais do ser humano e realizar movimentos naturais aos seres humanos. Todas as ações devem ser planejadas para que não gerem nenhum desconforto ou medo ao usuário. Por exemplo, durante a execução de uma tarefa de navegação do robô na casa simulada através da figura~\ref{fig:cenario}, percebeu-se que a aproximação do robô pode causar desconforto ao usuário. Esse desconforto é ocasionado, muitas vezes, pela invasão do espaço social do usuário, feita de maneira forçada e sem nenhum aviso prévio.

Além disso, cada pessoa tem uma reação de acordo com sua percepção sobre o robô. É importante a identificação do perfil de cada usuário que interage com o robô, para que a adaptação das ações seja possível. Essa tese apresenta um classificador de perfis de usuários utilizando as técnicas de rede bayesiana e Personas. O uso de Personas é fundamental, pois o alcance dos perfis dos usuários que são contemplados por cada Persona é maior do que o tratamento de pessoa a pessoa. Dessa maneira, o robô pode generalizar a adaptação do comportamento durante a interação com cada perfil. 

A técnica de rede bayesiana é utiliza devido um perfil de usuário poder existir variações de comportamento durante a interação. Esse fenômeno gera incertezas na classificação que são tratadas através das probabilidades existentes na rede bayesiana. É a técnica que melhor trabalha com as incertezas de uma classificação de acordo com a literatura apresentada ao longo da tese.

A percepção do robô no ponto de vista de cada usuário leva ao agrupamento deles, pois existe uma similaridade entre seus comportamentos. Cada grupo de perfil que formou uma Persona, percebeu as ações do robô e o avaliou de acordo com sua personalidade. A percepção e avaliação das ações e comportamento do robô feita pelos participantes, proporcionou um cruzamento de informações com as heurísticas de avaliação de usabilidade. Essa percepção auxiliou nas informações referentes a sinais de conforto, desconforto e/ou medo durante a interação de maneira uniforme em um mesmo grupo. Além disso, variáveis criadas com base nas heurística de avaliação de usabilidade em interação humano-robô apresentaram valores significativos para auxiliar na classificação das Personas. Isso valida a hipótese de que a percepção do usuário e as ações do robô auxiliam nesta tarefa.

Especificar de maneira sistêmica o projeto de interação humano-robô apresentou resultados importantes para a evolução e manutenção do robô. Substituição de partes do robô de maneira \emph{plug'n play} e adição de novos componentes de \emph{software} responsáveis pelo controle e inteligência do robô durante a execução da tarefa. É possível concluir que o classificador de Personas obteve um bom desempenho considerando as variáveis de interação utilizadas e ainda que elas são capazes de segmentas perfis de usuários.

Em relação a cultura, conclui-se que esta não pode ser sobreposta pela experiência do usuário, mesmo que o usuário afirme pertencer a uma cultura de sua preferência. Seu comportamento está alinhado de acordo com sua cultura origem ou de aprendizado. O ponto cultural permanece o mesmo já validado através da literatura de interação humano-roô e também interação humano-computador. A partir deste ponto, é preciso criar mecanismos de aprendizado para auxiliar na distribuição dos valores de probabilidade condicional. A criação de um mecanismo de tomada de decisão, a partir da classificação da Persona em interação, para que seja adaptado o comportamento do robô é impressindivél para o avanço das interações sociais. Além disso, trabalhos referentes a aparência do robô devem ser considerados para melhorar o processo de interação. Por fim, componentes para identificação de cada variável da rede bayesiana de maneira automática devem ser construídos e adicionados no robô.

Assim, defende-se a tese que as heurísticas de avaliação de usabilidade, aplicadas a interação humano-robô, auxiliaram a entender a percepção dos usuários sobre o sistema. E que cada usuário enfatizou algumas características do sistema que os segmentaram dentre as heurísticas utilizadas. A percepção que o usuário obteve, como apresentado na seção~\ref{sec:tpc} e discutido nos resultados, definiu o maneira e classifica-lo em um perfil diferente. Isso ocorreu pela importância dos fatores que o sistema apresentou de acordo com sua crença, de maneira natural. Esse é um resultado que valida a hipótese principal desta tese, sobre a percepção do usuário ser diferente para cada perfil, de acordo com as ações do sistema (robô).

Retomando os objetivos da tese, foi entregue um classificador de perfil do usuário utilizando informações de comportamento do sistema e a percepção do usuário sobre esse comportamento. O pacote do ROS utilizado para os experimentos dessa tese, também está disponível e funcional, através do endereço \url{https://github.com/amasiero/approach\_control}. Além disso, os capítulos~\ref{cap:projetoihr} e \ref{cap:proposta} apresentam uma metodologia para desenvolvimento de projetos em interação humano-robô, onde o projeto tenha uma fácil manuntenção e também consiga apresentar uma evolução de maneira escalável.

A partir desse ponto, trabalhos futuros podem auxiliar a evoluir a metodologia proposta e o classificador bayesiano. O primeiro ponto é a criação de um algoritmo de aprendizado para as probabilidades da rede bayesiana, de acordo com as interações ocorridas com o robô. Para cada nó da rede, pode-se criar um algoritmo para que a identificação das ações e reações na interação sejam automáticas, por exemplo, determinar se o usuário está confortável ou está com medo na interação. Novas variáveis podem ser investigadas para entender em mais detalhes a percepção do usuário sobre o sistema. Sejam essas novas variáveis baseadas em heurísticas ou em outras fontes de informação. Também pode-se aplicar de maneira mais sistêmica, metodologias ágeis de \emph{software} para auxiliar na gestão de manutenção e construção de projetos robóticos, principalmente voltado a interação. Além do estudo com diferentes robôs e comportamentos iguais para identificar o quanto a aparência influencia dado um comportamento desenvolvido com o intuito de ser centrado no usuário.