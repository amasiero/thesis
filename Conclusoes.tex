%!TEX root=Principal.tex
\chapter{CONCLUSÕES E TRABALHOS FUTUROS}
\label{cap:conclusoes}
Durante a leitura dos trabalhos em projetos de interação humano-robô, observou-se que existe uma falta de sistematização ao construir o robô e \textit{software} para este. Em engenharia de \textit{software} e usabilidade (principalmente projeto centrado no usuário) existem ferramentas que podem auxiliar na concepção e construção desse tipo de projeto. Assim, essa tese apresenta um método centrado no usuário para construção de robôs de serviços autônomos voltados a interação social. Com a apresentação desse método, torna-se mais organizado a criação, evolução e manutenção de um projeto de interação humano-robô, tendo os interesses do usuário como o foco na interação. Esse tipo de preocupação é essencial a partir do momento que a popularização da robótica torna-se cada vez maior, principalmente a robótica de serviço para ambientes domésticos onde os fins são para ajudar o ser humano.

Quando observa-se a interação, cada pessoa reage de acordo com sua percepção sobre o robô. Dessa maneira, a identificação do perfil do usuário é importante, pois torna possível o direcionamento do sistema robótico fazendo com que a experiência do usuário seja melhor e com mais qualidade. Essa tese apresenta um método centrado no usuário para interação com robôs de serviços autônomos. Um estudo de caso é realizado no cenário doméstico onde o robô simula a convivência com uma pessoa dentro de um apartamento. Nesse cenário é criado um classificador de perfis de usuários utilizando as técnicas de rede Bayesiana e Personas. O uso de Personas é fundamental, pois é uma maneira de manter o foco do projeto no usuário e ao mesmo tempo potencializar alcance dos perfis. Isso ocorre, pois a técnica de Personas consegue abranger um número muito maior de usuários e seu processamento é menor do que o tratamento de pessoa a pessoa. 

A percepção do robô no ponto de vista de cada usuário leva ao agrupamento deles, pois existem similaridades entre seus comportamentos. Cada grupo de perfil que formou uma Persona, percebeu as ações do robô e o avaliou de acordo com sua personalidade. A percepção e avaliação das ações e comportamento do robô feita pelos participantes, proporcionou um cruzamento de informações que resultaram em algumas variáveis de observação com base nas heurísticas de avaliação de usabilidade. Essa percepção auxiliou nas informações referentes a sinais de conforto, desconforto e/ou medo declaradas pelo grupo de maneira uniforme, durante a interação. Além disso, variáveis criadas com base nas heurística de avaliação de usabilidade em interação humano-robô apresentaram valores significativos para auxiliar na classificação das Personas. Essa característica proporcionou duas situações. A primeira, o usuário conseguir identificar e solicitar que o robô tenha características correspondentes as heurísticas de avaliação do usuário. A percepção do usuário, pode decorrer de uma nova maneira para classificar o perfil do usuário. A segunda é que as variáveis de observação sugeridas na seção~\ref{sec:variaveis} são consideradas como um ponto de partida para projetos de interação humano-robô, porém não deve ser descartado nenhum conjunto de variáveis que possam agregar valor ao mecanismo de tomada de decisão, e também as análises para melhorar o sistema para o usuário.

Especificar de maneira sistêmica o projeto de interação humano-robô apresentou resultados importantes para a evolução e manutenção do robô. Substituição de partes do robô de maneira \emph{plug'n play} e adição de novos componentes de \emph{software} responsáveis pelo controle e inteligência do robô durante a execução do projeto. O uso dos testes piloto para coleta de informações que posteriormente viram as Personas no projeto é importante para auxiliar na configuração mais fiel ao público alvo do projeto. Essa primeira análise auxilia a eliminar algumas tendências e comportamento do robô, onde podem ser ajustados antes dos testes efetivos e finais do ciclo de iteração proposto na metodologia de criação de projetos espiral.

A cultura pelo participante não sobrepõem a experiência do usuário, mesmo que o usuário afirme pertencer a uma cultura de sua preferência. Existe um enraizamento forte do comportamento do participante com o meio ao qual ele foi criado, que torna forte o seu vínculo com a cultura a qual ele foi criado. O ponto cultural permanece o mesmo já validado através da literatura de interação humano-robô e também interação humano-computador. Para um estudo mais afundo sobre a cultura dos participantes e sua influência sobre as ações do robô, é necessário um projeto utilizando o mesmo robô aplicado em diferentes continentes pelo mundo.

A criação de um mecanismo de tomada de decisão, a partir da classificação da Persona em interação, para que seja adaptado o comportamento do robô é imprescindível para o avanço das interações sociais. Além disso, trabalhos referentes a aparência do robô devem ser considerados para melhorar o processo de interação. Esses são pontos importantes e devem envolver algumas outras áreas como \textit{design} de projetos para potencializar a aparência do robô em interações que auxiliem a evolução do projeto de interação.

Após os testes, percebeu-se que o laser posicionado para desviar dos objetos mais baixos no cenário e auxiliar no controle da navegação esta em uma posição desfavorável. Foi notado esse ponto, quando o usuário teve seu pé atropelado pelas rodas da base do robô. Não houve lesão corporal ao participante, porém notou-se que o sensor deveria ser posicionado mais abaixo na base para evitar ruídos e conseguir enxergar melhor a área dos pés da pessoa. Para que isso seja realizado, é necessário informar no \textit{software} a posição exata do laser de modo que sejam calculadas as transformadas do laser para o centro do robô e não afete o mapeamento da área de atuação na navegação, além da leitura do laser para uma distância mais próxima da real. Em um projeto com arquitetura diferente da utilizada sobre as diretrizes do ROS (em camadas) essa alteração poderia ser custosa. Porém, na arquitetura proposta basta alterar parâmetros em um arquivo de configuração. Essa é mais uma das vantagens sobre o trabalho com um especificação de sistema mais formal.

Quanto ao projeto enviado ao comitê de ética, é importante ressaltar que toda a fase de concepção do método proposto possui os insumos necessários para a confecção da documentação. Entretanto, quando existir projetos que envolvam testes com crianças é importante consultar o comitê de ética, pois existe a obrigatoriedade de documentos a mais com um linguajar infantil e documento de autorização de participação aos pais. Esses documentos não foram contemplados nessa tese, uma vez que não houve a intenção de testes envolvendo crianças.

O pacote do ROS utilizado para os experimentos dessa tese, também está disponível e funcional, através do endereço \url{https://github.com/amasiero/approach\_control}. Esse pacote pode ser adaptado a qualquer base robótica, dado sua organização em camadas. Ele é direcionado as bases do PeopleBot e Kuka youBot, porém caso queira alterar para outra base, basta incluir na camada de \textit{driver} o pacote de comunicação com a base de preferência para o teste final.

A partir desse ponto, trabalhos futuros podem auxiliar a evoluir a metodologia proposta e o classificador Bayesiano, utilizado como estudo de caso. O primeiro ponto é a criação de um algoritmo de aprendizado para as probabilidades da rede Bayesiana, de acordo com as interações ocorridas com o robô. Para cada nó da rede, pode-se criar um algoritmo para que a identificação das ações e reações na interação sejam automáticas, por exemplo, determinar se o usuário está confortável ou está com medo na interação. Novas variáveis podem ser investigadas para entender em mais detalhes a percepção do usuário sobre o sistema. Sejam essas novas variáveis baseadas em heurísticas ou em outras fontes de informação. A evolução nas iterações do método para refinar as etapas propostas e consolidar com mais enfase as etapas descritas no capítulo~\ref{cap:projetoihr}. Além do estudo com diferentes robôs e comportamentos iguais para identificar o quanto a aparência influência dado um comportamento desenvolvido com o intuito de ser centrado no usuário. Outro ponto, a criação de um mecanismo de adaptação para o comportamento do robô de acordo com a Persona classificada e também uma maneira do robô assumir as características de um Persona com base no momento e cenário da interação.