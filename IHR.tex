%!TEX root=Principal.tex
\chapter{INTERAÇÃO HUMANO-ROBÔ}
\label{cap:ihr}
Interação Humano-Robô (IHR) é a área de estudo que procura compreender, avaliar e implementar robôs para que possam trabalhar em conjunto ou executar uma determinada tarefa onde a interação com o ser humano ocorra. A interação deve ser menos invasiva e mais colaborativa. O primeiro guia da IHR apareceu no conjunto de trabalhos de ficção científica de Isaac Asimov, que é apresentado como as primeiras leis da robótica por diversos prioneiros no tema. A primeira lei fala que um robô não pode ferir um ser humano e também deve proteje-lo para que nenhum mal o seja causado. A segunda lei diz que um robô deve obedecer as ordens dadas por seres humanos exceto nos casos que as ordens entrem em conflito com a primeira lei. E por fim a terceira lei diz um robô deve proteger sua própria existência desde que não entre em conflito com a primeira e/ou segunda leis. Essas leis regem os trabalhos em IHR até os dias atuais~\cite{goodrich:2007, weiss:2010}.

Qualquer tipo de robô possui um nível de interação, mesmo os completamente autônomos. A interação pode ocorrer de duas maneiras específicas: Interações Remotas (robôs e humanos em diferentes locais espaço-temporais), por exemplo, a operação do robô Curiosity\footnote{https://www.nasa.gov/mission\_pages/msl/index.html} em Marte e a NASA no planeta Terra; Interações Próximas (robôs e humanos estão no mesmo local, compartilhando o mesmo espaço), por exemplo, em indústrias ou residências como o robô Roomba~\cite{goodrich:2007}.

Robôs teleoperados são guiados por controles, como por exemplo \emph{joysticks} de video games. Já os robôs completamente autônomos devem consistir o ambiente, o cenário de atuação, agentes existentes no ambiente e os que estão direcionando-o para o seu objetivo final, além de atualizar constantemente esses dados e as restrições competentes. Muitos trabalhos são direcionados a interação através de um controle ou central de comando com a operação de um ser humano, mas a quantidade de trabalhos com robôs autônomos vêem crescendo principalmente em pesquisas de robótica assistiva e robótica para resgate em catástrofes, onde existem riscos a vida~\cite{goodrich:2007, weiss:2010}.

IHR é um estudo que necessita da participação de diversas outras áreas de pesquisa, como Ciências Cognitivas, Linguística, Psicologia, Antropologia, Engenharia, Ciências da Computação, Matemática, Engenharia dos Fatores Humanos e Design. É importante também, o estudo de padrões de interação adotadando pequenas perspectivas sobre soluções de problemas condizentes com a pesquisa, tornando mais fácil encontrar meios de corrigir um problema recorrente~\cite{goodrich:2007}.

Uma definição para interação é a atividade de trabalhar em conjunto para o mesmo objetivo. A IHR é afetada por cinco fatores de interação, que são: (I) Nível e comportamento de autonomia; (II) Troca natural de informação; (III) Estrutura do time; (IV) Adaptação, aprendizado e treinamento de pessoas e robôs; e (V) Definir as tarefas. Um robô que possui um grau de autonomia, consegue manter-se desatento por um período de continuar sua tarefa no mesmo ponto que parou. Contudo, em IHR a autonomia não é considerada com um resultado final, mas sim um meio que auxilia o processo de interação~\cite{goodrich:2007, weiss:2010}.

O nível de autonomia de um robô determina o quanto esse pode agir por conta própria. Existem diversas formas de medir e analisar esse nível. O mais utilizado é a escala de Sheridan~\cite{sheridan:1978} que apresenta um intervalo continuo desde de um robô que não realiza nenhuma tarefa por conta própria, ou seja, um robô teleoperado, até um robô totalmente independente e autônomo. Apesar do grande uso da escala de Sheridan, sua aplicabilidade ao cenário completo não é muito eficiente. Aconselha-se utilizar a escala dividindo o cenário em subtarefas~\cite{goodrich:2007, weiss:2010}.

Em IHR o nível de autonomia é melhor determinado por uma combinação entre o nível de interação com o humano e o quanto ambos, robô e pessoa, conseguem realizar tarefas de forma independente. O desenvolvimento de habilidades cognitivas é importante para o robô interagir com o humano de maneira natural e eficiente. Nos anos 80, Brooks apresentou um novo paradigma para autonomia de robôs, conhecida como robôs baseados em comportamento~\cite{brooks:1986, brooks:1991}. Outro modelo chamado de sinta-pense-aja também é apresentado na literatura como uma arquitetura híbrida que apresenta um problema de desenvolver comportamentos naturais e atividades robustas para robôs humanoides. Devido a isso, as áreas que trabalham no modelo cognitivo de aprendizagem e tomada de decisão tem crescido cada vez mais~\cite{goodrich:2007}.

Estudos de interação entre humanos e robôs não se limitam apenas ao nível de autonomia do robô. Modelos cognitivos, aplicações em ambientes sociais e principalmente em ambientes de cuidados pessoais, têm se tornado cada vez mais frequentes em novos estudos.

% Computação Afetiva
O tratamento de inteligência emocional em trabalho de IHR tendem a tornar as tarefas realizadas mais naturais. \citeonline{rani:2006} apresentam um modelo dos efeitos fisiológicos e correlaciona com os psicofisiológicos para que o robô seja capaz de inferir sobre o efeito da ansiedade nas pessoas. A partir desse modelo foi possível incentivar a melhora no desempenho de pessoas que tentavam fazer cestas em um jogo de basquete.

% Aprendizado por demonstração
\citeonline{giovannangeli:2007} apresentam um modelo de IHR onde o robô é capaz de aprender tarefas a partir de uma pessoa realizando o papel de treinador, onde o robô reproduz seus movimentos e consegue armazená-lo para situações futuras. No mesmo sentido, um trabalho com o robô Pepper da Softbank é apresentado por \citeonline{kitagawa:2016}. Nesse trabalho é realizado um controle de \emph{sleep} antes da reprodução dos gestos. Esse controle auxiliou na naturalidade da execução dos gestos e também um fato curioso, foi que a pessoa reproduzindo os gestos muitas vezes acabava por imitar o robô.

% Aparência
Outro fator importante para IHR é a aparência do robô em conjunto com a capacidade de execução de tarefas esperada para àquela aparência. Dessa maneira, \citeonline{minato:2007} apresentam uma plataforma robótica em formato de uma criança, mais precisamente um bebê, para realizar estudos de interação e principalmente a capacidade da cognição do robô durante a interação.

% Teorias de Psicologia (mente, ação)
Diversas teorias de psicologia também são aplicadas em trabalhos de IHR. Um exemplo é a teoria da mente que auxilia o robô na análise do comportamento de um indivíduo e possibilita a tomada de decisão para uma interação próxima a natural~\cite{hiatt:2011}. Assim como a teoria de ação, definida por Norman, que auxilia na definição de emoções humanas com base em ações tornando o comportamento do robô mais natural ao do ser humano~\cite{toumi:2013}.

% Plataformas de Teste
Utilizando um sensor de movimento Kinect foi criado uma plataforma de teste para interação humano-robô, onde o ser humano pode treinar o robô a distância e sem necessidade de contato físico. A idéia é poder fazer com que o robô não cause dano físico à pessoa e por consequência diminuir o medo de interação em participação de testes. A plataforma apresentada por \citeonline{rossmann:2013} transmite em tempo real o conhecimento dos gestos para o robô e este é reproduzido fielmente.

% Assistência Médica
Aplicações em serviços médicos também são explorados em IHR. \citeonline{briggs:2015} apresentam um estudo para auxiliar o tratamento de pessoas com a doença de Parkinson. Com essa doença o paciente pode perder a funcionalidade dos muscúlos faciais, levando assim a perda das expressões faciais. Quando um enfermeiro ou médico vai realizar o tratamento do paciente, pode interpretar que ele está desdenhando ou com nojo do profissional. A utilização do robô soluciona esse problema, já que o robô é capaz de filtrar as expressões faciais. Nos testes apresentados, os pacientes sentiram-se confortáveis com a interação junto ao robô NAO da Aldebaran.

% Colaboração
Trabalhos colaborativos são amplamente explorados entre os trabalhos de IHR. Muitos trabalhos vem promovendo debates sobre o assunto~\cite{strohkorb:2016,lampe:2016}. Alguns trabalhos já apresentam a integração com outros tipos de técnicas para desmontrar como que o robô pode assumir a liderança na tarefa ou apenas seguir as instruções da pessoa. Um trabalho nessa direção é apresentado por \citeonline{li:2015} que utiliza a teoria de jogos para solucionar o problema de lider ou seguidor na IHR.

% Casa Inteligente
Sistemas de detecção de anomalias com o propósito de servir idosos e pessoas com problemas físicos que vivem sozinhos é outro tema explorado. Após a detecção da anomalia com base no padrão de interação com o ambiente dessa pessoa, as entidades de assistência domésticas são acionadas e o processo de socorro da pessoa é iniciado. Para que esse trabalho atingisse uma boa acurácia (entre 80\% e 85\%), um robô móvel é utilizado em conjunto com senhores no ambiente de uma casa inteligente para detecção das anomalias~\cite{lundstrom:2015}.

% Educação
Como pode-se observar, a IHR tem sido aplicada em diversas áreas de atuação e uma que cresceu o número de trabalhos dedicados é a área da educação. \citeonline{martelaro:2016} investigam métodos para criação de comportamento do robô em função de aumentar a confiança entre humanos que interagem com robôs. Dois meios de medida para o aumento da confiança são aplicados, vulnerabilidade vs expressividade. Notou-se nos testes que robôs mais vulneráveis aumentam mais a confiança do que robôs mais expressivos. Os testes foram realizados com base em um robô tutor.

Outra proposta é identificar se crianças conseguem aprender melhor com o um tutor humano ou um robô. Estatisticamente, não houve significância considerável entre os dois resultados. Porém, o índice de Carson mostrou que o humano conseguiu ser melhor que o robô, principalmente na questão social. Um dos pontos que mais faltaram ao robô durante a interação com as crianças foi a questão de olhar mútuo. Apesar do estudo feito, não existe a ideia de substituir o ser humano com o robô, mas sim utilizar o robô como uma ferramenta complementar na sala de aula~\cite{kennedy:2016}.

% Robótica de Serviço
Um questionário é conduzido dentro de um hotel no Japão que utiliza de robôs para realizar alguns serviços. O objetivo é identificar se o robô é capaz de substituir uma pessoa nas tarefas. Entrevistaram em primeiro lugar o gerente do hotel e depois direcionaram as entrevistas às pessoas que tinham maior contato com o robô no dia-a-dia. Chegou-se a conclusão de que robôs podem substituir o ser humano nas tarefas, porém esse é um trabalho que deve ser realizado de maneira bem planejada~\cite{osawa:2017}.

Um robô com aspecto de humanoide foi desenvolvido para fazer a patrulha de um shopping durante a noite e durante o dia servir de apoio ao visitantes. A área de segurança deve ser muito explorada, pois a quantidade de pessoas treinadas para executar esse trabalho tem diminuído e as empresas não estão conseguindo repor a necessidade do mercado. Como patrulha, o robô teve um desempenho esperado, só que a atuação como cartão de boas vindas ao shopping foi além do esperado. Os resultados mostraram que as pessoas tiveram empatia pelo robô e acabou fazendo com que ele atraísse mais consumidores ao shopping~\cite{lopez:2017}. Isso demonstra que o robô pode apresentar múltiplas aplicações, apenas trocando as funções de seu programa.

% Aprendizado
Para que o robô seja capaz de realizar todas essas atividades, várias técnicas devem ser empregadas. Para realizar a personalização da interação do robô para cada pessoa, \citeonline{suga:2006} aplicam uma técnica de computação evolucionária interativa. Essa técnica funciona como um algoritmo evolucionário qualquer, porém a função \emph{fitness} é dada pela avaliação direta da pessoa. Como meio de melhorar essa função, é apresentado uma função híbrida onde uma parte dos genes são modificadas pelo usuário e outra parte modificada pelo próprio algoritmo usando como base as modificações manuais. Os resultados apontaram que o robô foi capaz de adaptar o comportamento de cada sujeito de teste.

Outro algoritmo evolucionário é aplicado para melhorar a aparência do robô com a ajuda e retorno sobre o gosto da pessoa. A fase de seleção do algoritmo era feita sobre a preferência do usuário que avaliava uma versão digital da morfologia do robô. Após o estacionamento da aparência do robô na otimização feita pelo algoritmo, o robô era confeccionado~\cite{debeir:2016}.

% Eyetracking
O uso da percepção sobre o olhar em uma interação entre humanos é comum e auxilia a melhorar a comunicação entre os indivíduos. A abordagem dessa característica é pouco utilizada em interação humano robô e geralmente é realizado com base na orientação da cabeça apenas. Utilizar a orientação da cabeça pode gerar um erro grande e uma noção de comunicação falha. Por isso, \citeonline{palinko:2016} desenvolveram um algoritmo de rastreamento do olho com baixo custo passivo para um robô humanoide. Os resultados apresentaram um bom rastreamento do movimento do olho. Agora estudos tendem a evoluir para o rastreamento da pupila e da posição da cabeça para melhorar a robustez do algoritmo.

% Contato físico
Tratando-se de IHR é impossível não tratar a questão do contato físico entre os agentes. Com base nessa premissa, foi criado um modelo de controle de impedância, para reduzir e controlar a força do robô após o contato com qualquer superfície. É utilizado um sensor de RGB-D (Kinect) para identificar o ponto de contato e em seguida o planejamento do movimento é feito. Caso haja um contato não planejado durante sua trajetória, o robô é capaz de identificar a pressão exercida, regulando assim a força exercida sobre a superfície mantendo a segurança na interação~\cite{magrini:2015}.

Outra questão investigada na interação com contato físico é o efeito de um abraço dado por um robô de pelúcia gigante sobre a vontade de fazer caridade das pessoas abraçadas. Resultados não foram estatisticamente significantes, porém acreditasse que é necessário realizar uma investigação melhor sobre essa questão~\cite{nakata:2017}.

% Falha de Sistema
É investigado também a questão sobre a preocupação com a segurança pessoal ou o custo financeiro dos danos causados pelo robô em caso de falha. Para conduzir o experimento alguns videos de situações e tarefas de interação humano-robô foram apresentados para pessoas. Após o video elas avaliavam o grau de criticidade de cada situação.
Os resultados apresentados são interessantes, pois as pessoas deram um nível maior de criticidade para o robô derrubando líquido no laptop do que ele esbarrar e machucar uma pessoa. Contudo, estudos mais realísticos devem ser efetuados~\cite{adubor:2017}.

% Privacidade, Segurança e Ética
Outras discussões também tem sido endereçadas na questão de robótica social, assistiva e de serviço. Com robôs entrando em nosso dia-a-dia começa a preocupação sobre questões de invasão de privacidade das pessoas por parte destes agentes. Questões éticas sobre a aplicação dos robôs no dia-a-dia. Essas linhas de pesquisa têm ganhado força em debates da comunidade~\cite{rueben:2017}.

Ao observar os trabalhos relacionados a IHR, pode-se perceber que quando existe uma interação social, o primeiro passo é a aproximação entre dois agentes. Dessa forma, existe a necessidade do robô aprender como se comportar, de acordo com algumas normas sociais, durante a aproximação de um ser humano. Assim, um modelo é apresentado com o objetivo principal voltado para o mapeamento do espaço social e também a análise do comportamento humano. Este modelo tem como sua essência um conceito apresentado por \citeonline{hall:1969}, chamado de \emph{Proxemics}. O modelo serve de base para essa tese e é apresentada em detalhes no capítulo~\ref{cap:proxemics}.
