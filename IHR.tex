%!TEX root=Principal.tex
\chapter{INTERAÇÃO HUMANO-ROBÔ}
\label{cap:ihr}
Interação Humano-Robô (IHR) é a área de estudo que procura compreender, avaliar e implementar os robôs para que possam trabalhar em conjunto ou para o ser humano onde a interação seja menos invasiva e mais colaborativa. O primeiro guia da IHR apareceu em um trabalho de ficção científica de Isaac Asimov, que apresentou as primeiras leis da robótica. A primeira lei fala que um robô não pode ferir um ser humano e também deve proteje-lo para que nenhum mal o seja causado. A segunda lei diz que um robô deve obedecer as ordens dadas por seres humanos exceto nos casos que as ordens entrem em conflito com a primeira lei. E por fim a terceira lei diz um robô deve proteger sua própria existência desde que não entre em conflito com a primeira e/ou segunda leis. Essas leis regem os trabalhos voltados a IHR até nos dias atuais~\cite{Goodrich:2007, Weiss:2010}. 

Qualquer tipo de robô necessita de interação, mesmo os completamente autônomos. A interação pode ocorrer de duas maneiras específicas: Interações Remotas (robôs e humanos estão em diferentes locais espaço-temporais), por exemplo, a operação do robô Curiosity\footnote{https://www.nasa.gov/mission\_pages/msl/index.html} em Marte e a NASA aqui no planeta Terra; Interações Próximas (robôs e humanos estão em um mesmo local, compartilhando o mesmo espaço), por exemplo, em indústrias ou residências como o robô Roomba~\cite{Goodrich:2007}. 

Robôs teleoperados devem ser guiados por controles, como \emph{joysticks}, por exemplo. Já os robôs completamente autônomos devem consistir o ambiente, o cenário de atuação, os seres humanos que existem no ambiente e os que estão direcionando-o para o seu objetivo final, além de atualizar constantemente informações sobre o ambiente e suas restrições. Muitos trabalhos são direcionados a interação através de um controle ou central de comando com a operação de um ser humano, mas a quantidade de trabalhos com robôs autônomos vêem crescendo principalmente em pesquisas de robótica assistiva e/ou robótica para resgate em catástrofes, onde existe riscos a vida de seres humanos que procuram por vitimas~\cite{Goodrich:2007, Weiss:2010}.

IHR é um estudo que necessita da participação de diversas outras áreas de pesquisa, como Ciências Cognitivas, Linguística, Psicologia, Engenharia, Ciências da Computação, Matemática, Engenharia dos Fatores Humanos e Design. Além disso, é importante o estudo de padrões de interação para que sejam adotados pequenas perspectivas sobre soluções de problemas com interação, tornando mais fácil encontrar uma solução a algum problema que seja recorrente~\cite{Goodrich:2007}.

A interação pode ser defina pela atividade de trabalhar em conjunto para atingir o mesmo objetivo. A IHR afetada cinco fatores de interação, que são: (I) Nível e comportamento de autonomia; (II) Troca natural de informação; (III) Estrutura do time; (IV) Adaptação, aprendizado e treinamento de pessoas e robôs; e (V) Definir as tarefas. Um robô que possui um grande grau de autonomia é aquele que consegue permanecer desatento por um longo período de tempo sem realizar nenhum tipo de interação. Contudo em IHR a autonomia não é considerada com um resultado final, mas é um meio que auxilia no processo de interação~\cite{Goodrich:2007, Weiss:2010}.

O nível de autonomia de um robô determina o quanto esse pode agir por sua própria conta. Existem diversas formas de medir e analisar esse nível. O mais utilizado é a escala de Sheridan~\cite{Sheridan:1978} que apresenta um intervalo continuo desde de um robô que não realiza nenhuma tarefa por conta própria, ou seja, um robô teleoperado até um robô totalmente independente e autônomo. Apesar do grande uso da escala de Sheridan, sua aplicabilidade ao cenário completo pode não ser muito eficiente sendo melhor aplicado em subtarefas~\cite{Goodrich:2007, Weiss:2010}.

Em IHR o nível de autonomia pode ser melhor determinado por uma combinação entre o nível de interação entre humano e robô e o quanto ambos conseguem realizar as tarefas de forma independente. O desenvolvimento de habilidades cognitivas é importante para o robô interagir com o humano de maneira natural e eficiente. Nos anos 80, Brooks apresentou um novo paradigma para autonomia de robôs, conhecida com robôs baseados em comportamento~\cite{Brooks:1986, Brooks:1991}. Outro modelo chamado de sinta-pense-aja também é apresentado na literatura como uma arquitetura híbrida que apresenta um problema de desenvolver comportamentos que sejam natural e atividades robustas para robôs humanoides. Devido a isso, as áreas que trabalham no modelo cognitivo de aprendizagem e tomada de decisão tem crescido cada vez mais~\cite{Goodrich:2007}. 

Contudo, o estudo de interação entre humanos e robôs não se limita apenas ao nível de autonomia do robô. Modelos cognitivos, aplicações em ambientes sociais e principalmente em ambientes de cuidados médicos pessoais, têm se tornado cada vez mais frequentes nos novos estudos. \citeonline{Giovannangeli:2007} apresentam um modelo de IHR onde o robô é capaz de aprender tarefas a partir de uma pessoa realizando o papel de treinador, onde o robô reproduz seus movimentos e consegue armazena-lo para situações futuras. A teoria da mente também é aplicada em trabalhos de IHR. Ela auxilia o robô na análise do comportamento de um indivíduo e possibilita a tomada de decisão para uma interação próxima a natural~\cite{Hiatt:2011}.

Outro fator importante para IHR é a aparência do robô em conjunto com a capacidade de execução de tarefas esperada para àquela aparência. Dessa maneira, \citeonline{Minato:2007} apresentam uma plataforma robótica em formata de uma criança, mais precisamente um bebê, para realizar estudos de interação e principalmente a capacidade da cognição do robô durante a interação. Um outro modelo é apresentado nos estudos de IHR com o objetivo principal voltado para o mapeamento e análise do comportamento humano. Este modelo tem como sua essência a teoria de \emph{Proxemics}, que serve de base para essa tese e é apresentada em detalhes na seção~\ref{sec:proxemics}, a seguir.

