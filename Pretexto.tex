%!TEX root=Principal.tex
\maketitle{}

\begin{folhaderosto}
Tese de Doutorado apresentada ao Centro Universitário FEI para obtenção do título de Doutor em Engenharia Elétrica, orientado pelo Prof. Dr. Plinio Thomaz Aquino Junior e coorientado pelo Prof. Dr. Flavio Tonidandel.
\end{folhaderosto}

\fichacatalografica
%\folhadeaprovacao

%%%%%%%%%%%%%%%%%%%%%%%%%%%%%%%%%%%%%%%%%%%%%%%%%%%%%%%%%

\dedicatoria{A Deus e a minha família que são o alicerce de minha vida.}

%%%%%%%%%%%%%%%%%%%%%%%%%%%%%%%%%%%%%%%%%%%%%%%%%%%%%%%%%
\begin{agradecimentos}
Em primeiro lugar gostaria de agradecer a Deus, que sempre me trouxe sabedoria e luz, mesmo nos momentos difíceis dessa jornada e de tantas outras.

À minha mãe Kathia, que desde o primeiro momento me apoiou e incentivou, mesmo quando tudo parecia impossível e eu não conseguia ver a luz no fim do túnel.

À minha irmã Andressa, que me suportou quando fiquei exaltado de felicidade ou tristeza perante as dificuldades.

Aos meus avós, Hélio e Rachel, que mesmo não presentes em carne, continuam iluminando minha vida e me guiam pelos caminhos que percorro deixando a sensação de sempre estar seguro.

Ao professor e orientador Plinio Thomaz Aquino Junior, que me auxilia a direcionar nos caminhos ao longo da jornada acadêmica e pessoal, com seus sábios conselhos e cumplicidade, fortalecendo a parceira a cada momento nesses últimos anos.

Ao professor e coorientador Flavio Tonidandel, que ajudou a tornar esse trabalho possível, com seus conselhos e ensinamentos, além de sempre puxar a minha orelha quando algo estava estranho ou elogiar sempre que eu conseguia um bom resultado. Tudo isso faz com que nossa parceria seja majestosa, desde a época do mestrado.

Aos professores da FEI, que compartilharam ao longo desse período seus conhecimentos e amizade, ajudando na evolução desse trabalho e também a minha como pessoa.

Aos meus amigos, que sem esse laço seria impossível avançar mais um passo neste caminho cheio de curvas. Os momentos de descontração, de discussão, almoços e principalmente cafés foram e são de extrema importância para nos ajudar a andar no caminho chamado vida.

E por fim a todos que de alguma maneira contribuíram para mais essa conquista.

\end{agradecimentos}

%%%%%%%%%%%%%%%%%%%%%%%%%%%%%%%%%%%%%%%%%%%%%%%%%%%%%%%%%
\epigrafe{O olho vê, a lembrança revê, e a imaginação transvê. É preciso transver o mundo}{Manoel de Barros}

%%%%%%%%%%%%%%%%%%%%%%%%%%%%%%%%%%%%%%%%%%%%%%%%%%%%%%%%%
\begin{resumo}
Com o passar dos anos é possível acompanhar a evolução dos sistemas computacionais. Por exemplo telefones móveis, computadores pessoais e portáteis, televisões, e também os robôs para tarefas domésticas, como aspirar pó e assistência pessoal. Contudo, o robô inserido no ambiente doméstico atual, é identificado apenas como um dispositivo tecnológico que uma pessoa possui. Aos poucos um robô autônomo que realize tarefas domésticas e cuidados médicos pessoais deve ser inserido no ambiente. Ele deve realizar interações através de voz, gestos e toque físico com o ser humano. A partir desse momento o sentimento não será mais de um dispositivo tecnológico, mas sim de um novo membro no ambiente. Porém, uma preocupação é como prover uma boa experiência ao ser humano durante a interação. Para que essa boa experiência ocorra, é necessário que identifique o perfil da pessoa que está interagindo com o robô ou convivendo com ele. Essa tese propõem um classificador bayesiano do perfil do usuário para auxiliar na interação do robô com o ser humano. Para construir o classificador são utilizados metodologias de desenvolvimento \emph{software} para determinar o contexto de uso, variáveis utilizadas, projeto do robô, entre outros aspectos. O uso dessas metodologias auxiliam na manutenção e evolução do projeto ao longo do seu ciclo de vida. O cenário de teste é baseado em tarefas que o robô deve executar durante a convivência em uma residência. O classificador apresentou um resultado com taxa de acerto de 68,5\%. Ajustes nas tabelas de probabilidades condicionais devem ser feitos para melhorar essa taxa de acerto. O classificador utilizou informações das ações do robô e sobre a percepção do usuário. O projeto avaliou o conforto e o medo do usuário durante a aproximação do robô. Essas informações foram declaradas pelo usuário durante os procedimentos do teste. O resultado da interação foi positivo para a maioria dos usuários.


\palavraschave{Robótica Social, Proxemics, Redes Bayesianas, Interação Humano-Robô, QG-SIM}

\end{resumo}
%%%%%%%%%%%%%%%%%%%%%%%%%%%%%%%%%%%%%%%%%%%%%%%%%%%%%%%%%
\begin{abstract}
Over the years, it is possible to follow the evolution of computational systems. For example mobile phones, personal and portable computers, televisions. As well as robots for housework such as vacuuming for dust and personal help. But, robots in the current domestic environment is only identified as technological device.  A standalone robot must be inserted into the environment. It must perform interactions through voice, gestures and physical touch. From that moment the feeling will no longer be a technological device. But, a concern is how to provide a good experience to the human being during the interaction. For a good experience, it is necessary to identify the profile of the person. This thesis proposes a Bayesian classifier of the user profile to aid in the interaction. To build the classifier, it uses software development methodologies. It determines the context of use, variables used, robot design, among other aspects. The use of these methodologies assists the maintenance and evolution of the project. The based test scenario is on tasks that the robot must perform during living in a house. The classifier presented a result with a success rate of 68.5\%. It is necessary some  adjustments in the conditional probability tables. It will improve this hit rate. The classifier used information about the robot's actions and about the user's perception. The project evaluated the comfort and fear of the user during the approach of the robot. This information declared by user during the test procedures. The result of the interaction was positive for most users.

\keywords{Social Robotic, Proxemics, Machine Learning, Human-Robot Interaction}
\end{abstract}



%%%%%%%%%%%%%%%%%%%%%%%%%%%%%%%%%%%%%%%%%%%%%%%%%%%%%%%%%
\listoffigures
\listoftables
% \listofalgorithms
% \printglossaries

\tableofcontents

%%%%%%%%%%%%%%%%%%%%%%%%%%%%%%%%%%%%%%%%%%%%%%%%%%%%%%%%%
