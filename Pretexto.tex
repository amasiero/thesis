%!TEX root=Principal.tex
\maketitle{}

\begin{folhaderosto}
Tese de Doutorado apresentada ao Centro Universitário FEI para obtenção do título de Doutor em Engenharia Elétrica, orientado pelo Prof. Dr. Plinio Thomaz Aquino Junior e coorientado pelo Prof. Dr. Flavio Tonidandel.
\end{folhaderosto}

%\fichacatalografica
%\folhadeaprovacao

%%%%%%%%%%%%%%%%%%%%%%%%%%%%%%%%%%%%%%%%%%%%%%%%%%%%%%%%%

\dedicatoria{A Deus e a minha família que são o alicerce de minha vida.}

%%%%%%%%%%%%%%%%%%%%%%%%%%%%%%%%%%%%%%%%%%%%%%%%%%%%%%%%%
\begin{agradecimentos}
Em primeiro lugar gostaria de agradecer a Deus, que sempre me trouxe sabedoria e luz, mesmo nos momentos difíceis dessa jornada e de tantas outras.

À minha mãe Kathia, que desde o primeiro momento me apoiou e incentivou, mesmo quando tudo parecia impossível e eu não conseguia ver a luz no fim do túnel.

À minha irmã Andressa, que me suportou quando fiquei exaltado de felicidade ou tristeza perante as dificuldades.

Aos meus avós, Hélio e Rachel, que mesmo não presentes em carne, continuam iluminando minha vida e me guiam pelos caminhos que percorro deixando a sensação de sempre estar seguro.

Ao professor e orientador Plinio Thomaz Aquino Junior, que me auxilia a direcionar nos caminhos ao longo da jornada acadêmica e pessoal, com seus sábios conselhos e cumplicidade, fortalecendo a parceira a cada momento nesses últimos anos.

Ao professor e coorientador Flavio Tonidandel, que ajudou a tornar esse trabalho possível, com seus conselhos e ensinamentos, além de sempre puxar a minha orelha quando algo estava estranho ou elogiar sempre que eu conseguia um bom resultado. Tudo isso faz com que nossa parceria seja majestosa, desde a época do mestrado.

Aos professores da FEI, que compartilharam ao longo desse período seus conhecimentos e amizade, ajudando na evolução desse trabalho e também a minha como pessoa.

Aos meus amigos, que sem esse laço seria impossível avançar mais um passo neste caminho cheio de curvas. Os momentos de descontração, de discussão, almoços e principalmente cafés foram e são de extrema importância para nos ajudar a andar no caminho chamado vida.

E por fim a todos que de alguma maneira contribuíram para mais essa conquista.

\end{agradecimentos}

%%%%%%%%%%%%%%%%%%%%%%%%%%%%%%%%%%%%%%%%%%%%%%%%%%%%%%%%%
\epigrafe{O olho vê, a lembrança revê, e a imaginação transvê. É preciso transver o mundo}{Manoel de Barros}

%%%%%%%%%%%%%%%%%%%%%%%%%%%%%%%%%%%%%%%%%%%%%%%%%%%%%%%%%
\begin{resumo}
EM CONSTRUÇÃO


\palavraschave{Engenharia de Software, Projeto centrado no Usuário, Robótica de Serviço, Interação Humano-Robô, Personas}

\end{resumo}
%%%%%%%%%%%%%%%%%%%%%%%%%%%%%%%%%%%%%%%%%%%%%%%%%%%%%%%%%
\begin{abstract}
UNDER CONSTRUCTION

\keywords{Software Engineering, User Centred Design, Service Robotics, Human-Robot Interaction, Personas}
\end{abstract}



%%%%%%%%%%%%%%%%%%%%%%%%%%%%%%%%%%%%%%%%%%%%%%%%%%%%%%%%%
\listoffigures
\listoftables
% \listofalgorithms
% \printglossaries

\tableofcontents

%%%%%%%%%%%%%%%%%%%%%%%%%%%%%%%%%%%%%%%%%%%%%%%%%%%%%%%%%
