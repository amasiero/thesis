%!TEX root=Principal.tex
\maketitle{}

\begin{folhaderosto}
Tese de Doutorado apresentada ao Centro Universitário FEI para obtenção do título de Doutor em Engenharia Elétrica, orientado pelo Prof. Dr. Plinio Thomaz Aquino Junior e coorientado pelo Prof. Dr. Flavio Tonidandel.
\end{folhaderosto}

%\fichacatalografica
%\folhadeaprovacao

%%%%%%%%%%%%%%%%%%%%%%%%%%%%%%%%%%%%%%%%%%%%%%%%%%%%%%%%%

\dedicatoria{A Deus e a minha família que são o alicerce de minha vida.}

%%%%%%%%%%%%%%%%%%%%%%%%%%%%%%%%%%%%%%%%%%%%%%%%%%%%%%%%%
\begin{agradecimentos}
Em primeiro lugar gostaria de agradecer a Deus, que sempre me trouxe sabedoria e luz, mesmo nos momentos difíceis dessa jornada e de tantas outras.

À minha mãe Kathia, que desde o primeiro momento me apoiou e incentivou, mesmo quando tudo parecia impossível e eu não conseguia ver a luz no fim do túnel.

À minha irmã Andressa, que me suportou quando fiquei exaltado de felicidade ou tristeza perante as dificuldades.

Aos meus avós, Hélio e Rachel, que mesmo não presentes em carne, continuam iluminando minha vida e me guiam pelos caminhos que percorro deixando a sensação de sempre estar seguro.

Ao professor e orientador Plinio Thomaz Aquino Junior, que me auxilia a direcionar os caminhos ao longo da jornada acadêmica e pessoal, com seus sábios conselhos e cumplicidade, fortalecendo a parceira a cada momento nesses últimos anos.

Ao professor e coorientador Flavio Tonidandel, que ajudou a tornar esse trabalho possível, com seus conselhos e ensinamentos, além de sempre puxar a minha orelha quando algo estava estranho ou elogiar sempre que eu conseguia um bom resultado. Tudo isso faz com que nossa parceria seja majestosa, desde a época do mestrado.

Aos professores da FEI, que compartilharam ao longo desse período seus conhecimentos e amizade, ajudando na evolução desse trabalho e também a minha como pessoa.

Aos meus amigos, que sem esse laço seria impossível avançar mais um passo neste caminho cheio de curvas. Os momentos de descontração, de discussão, almoços e principalmente cafés foram e são de extrema importância para nos ajudar a andar no caminho chamado vida.

E por fim a todos que de alguma maneira contribuíram para mais essa conquista.

\end{agradecimentos}

%%%%%%%%%%%%%%%%%%%%%%%%%%%%%%%%%%%%%%%%%%%%%%%%%%%%%%%%%
\epigrafe{O olho vê, a lembrança revê, e a imaginação transvê. É preciso transver o mundo}{Manoel de Barros}

%%%%%%%%%%%%%%%%%%%%%%%%%%%%%%%%%%%%%%%%%%%%%%%%%%%%%%%%%
\begin{resumo}
O cenário da robótica no mundo tem mudado com o passar dos últimos anos. Antes o ambiente de robô tinha seu foco apenas na indústria, e agora existe uma ênfase maior nos chamados robôs de serviços. A \textit{International Federation of Robotics} (IFR) define robôs de serviços como qualquer robô que esteja fora de um cenário industrial. Eles encontram-se em áreas como agricultura, hospitais, hotéis, escritórios e residências. Dessa maneira, a necessidade de interação entre os robôs e os seres humanos torna-se inevitável. Assim, pesquisadores têm se dedicado a criação de projetos cada vez mais preocupados com a interação social entre humanos e robô. Técnicas de experiência do usuário, controle e inteligência artificial são empregadas nos trabalhos de interação humano-robô com o intuito garantir mais qualidade no contato entre ambos. Entretanto, os projetos de interação humano-robô discutidos na literatura têm foco em pequenas partes da interação, como controle de toque, naturalidade dos gestos e movimentos, desvios de obstáculos, reconhecimento de pessoas, entre outros. Essas partes se preocupam mais com o comportamento e habilidades do robô do que com o usuário. Há a falta de um método sistêmico que possibilite a documentação, evolução e manutenção de robôs autônomos de serviço. Além do método sistêmico, um método que seja centrado no usuário é importante dada a atividade conjunta ao ser humano. Essa tese apresenta um método centrado no usuário para a construção de um robô autônomo, que tem como foco a interação entre humanos e robô. O método proposto é aplicado em um estudo de caso baseado em um cenário doméstico, onde o robô interage através de voz e possui navegação totalmente autônoma. A técnica de Personas é utilizada para identificar o perfil do usuário e um classificador Bayesiano é apresentado como meio de ilustrar a proposta de uma parte da etapa de tomada de decisão do robô. Os resultados mostram a importância de método, principalmente pela evolução do projeto e manutenções que foram necessárias durante os testes na aplicação do estudo de caso.

\palavraschave{Engenharia de Software, Projeto centrado no Usuário, Robótica de Serviço, Interação Humano-Robô, Personas}

\end{resumo}
%%%%%%%%%%%%%%%%%%%%%%%%%%%%%%%%%%%%%%%%%%%%%%%%%%%%%%%%%
\begin{abstract}
The robotics scene in the world has changed over the last few years. Before the robot environment had its focus only on the industry. Now there is a greater emphasis on so-called service robots. The International Federation of Robotics (IFR) defines service robots as any robot out of an industrial setting. They are in areas such as agriculture, hospitals, hotels, offices and residences. In this way, the need for interaction between robots and humans becomes inevitable. Thus, researchers have dedicated to creating projects concerned with social human-robot interaction.  The human-robot interaction works to guarantee more quality in interaction. It uses techniques of user experience, control and artificial intelligence. Human-robot interaction discussed in the literature focus on small parts of the project. Some examples are: touch control, naturalness of gestures and movements, deviations of obstacles, recognition of people, among others. These parties are more concerned with the behavior and skills of the robot than with the user. There is a lack of a systemic method that enables the documentation, evolution and maintenance of autonomous service robots. Besides to the systemic method, one that is user centered is important given the joint activity to the human being. This thesis presents a user-centered method for constructing an autonomous robot. It focus on human-robot interaction. The proposed method is applied in a case study based on a domestic scenario, where the robot interacts through voice and has autonomous navigation. The Personas technique is used to identify the user profile. A Bayesian classifier is presented as a way to illustrate the proposal of a part of the decision making stage of the robot. The results show the importance of method, by the design evolution. Also, maintenance that were necessary during the tests in the application of the case study.

\keywords{Software Engineering, User Centred Design, Service Robotics, Human-Robot Interaction, Personas}
\end{abstract}



%%%%%%%%%%%%%%%%%%%%%%%%%%%%%%%%%%%%%%%%%%%%%%%%%%%%%%%%%
\listoffigures
\listoftables
% \listofalgorithms
% \printglossaries

\tableofcontents

%%%%%%%%%%%%%%%%%%%%%%%%%%%%%%%%%%%%%%%%%%%%%%%%%%%%%%%%%
