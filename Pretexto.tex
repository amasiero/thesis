%!TEX root=Principal.tex
\maketitle{}

\begin{folhaderosto}
Tese de Doutorado apresentada ao Centro Universitário da FEI para obtenção do título de Doutor em Engenharia Elétrica, orientado pelo Prof. Dr. Plinio Thomaz Aquino Junior e coorientado pelo Prof. Dr. Flavio Tonidandel.
\end{folhaderosto}

%\fichacatalografica
%\folhadeaprovacao

%%%%%%%%%%%%%%%%%%%%%%%%%%%%%%%%%%%%%%%%%%%%%%%%%%%%%%%%%

\dedicatoria{A Deus e a minha família que são o alicerce de minha vida.}

%%%%%%%%%%%%%%%%%%%%%%%%%%%%%%%%%%%%%%%%%%%%%%%%%%%%%%%%%
\begin{agradecimentos}
Em primeiro lugar gostaria de agradecer a Deus, que sempre me trouxe sabedoria e luz, mesmo nos momentos difíceis dessa jornada e de tantas outras.

À minha mãe Kathia, que desde o primeiro momento me apoiou e incentivou, mesmo quando tudo parecia impossível e eu não conseguia ver a luz no fim do túnel.

À minha irmã Andressa, que me suportou quando fiquei exaltado de felicidade ou tristeza perante as dificuldades.

Aos meus avós, Hélio e Rachel, que mesmo não presentes em carne, continuam iluminando minha vida e me guiam pelos caminhos que percorro deixando a sensação de sempre estar seguro.

Ao professor e orientador Plinio Thomaz Aquino Junior, que me auxilia a direcionar nos caminhos ao longo da jornada acadêmica e pessoal, com seus sábios conselhos e cumplicidade, fortalecendo a parceira a cada momento nesses últimos anos.

Ao professor e coorientador Flavio Tonidandel, que ajudou a tornar esse trabalho possível, com seus conselhos e ensinamentos, além de sempre puxar a minha orelha quando algo estava estranho ou elogiar sempre que eu conseguia um bom resultado. Tudo isso faz com que nossa parceria seja majestosa, desde a época do mestrado.

Aos professores da FEI, que compartilharam ao longo desse período seus conhecimentos e amizade, ajudando na evolução desse trabalho e também a minha como pessoa.

Aos meus amigos, que sem esse laço seria impossível avançar mais um passo neste caminho cheio de curvas. Os momentos de descontração, de discussão, almoços e principalmente cafés foram e são de extrema importância para nos ajudar a andar no caminho chamado vida.

E por fim a todos que de alguma maneira contribuíram para mais essa conquista.

\end{agradecimentos}

%%%%%%%%%%%%%%%%%%%%%%%%%%%%%%%%%%%%%%%%%%%%%%%%%%%%%%%%%
\epigrafe{O olho vê, a lembrança revê, e a imaginação transvê. É preciso transver o mundo}{Manoel de Barros}

%%%%%%%%%%%%%%%%%%%%%%%%%%%%%%%%%%%%%%%%%%%%%%%%%%%%%%%%%
\begin{resumo}
A evolução da tecnologia torna-se cada vez mais evidente com o passar dos anos. As pessoas possuem computadores portáteis menores e com melhor configuração, \emph{tablets}, aparelhos de telefonia móvel inteligentes interligados com relógios e também robôs que possuem tarefas específicas como aspirar o pó da casa ou monitorar o ambiente a partir de um determinado ponto. Contudo, o robô inserido no ambiente doméstico ou pessoal atual, é apenas mais um dispositivo tecnológico que a pessoa possui. Caso um robô autônomo capaz de realizar diversas tarefas domésticas e de cuidados pessoais médicos seja inserido nesse ambiente e ainda ele realize interações através de voz, gestos e toque com o ser humano, o sentimento a partir desse momento não seria mais de um dispositivo tecnológico no ambiente. Existe uma possibilidade do ser humano ficar de uma certa maneira desconfortável com a presença do robô. Considerando a situação de desconforto do ser humano com o robô, essa tese propõem uma metodologia que mapeia o conjunto de ações que o robô é capaz de executar visando a maximização da probabilidade de uma interação humano-robô com maior qualidade, baseando-se no comportamento e características do indivíduo. A partir do mapeamento de comportamento da pessoa é possível determinar o comportamento que o robô deve ter para proporcionar uma situação confortável para a interação com o ser humano. Como resultado espera-se um \emph{framework} que possa aprender e analisar o comportamento do ser humano e que também seja capaz de transferir esse conhecimento com o robô inserido no ambiente, aumentando a eficácia da interação entre humanos e robôs.

\palavraschave{Robótica Social, Proxemics, Aprendizado de Máquina, Interação Humano-Robô}
\end{resumo}
%%%%%%%%%%%%%%%%%%%%%%%%%%%%%%%%%%%%%%%%%%%%%%%%%%%%%%%%%
\begin{abstract}
The technology's evolution has increased over the years. People have smaller laptops with better set up, tablets, smartphones interconnected with watches and also robots, which have specific tasks such as vacuuming or monitoring the environment from a certain point. However, the robot inserted into the current household or staff, is just another technological device that the person has. If an autonomous robot, able to perform various household chores and personal care doctors to be entered in this environment and still perform it interactions via voice, gestures and touch with the human being, the feeling would be no more than a technological device into the environment. There is a possibility of human beings in a way become uncomfortable with the presence of the robot. Considering the uncomfortable situation of the human being with the robot, this thesis proposes a methodology that maps the set of actions that the robot is able to perform in order to maximize the likelihood of human-robot interaction with higher quality, based on behavior and characteristics of the individual. From the behavior of the person mapping you can determine the behavior that the robot should have to provide a comfortable situation for interaction with humans. As a result we expect a framework that can learn and analyze the human behavior and also be able to transfer this knowledge to the robot inserted in the environment, increasing the effectiveness of the interaction between humans and robots.

\keywords{Social Robotic, Proxemics, Machine Learning, Human-Robot Interaction}
\end{abstract}



%%%%%%%%%%%%%%%%%%%%%%%%%%%%%%%%%%%%%%%%%%%%%%%%%%%%%%%%%
\listoffigures
% \listoftables
% \listofalgorithms
% \printglossaries

\tableofcontents

%%%%%%%%%%%%%%%%%%%%%%%%%%%%%%%%%%%%%%%%%%%%%%%%%%%%%%%%%
