%!TEX root=Principal.tex
\chapter{EXPERIÊNCIA DE USUÁRIO}
\label{cap:ux}
Os novos sistemas computacionais e maneiras de interação fazem com que especialistas fiquem empolgados e utilizem processos para criar e refinar as aplicações básicas que são apresentadas no mercado ao longo do tempo~\cite{hartson:2012}. Assim como sistemas robóticos, que são sistemas capazes de interagir de diversas maneiras com o usuário além de ser totalmente ativo nos cenários encontrados no dia-a-dia.

Dessa maneira, pode-se dizer que sistemas computacionais vão além de computadores de mesa ou notebooks, além de interface gráfica com usuário, seja em sistemas locais ou que são executados em servidores na nuvem e web. Cada vez mais, sistemas computacionais tornam-se ubíquos, ou seja, difundindo entre os produtos mais inesperados do mercado, sendo peças de roupas ou eletrodomésticos~\cite{hartson:2012}.

Ao desenvolver um produto voltado para seres humanos, este produto necessariamente terá um usuário. Sendo assim, toda vez que esse produto for utilizado, ele proverá uma experiência~\cite{garrett:2010}.

Pode-se definir como experiência de usuário, a experiência criada por um produto em pessoas que fazem seu uso no dia-a-dia dentro do mundo real. A experiência de usuário é parte de uma equação de ``como isso funciona'', geralmente em um pedaço que não tem muita atenção no projeto, mas é essencial para determinar o sucesso ou a falha no lançamento deste produto~\cite{garrett:2010}.

Experiência de usuário refere-se em como o produto funciona fora do laboratório, quando pessoas em situações reais entram em contato com ele no seu dia-a-dia. Observando o assunto de uma certa maneira, todos os produtos existentes e disponíveis para consumo geram uma experiência de usuário, de garrafas de ketchup à suéteres, de livros a computadores, e qualquer outro produto que possa imaginar~\cite{garrett:2010}.

Um produto desenvolvido para prover boa experiência de usuário, vai além de funcionalidades e estética. Desenvolver um produto corretamente refere-se a questões psicológias e comportamentais com os próprios usuários durante o uso. Quanto mais complexo for um produto, maior a dificuldade de entregar um experiência adequada ao usuário~\cite{garrett:2010}.

A maneira mais eficiente de prover experiência de usuário correta em um produto, é utilizando o projeto centrado no usuário. Esse tipo de projeto, considera o usuário durante todas as etapas do projeto, guiando o produto para resultados surpreendentes apesar de mais complexos para análise~\cite{garrett:2010}.

Garantir uma boa experiência de usuário pode ser realizado através do conceito de usabilidade, presente nos conceitos de interação humano computador. Uma interação humano computador ocorre quando um usuário (humano) e um sistema (computador) trabalham juntos com o objetivo de realizar algo em comum~\cite{hartson:2012}.

A usabilidade é um conceito que tem como objetivo principal garantir a interação com efetividade, eficiência e satisfação para o usuário. A ISO 9241-11 de 1997, define algumas características para usabilidade: (I) Fácil de usar; (II) Produtividade; (III) Eficiência; (IV) Efetividade; (V) Fácil aprendizado; (VI) Retenção de conhecimento; e (VII) Satisfação do usuário~\cite{hartson:2012}.
