%!TEX root=Principal.tex
\chapter{EXPERIÊNCIA DE USUÁRIO}
\label{cap:ux}
Novos sistemas computacionais e maneiras de interação fazem com que especialistas fiquem empolgados e utilizem processos para criar e refinar as aplicações básicas que são apresentadas no mercado ao longo do tempo~\cite{hartson:2012}. Assim como sistemas robóticos, que são sistemas capazes de interagir de diversas maneiras com o usuário além de poderem ser totalmente ativo nos cenários encontrados nas tarefas do dia-a-dia.

Dessa maneira, pode-se dizer que sistemas computacionais vão além de computadores de mesa ou notebooks, além de interface gráfica com usuário, seja em sistemas locais ou executados em servidores na nuvem e web. Cada vez mais sistemas computacionais tornam-se ubíquos, ou seja, difundido entre os produtos mais inesperados do mercado, sendo peças de roupas ou eletrodomésticos~\cite{hartson:2012}.

Ao desenvolver um produto voltado para seres humanos, este produto necessariamente terá um usuário. Então toda vez que esse produto for utilizado, ele proverá uma experiência para a pessoa que o usou~\cite{garrett:2010}.

Essa experiência vivida, a experiência do usuário, é definida como a experiência criada por um produto em pessoas que fazem seu uso no dia-a-dia dentro do mundo real. Ela é parte de uma equação de ``como isso funciona'', geralmente em um pedaço que não tem muita atenção no projeto mas, é essencial para determinar o sucesso ou a falha no lançamento deste produto~\cite{garrett:2010}.

Experiência de usuário refere-se em como o produto funciona fora do laboratório, quando pessoas em situações reais entram em contato com ele diariamente. Observando o assunto, de uma certa maneira todos os produtos existentes e disponíveis para consumo geram uma experiência de usuário, de garrafas de ketchup à suéteres, de livros a computadores, e quaisquer outros produtos que possam imaginar~\cite{garrett:2010}.

Um produto desenvolvido para prover boa experiência de usuário, vai além de funcionalidades e aparência estética. Desenvolver um produto corretamente refere-se a questões psicológias e comportamentais com os próprios usuários durante o uso. Quanto mais complexo for um produto, maior a dificuldade de entregar um experiência adequada ao usuário~\cite{garrett:2010}.

A maneira mais eficiente de prover experiência de usuário correta em um produto, é utilizando o projeto centrado no usuário. Esse tipo de projeto considera o usuário durante todas suas etapas, guiando o produto para resultados surpreendentes apesar de mais complexos para análise~\cite{garrett:2010}.

Garantir uma boa experiência de usuário pode ser realizado através do conceito de usabilidade, presente nos conceitos de interação humano-computador~(IHC). Uma interação entre humano e computador ocorre quando um usuário (humano) e um sistema (computador) trabalham juntos com o objetivo de realizar algo em comum~\cite{hartson:2012}.

A usabilidade é um conceito que tem como objetivo principal garantir a interação com efetividade, eficiência e satisfação para o usuário. A ISO 9241-11 de 1997, define algumas características para usabilidade: (I) Fácil de usar; (II) Produtividade; (III) Eficiência; (IV) Efetividade; (V) Fácil aprendizado; (VI) Retenção de conhecimento; e (VII) Satisfação do usuário~\cite{hartson:2012}.

Um produto que entrega uma experiência de usuário adequada, é mais importante do que um produto que possui muitas funcionalidades, um exemplo apresentado por \citeonline{hartson:2012} é o do Blackberry que comparado ao iPhone possui muito mais funcionalidades, porém a experiência entregue de maneira inadequada fez com que ele fosse desbancado pelo último no mercado. A experiência pela interação é o sistema em si, no ponto de vista do usuário.

A experiência de usuário possui o seguinte escopo~\cite{hartson:2012}:

\begin{itemize}
    \item efeitos com base nos fatores de usabilidade;
    \item efeitos com base nos fatores de utilidade;
    \item efeitos com base nos fatores de impacto emocional.
\end{itemize}

Dentro dos fatores que afetam os efeitos da experiência de usuário, pode-se listar pelo menos 5 (cinco) qualidades diferentes que impactarão a experiência de um usuário ao interagir com um determinado sistema~\cite{hartson:2012}:

\begin{itemize}
    \item \textbf{Utilidade}: talvez a mais fundamental das qualidades. Está ligada ao conceito do que serve o sistema. Se é importante para o usuário ou o quão interessante é o conteúdo exposto no sistema. Lembrando que um conteúdo poder ser interessante para um usuário, mas não para o outro. Um determinado produto pode atingir múltiplos de maneira diferente, de acordo com o interesse de cada um. É importante conhecer e manter sólido esse conhecimento sobre o público principal do produto.
    \item \textbf{Integridade Funcional}: é a qualidade de manter o sistema funcionando, como ele deve funcionar. A falta de integridade funcional resulta em um produto com muitos erros e até mesmo vírus no código produzido do sistema.
    \item \textbf{Usabilidade}: refere-se ao quanto é fácil de aprender a usar, quando trata-se de usuários de primeira viagem e esporádicos, e o quanto é fácil de usar, tratando-se de usuários frequentes ao uso do sistema. Um produto pode atender as questões de utilidade e integridade funcional, porém o seu uso pode ser difícil e ainda apresentar tédio para o usuário.
    \item \textbf{Persuasividade}: é quando o produto consegue, em um determinado nível, incentivar o seu uso, manter uma conversa com o usuário para que ele sinta-se atraído, além de direcionar comportamentos específicos durante seu contato.
    \item \textbf{Aparência (Design Gráfico)}: são as cores, tipografias e todos os elementos referentes a aparência do produto onde é possível gerar um grande impacto na experiência do usuário. Todos os elementos geram impactos emocionais nos usuários do sistema e podem fazer total diferença na hora dele optar por continuar ou não o uso.
\end{itemize}

Todas essas qualidades de experiência de usuário contribuem entre si, porém considerá-las de maneira separa auxilia na aplicação efetiva durante o projeto do produto, sendo este um website, uma caixa de presente ou até mesmo um robô social de serviço doméstico.

Alguns especialistas definem experiência de usuário como uma sequência de efeitos sentidos pelo usuário, em seu interior, ao interagir com algo ou alguma coisa. Contudo, nem todos os sentimentos causados pelos efeitos do uso ou interação com o sistema são internos. Muitos dos efeitos podem ser causados pela aplicação de técnicas envolvendo o conceito de usabilidade e utilidade. Sendo assim, pode-se dizer em vias gerais que a usabilidade e utilidade de um produto auxiliam na promoção da experiência do usuário~\cite{hartson:2012}.

Outro ponto importante apontado por especialistas é que uma experiência de usuário não pode ser projetada. Ela é experimentada durante o uso de um produto ou sistema qualquer. A experiência de usuário ocorre em um determinado contexto de aplicação e depende do usuário e seu estado emocional o que vai sentir naquele instante de tempo. O mesmo projeto, aplicado em outro contexto, pode gerar uma experiência totalmente diferente para o mesmo usuário e também todos os demais~\cite{hartson:2012}.

O quanto é apresentado de impacto emocional durante a experiência, fica implícito que são questões referentes a diversão, estética/aparência, sensações, experimentação, originalidade e inovação. Em outras palavras, refere-se ao impacto emocional durante o processo de interação entre o usuário e o produto/sistema. Geralmente, usuários não se encantam mais por eficiência e eficácia dos produtos no mercado. Eles buscam ``sentir'' mais os produtos com os quais interagem~\cite{hartson:2012}.

A experiência de usuário é tratada como algo que pode impactar as emoções de uma pessoa, como algo transcedente ao ser. Ela afeta pessoas de maneiras diferentes e até afeta de forma espiritual. Uma área definidade como tecnoespiritualidade estuda as causas da experiência do usuário como algo que pode ser mundano, natural até algum fator místico ou de crença do próprio ser~\cite{hartson:2012}.

É preciso compreender bem o projeto de um produto para que seja possível o mapeamento adequado dos reais usuários e na sequência utilizar as técnicas desenvolvidas ao longo dos anos para maximizar a experiência positiva da interação com o produto. Para isso, é necessário definir os objetivos dessa experiência. Esses objetivos são, geralmente, de alto nível dentro de um projeto de interação, onde torna-se possível antecipar a experiência do usuário junto ao produto. Como exemplos de objetivos de experiência de usuário, pode-se citar fácil de usar, evitar erros para usuários esporádicos, alta satisfação do cliente, entre alguns outros~\cite{hartson:2012}.

Um ponto chave para auxiliar a atender os objetivos é a identificação dos usuários e quais seus perfis. Esse mapeamento facilita a comunicação das tomadas de decisões na construção do projeto e também na evolução e adaptação do sistema para o perfil do usuário. Uma das técnicas utilizadas para esse tipo de tarefa é a teoria de modelagem de usuário como Personas. Personas é a técnica que tem sido mais adotada em trabalhos de projeto de interfaces com o usuário. A seção~\ref{sec:personas} apresentará como a técnica de Personas auxilia no entendimento das necessidades e identificação do perfil do usuário real do sistema.

\section{Entendedo o usuário através de Personas}
\label{sec:personas}

Perfis de usuário são construídos através de informações detalhadas, coletadas em um processo interativo e vinculados com o objetivo principal do sistema ao qual será utilizado pelo perfil. As informações que o compõe devem ser voltadas para o ponto principal do produto ou sistema. Informações pessoais, a familiaridade dele com a tecnologia, o domínio que ele tem sobre o assunto e também informações dele com relação ao produto~\cite{barbosa:2010}.

Um perfil de usuário pode ser utilizado de diversas maneiras, inclusive para definir papéis e classes de usuário. Quando o objetivo é obter o perfil de usuário, a ferramenta mais adequada para essa tarefa são as Personas. Elas são ótimas quando trabalhas em conjunto com histórias, cenários e encenações~\cite{hartson:2012}.

Personas não são definidas como usuários reais, mas sim como arquétipos hipotéticos ou possíveis usuários do produto. Também pode ser definida como um personagem fictício capaz de representar um grupo de usuários reais com características similares~\cite{aquino:2005, barbosa:2010, hartson:2012, masiero:2013}.

O uso de Personas faz-se importante para criar as funcionalidades corretas aos usuários corretos. Evitar discussões de projeto sem necessidade no momento é uma das suas principais características. O uso desta ferramenta auxilia na comunicação da equipe, facilitando e mantendo o foco no usuário~\cite{aquino:2005, hartson:2012, masiero:2013}.

Ao se projetar um sistema é natural que o especialista pense em como será sua reação com a funcionalidade X dada a aparência Y. Porém, este tipo de comportamento, em muitos casos, leva a falha do produto. Personas auxiliam projetistas a não cometerem este equívoco, forçando-os a pensar como a Persona Maria, por exemplo, irá reagir a uma determinada funcionalidade ou interface apresentada~\cite{hartson:2012}.

A Persona pode ser classificada como primária ou secundária. A primária deve ser totalmente atendida no projeto final, ela deve estar 100\% satisfeita e feliz com o produto. A melhor experiência ao interagir com o produto deve ser dela. Ao mesmo tempo, as Personas secundárias são atendidas com um alto grau de satisfação, porém com um percentual de satisfação sempre abaixo da primária~\cite{hartson:2012}.

Para que a efetividade da ferramenta seja maior e também para identificar quem são as personas secundárias e quem é a primária, é importante que seja definido um cenário de interação. Um cenário é uma narrativa, seja ela textual ou através de figuras (pictóricas), concreta e com um alto nível de detalhes descrevendo pessoas executando alguma atividade~\cite{barbosa:2010}.

A partir do momento que as informações do usuário e cenário estão definidas, é necessário fazer uma validação do sistema através de um método de avaliação da interface e interação do usuário. Alguns métodos são conhecidos em trabalhos de IHC. Eles tem o objetivo de minimizar erros que possam vir a acontecer no uso do sistema. A seção~\ref{sec:avaliacao} apresenta um meio de realizar a atividade de avaliação da interface e interação do produto.

\section{Avaliando a interação com o usuário}
\label{sec:avaliacao}
Existem muitos tipos de avaliação de interface e interação do usuário com o sistema. Alguns exemplos são a avaliação heurística, o percuso cognitivo, o teste com usuários, grupo focal, entre outros~\cite{barbosa:2010}.

No percurso cognitivo um especialista vai narrando o cenário e o usuário diz quais são as ações que devem ser tomadas no sistema. Testes com o usuário são realizados com o uso do sistema, onde especialistas devem fazer anotações e observações da interação e erros enquanto o usuário narra todos os seus pensamentos e passos em voz alta. O grupo focal, basicamente, é um grupo de usuários que vão discutindo sobre o que eles acharam do produto, onde tiveram dificuldades e se compreenderam o objetivo~\cite{barbosa:2010}. Dentre todos, a avaliação heurística é o que tem o custo menor, é o mais simples de aplicar e também é o mais utilizado em projetos de avaliação de interação do usuário~\cite{tsui:2010}.

A avaliação heurística é uma avaliação onde o especialista percorre sistematicamente todo o sistema buscando por problemas que venham impactar na usabilidade, e consequentemente podendo gerar uma experiência negativa de interação para o usuário~\cite{barbosa:2010, benyon:2011}.

Esse método é composto por algumas diretrizes de usabilidade, onde é possível identificar se a interação e a interface possuem características desejadas e de alto valor para o usuário~\cite{barbosa:2010, benyon:2011}. As diretrizes, também chamadas de heurísticas, mais populares entre os especialistas são as de \citeonline{nielsen:1994}, que foram as primeiras heurísticas apresentas com esse objetivo. Ao todo foram criadas 10 heurísticas com base em problemas frequentes encontrados por \citeonline{nielsen:1994} ao longo de alguns anos de trabalho. A tabela~\ref{tab:heuristicasnielsen} apresenta as 10 heurísticas originais de Nielsen.

\begin{table}[!ht]
	\caption{As 10 heurísticas de Nielsen}
	\label{tab:heuristicasnielsen}
	\centering
	\begin{tabular}{ c | m{4cm} | m{10cm} }
		\hline
		ID & Heurística & Descrição \\
		\hline
		01 & Visibilidade do estado do sistema & Sempre informar o usuário sobre o que está acontecendo no sistema de maneira adequada e no tempo correto. \\
		\hline
        02 & Correspondência entre os sistemas e o mundo real & O uso de linguagens comuns para os usuários. A ordem das informações devem manter uma sequência natural e lógica, de acordo com o esperado pelo usuário. \\
		\hline
        03 & Controle e liberdade & O sistema deve permitir que o usuário desfaça e refaça suas ações. \\
		\hline
        04 & Consistência e padronização & Manter as convenções da plataforma ou do ambiente computacional. \\
		\hline
        05 & Reconhecimento em vez de memorização & As intruções do uso devem ser de fácil acesso no momento que o usuário desejar. \\
		\hline
        06 & Flexibilidade e eficiência de uso & Possibilidade de atalhos que facilitem a operação do sistema por parte do usuário. Possibilidade de personalização da interface. \\
		\hline
        07 & Projeto estético e minimalista & A interface não deve possuir informações desnecessárias a tarefa realizada. \\
		\hline
        08 & Prevenção de erros & O sistema é capaz de contornar erros mantendo o seu funcionamento. \\
		\hline
        09 & Ajude os usuários a reconhecerem, diagnosticarem e se recuperarem de erros & Uso de linguagem simples ao apresentar erros e mostrar explicitamente a solução para tal. \\
		\hline
        10 & Ajuda e documentação & Uma boa documentação deve sempre estar disponível para que o usuário possa acessar adequadamente. \\
		\hline
	\end{tabular}
	\smallcaption{Fonte: \citeonline{nielsen:1994}.}
\end{table}

Esse conjunto de heurísticas apresentado na tabela~\ref{tab:heuristicasnielsen}, pode ser considerado como mínimo e pode receber novas diretrizes com o intuito de expandir ou ajustar de acordo com a necessidade do projeto e avaliadores~\cite{barbosa:2010, benyon:2011}.

Desde que sistemas robóticos começaram a coexistir com os seres humanos, pesquisadores de IHC e robótica começaram a se preocupar com as interações e também as interfaces entre os dois. A seção~\ref{sec:ihrux} apresenta os trabalhos relacionados envolvendo trabalhos que utilizam técnicas de IHC com o intuito de melhorar a interação entre os seres humanos e robôs.

\section{A experiência de usuário em interações com robô}
\label{sec:ihrux}
Alguns trabalhos tem tratado experiência de usuário e técnicas de IHC para melhorar a qualidade de projetos em robótica social, de serviço e assistiva. Pesquisadores em experiência de usuário têm se mobilizado para entender como as pessoas estão se sentindo em relação a essa nova tecnologia e como melhorar a experiência com os robôs, principalmente os autônomos.

A técnica de GOMS (\emph{Goals, Operators, Methods and Selections}) foi adaptada para entender projetos de interação humano-robô~(IHR) como modelos de processamento humano. Assim, a modelagem de tarefas do robô em diversos cenários pode ter benefícios e maior efeciência na execução~\cite{drury:2007}.

\citeonline{clarkson:2007} apresentam um conjunto de heurísticas para a avaliação de projetos em IHR. A construção e validação desse conjunto foram feitas através da adaptação das heurísticas de Nielsen e Scholtz, e aplicação de métricas apresentadas no método de Nielsen. Com base nas métricas a avaliação deve ser realizada com 3-5 avaliadores e o total de problemas encontrados com as heurísticas deve estar em torno de 40-60\%, utilizando métodos e projetos diferentes. As avaliações entre sistemas diferentes não é estatisticamente relevante, porém como atenderam as métricas, os autores afirmam que podem ser utilizadas em outros projetos. As avaliações ocorreram em um robô para cenário de resgate. Após as adaptações das heurísticas, foi definido um conjunto com 8 heurísticas para sistemas de IHR. Elas são apresentadas na tabela~\ref{tab:heuristicasihr}.

\begin{table}[!ht]
	\caption{8 heurísticas de IHR baseada nos conjuntos de Nielsen e Scholtz}
	\label{tab:heuristicasihr}
	\centering
	\begin{tabular}{ c | m{4cm} | m{10cm} }
		\hline
		ID & Heurística & Descrição \\
		\hline
		01 & Design de informações suficientes & As interfaces devem prover informações o suficiente para que o usuário possa determinar se precisa intervir, mas também não pode sobrecarregá-lo com excesso de informação. \\
		\hline
        02 & Visibilidade do estado do sistema & O sistema deve sempre manter o usuário informado sobre o que está acontecendo, através de um retorno com tempo apropriadamente calculado. O sistema deve prover um modelo do mundo real de maneira completa e permitir que o usuário possa ver isso, tendo total entendimento da situação. O sistema deve auxiliar o usuário a ter consciência da situação. \\
		\hline
        03 & Apresentação apropriada da informação & A interface deve apresentar informações claras sobre os sensores, que devem ser de fácil compreensão, e de maneira útil ao usuário. O sistema deve utilizar o princípio do reconhecimento por recuperação, externalização de memória. Deve apoiar o gerenciamento da atenção do usuário. \\
		\hline
        04 & Uso de sugestões naturais & A linguagem utilizada para a comunicação do sistema com o usuário deve acontecer por palavras, frases e conceitos familiares ao usuário e não em termos orientados a sistemas. Seguir convenções do mundo real, apresentar informações em ordem lógica e de maneira natural. \\
		\hline
        05 & Síntese do sistema e interface & A interface e o sistema devem trabalhar como um só fazendo com que a interface seja uma extensão do sistema, do usuário e por representação, do mundo. A interface deve facilitar de maneira eficiente e com eficácia a comunicação entre o sistema e o usuário, em uma via dupla. \\
		\hline
        06 & Ajudar o usuário a reconhecer, diagnosticar, e recuperar de erros & O sistema com mal funcionamento deve se expressar através da linguagem simples (sem códigos), precisamente indicar o problema, e de maneira construtiva sugerir uma solução. A informação deve ser suficiente a ponto do usuário poder identificar se o ambiente contribuiu de alguma forma ao problema. \\
		\hline
        07 & Flexibilidade da Arquitetura da Informação & Se o sistema será utilizado por um longo período, a interface deve ser capaz de suportar novos itens como capacidade de sensores e atuadores, mudanças de comportamento e alterações físicas. A capacidade de sensores e atuadores devem ser adequados ao tipo de tarefa e ambiente esperados para o sistema. \\
		\hline
        08 & Projeto minimalista e estético & Informações do sistema deve ser apenas necessárias, sem o uso de informações irrelevantes. O formato físico deve ser agradável e de acordo com a função pretendida. \\
		\hline
	\end{tabular}
	\smallcaption{Fonte: \citeonline{clarkson:2007}.}
\end{table}

Como mesmo propósito, porém em um domínio diferente, \citeonline{elara:2007} também apresenta um conjunto de heurísticas com o foco na interação humano e robô humanóide~(IHRH) dentro do domínio de futebol. O conjunto apresentado é uma adaptação direta das 10 heurísticas de Nielsen. As heurísticas propostas tiveram uma resposta de aproximadamente 35\% mais problemas encontrados do que as originais de Nielsen. O conjunto de heurísticas para IHRH são apresentadas na tabela~\ref{tab:heuristicasfutebol}.

\begin{table}[!ht]
	\caption{8 heurísticas de IHRH adaptadas do conjunto de Nielsen}
	\label{tab:heuristicasfutebol}
	\centering
	\begin{tabular}{ c | m{4cm} | m{10cm} }
		\hline
		ID & Heurística & Descrição \\
		\hline
        01 & Visibilidade do estado do sistema & O sistema deve sempre manter o usuário informado sobre o que está acontecendo, através de um retorno com tempo apropriado. \\
		\hline
        02 & Clareza na apresentação da informação & A interface deve ser desenvolvida para apresentar de maneira clara e compreensiva a informação de sensores e atuadores. \\
		\hline
        03 & Correspondência entre o sistema e o mundo real & A linguagem utilizada para a comunicação do sistema com o usuário deve acontecer por palavras, frases e conceitos familiares ao usuário e não em termos orientados a sistemas. \\
		\hline
        04 & Posicionamento prioritário de informações & Posicionamento prioritário dos botões de controle de acordo com a importância e frequência de uso. \\
		\hline
        05 & Extendibilidade do sistema & O sistema devem permitir a evolução, como inclusão de sensores, atuadores, componentes de comportamento e habilidades. \\
		\hline
        06 & Ajudar o usuário a reconhecer, diagnosticar, e recuperar de erros & O sistema com mal funcionamento deve se expressar através da linguagem simples (sem códigos), precisamente indicar o problema, e de maneira construtiva sugerir uma solução. Informações abstratas do robô humanóide que o ambiente pode prover para o usuário com fins de depuração. \\
		\hline
        07 & Arquitetura de comunicação efetiva & A interface e o sistema devem trabalhar como um só fazendo com que a interface seja uma extensão do sistema, do usuário e por representação, do mundo. A interface deve facilitar de maneira eficiente e com eficácia a comunicação entre o sistema e o usuário, em uma via dupla. \\
		\hline
        08 & Projeto minimalista e estético & Informações do sistema deve ser apenas necessárias, sem o uso de informações irrelevantes. O formato físico deve ser agradável e de acordo com a função pretendida. \\
		\hline
	\end{tabular}
	\smallcaption{Fonte: \citeonline{elara:2007}.}
\end{table}

O uso das heurísticas criadas por \citeonline{clarkson:2007} é apresentado no trabalho de \citeonline{lohse:2008}. É construído um robô social que realiza interação com usuários ingênuos (que não tem contato prévio com robôs) durante uma visita guiada por uma casa. Para avaliar o sistema robótico construído, foi feita a avaliação heurística.

Com a avaliação alguns pontos importantes foram apresentados. A avaliação e o desenvolvimento do projeto deve ser iterativo. Testes com usuários reais e ambientes reais apresentam melhores resultados. Os testes devem ser realizados com robôs totalmente autônomos, sem o uso da técnica de Wizard of Oz~(WoZ) onde o robô é teleoperado sem o conhecimento do usuário em teste. As tarefas e contextos devem estar de acordo com o projeto do robô e as heurísticas devem ser incorporadas no projeto de contrução do robô~\cite{lohse:2008}.

\citeonline{lohse:2008} questionam ao final do trabalho, como essa avaliação heurística pode ser incorporada como métricas compreensivas para aceitação social de robôs reais e também como podem afetar o impacto social dos robôs.

Por ser um método de avaliação de interface com baixo custo, simples e com ampla aplicabilidade, o uso de heurísticas é utilizado em diversas pesquisas. Porém, nas aplicações do método sempre existe uma adaptação das heurísticas de Nielsen, por serem as pioneiras. A adaptação do método é feita para que domínios específicos sejam melhor atendidos~\cite{tsui:2010}.

A criação de heurística sempre seguem dois formatos: baseado em métodos de pesquisa ou em métodos empíricos. A validação é feita através de testes de usabilidade realizados empiricamente ou comparando com avaliações feitas através das heurísticas de Nielsen~\cite{tsui:2010}.

No trabalho de \citeonline{tsui:2010} é proposto o desenvolvimento de heurísticas focadas em robôs assistivos. Utilizam, como ferramenta, um manipulador robótico montado em uma cadeira de rodas que se movimenta através de teleoperação. Durante a pesquisa, foram encontrados quatro erros de grande impacto no sistema: segurança; confiança; erros do sistema; e flexibilidade. Para cobrir esses erros foram criadas heurísticas adicionais com base nas de Nielsen, na literatura de acessibilidade e robótica social.

O conjunto de heurísticas exclusivo para robótica assistiva, com o mesmo cenário de teleoperação, apresentou resultados melhores que as de Nielsen. De 39 problemas, as heurísticas de Nielsen são capaz de cobrir apenas 13, enquanto as heurísticas adaptadas cobriram 33. O teste foi realizado apenas com 2 avaliadores, e precisam de mais experimentos para uma avaliação mais significativa~\cite{tsui:2010}.

Um estudo empírico em experiência de usuário foi realizado através de interações com o robô por voz. \citeonline{jokinen:2013} apresentam uma maneira de realizar uma avaliação da interação por voz dando ênfase ao que o usuário realiza para se comunicar com o robô e também em seu processo cognitivo durante a interação. A atividade comunicativa do usuário se correlaciona com o sistema, onde assume-se que a experiência do usuário é positiva quando a sua participação na interação é avaliada pelo próprio como concluída com sucesso ao final do processo.

A experiência não é medida apenas com a informação de sucesso pelo próprio usuário e adoção dele após a interação com o sistema robótico, mas também com o processo psicológico que refere-se a atenção, motivação e percepção do usuário durante o cenário executado~\cite{jokinen:2013}.

Os resultados foram promissores e mostraram que a experiência do usuário caminha para uma direção natural de interação. Contudo, um estudo mais aprimorado sobre a percepção e cognição é necessário, assim como a adição de variáveis para o estudo~\cite{jokinen:2013}.

Após a instalação de um novo manipulador robótico, agora sem a proteção de uma grade, \citeonline{buchner:2013} investigam qual seria a experiência do usuário ao utilizá-lo nas tarefas da fábrica. Uma variável relativa ao tempo de uso da tecnologia foi adicionada ao experimento, com o intuito de identificar o comportamento do usuário conforme o passar dos dias de trabalho. Para isso, questionários foram aplicados em diferentes tempos da produção. O primeiro foi realizado no momento da inauguração do novo manipulador. Esses questionários foram distribuídos entre os operadores no novo robô e de um segundo robô que já estava na fábrica a mais de 10 anos com o espaço protegido por uma cerca. Mais duas rodadas de questionários foram realizadas, após 12 meses da instalação do robô e também 18 meses após este momento.

Resultados apontam que o usuário se acostuma com a tecnologia ao longo do tempo e a experiência não apresenta diferenças após um período. Contudo, não quer dizer que houve melhora na expeirência do usuário com o passar do tempo. Principalmente tratando-se de um robô industrial, o usuário acaba interagindo por obrigação e dever a cumprir com a empresa. Mais pesquisas devem ser realizadas para obter detalhes dos fatores que são importantes e realmente influenciam a experiência do usuário ao longo do tempo~\cite{buchner:2013}.

Um estudo sobre uma interface de controle para robôs de regaste é realizado através de avaliações heurísticas. A melhora da interface é proposta após a identificação de alguns erros. \citeonline{naveed:2014} identificam alguns pontos que são problemas de interação com a interface de controle através de uma avaliação empírica e sem o uso de heurísticas já definidas para o domínio. Após a identificação dos erros, é proposto uma nova interface de controle ao sistema.

\citeonline{saariluoma:2014} aborda o conceito de psicologia de usuário que tem como objetivo utilizar conceitos, teorias e resultados, como meio para estruturar problemas em investigações sobre interações entre humanos e computadores. Três estudos são apresentados com cenários em diferentes tecnologias voltadas a interação. Dois estudos são em laboratório e um em campo. Com os resultados extraídos dos estudos é proposto um modelo bipolar competência-frustação para melhorar o compreendimento de aspectos emocionais da experiência do usuário.

Uma maneira para definir emoção é como sendo parte crítica para uma tomada de decisão efetiva, assim como meio melhor de aprender sobre algo ou situação. O foco da pesquisa apresentada por \citeonline{saariluoma:2014} é em questionários sobre emoções básicas, que são a base de investigações sobre psicologia de usuário em tecnologia envolvendo a base de conceitualização, definições operacionais, interpretação de resultados e explicação de resultados.

Os resultados obtidos através dos experimentos conduzidos foram representações mentais sobre experiência de usuário emocional com o foco principal sobre o efeito de emoções básicas na experiência durante a interação. Para realizar os experimentos de maneira empírica os usuários foram solicitados a refletir sobre como estavam se sentido e expressar-se através das palavras que representavam tais emoções. Assim, foi possível encontrar dois grupos entre as emoções básicas, o positivo nomeado de competências e o negativo como frustrações. Com eles é possível realizar o mapeamento do estado mental da pessoa~\cite{saariluoma:2014}.

Investigações sobre como comportamentos multimodais do usuário podem auxiliar a medir o compromisso durante a interação com o robô, são conduzidas por \citeonline{jokinen:2015}. Foram mapeados diversas combinações de comportamentos do usuário, como direção do olhar, expressões faciais, e postura corporal, para auxiliar na predição da experiência do usuário e avaliação da interação por voz entre 5 categorias, capacidade de resposta, expressividade, interface, usabilidade e impressões gerais.

Após determinar as características para cada comportamento do usuário e também para algumas ações do robô, foi utilizado algoritmos de regressão logística e \emph{support vector machines}~(SVM) para classificá-los entre as cinco categorias de interação mapeadas~\cite{jokinen:2015}.

Como a diferença estatística entre os dois algoritmos não demonstrou representatividade, optou-se por utilizar o SVM. Cada comportamento contribui para classificar uma determinada categoria porém, essas informações são muito complexas e necessitam de mais características para aumentar a acurácia da classificação~\cite{jokinen:2015}.

\citeonline{broadbent:2016} apresenta um guia, no ponto de vista psicológico, sobre quais pontos devem ser estudados para aprimorar a compreensão da interação humano robô de maneira a otimizar o comportamento do robô. O trabalho apresenta conceitos, teorias e modelos de interação humano-humano na psicologia.

Um estudo de caso investiga a experiência do usuário ao trabalhar de forma colaborativa com um robô fixo, em uma linha de produção de veículos. Após trabalharem por três semanas, os usuários que trabalharam em parceria com o robô foram entrevistados utilizando técnicas de questionários em usabilidade. Os usuários sentiram-se limitados com o auxílio do robô. Ele delimita consideravelmente o espaço de trabalho e também a velocidade que cada indivíduo realiza as tarefas ao longo da jornada de trabalho~\cite{weiss:2016}.

Esse tipo de comportamento resultou em uma perda de produtividade e impacto direto na experiência do usuário ao compartilhar as tarefas com o robô. Dado o cenário de experiência de usuário ruim, a primeira tomada decisão na empresa foi regulamentar algumas soluções técnicas fazendo com que o usuário se adapte melhor ao trabalho colaborativo, por exemplo, treinamentos e sequências de ações no trabalho. Na sequência fatores de experiência de usuário devem ser melhor investigados e agregados ao projeto do robô de maneira a aprimorar o cenário~\cite{weiss:2016}.

Trabalhos com experiência de usuário em reabilitação de pacientes também é investigado por pesquisadores da área. \citeonline{shirzad:2016} utilizaram um robô que auxiliou no aumento de casos de sucesso do tratamento e a reabilitação dos pacientes tornou-se mais divertida.

\citeonline{lindblom:2016} defendem a ideia de utilizar técnicas de experiência de usuário em robótica social e falam que é o melhor caminho para o desenvolvimento de robôs sociais com maior aceitação. Utilizar tais técnicas torna-se importante, pois auxiliam em aspectos que aumentam a aceitação, usabilidade e credibilidade dos sistemas robóticos em âmbito social. Para auxiliar os futuros projetos de robótica, os autores informam sobre 3 desafios que devem ser vencidos ao longo dos próximos anos:

\begin{itemize}
    \item Adoção de um processo iterativo de UX design;
    \item Incorporar metas de UX para garantir uma boa experiência;
    \item Projetistas de robótica devem adquirir o conhecimento adequado para a avaliação de UX.
\end{itemize}

É importante olhar cada um desses desafios, pois a aplicação de UX em HRI fará com que os robôs sociais sejam melhor aceitos em diversos ambientes e por pessoas dos mais diferentes perfis~\cite{lindblom:2016}.

Robôs contruídos para serviços no setor de agricultura encontram muitas dificuldades, e quando são totalmente autônomos, no geral, possuem diversas limitações devido ao ambiente sem controle e aberto. Sendo assim, robôs teleoperados são melhores aceitos para o trabalho, pois apresentam resultados 4\% melhores que seres humanos e 14\% melhores que robôs autônomos. Com base nesse cenário, \citeonline{adamides:2017} conduzem uma pesquisa sobre a melhor configuração de interface para o usuário operar o robô. Três variáveis foram consideradas para montar a configuração de interface, são elas: (I) tipo de saída de video (monitor ou capacete de realidade virtual); (II) número de visões ou telas (única ou múltiplas); e (III) tipo de controle do robô (\emph{joystick} ou teclado).

Testes com operadores foram conduzidos para verificar a resposta perante cada configuração. Em seguida, questionários foram aplicados para identificar as configurações que guiaram uma melhor experiência. O uso de múltiplas telas foi a melhor configuração para a variável, pois pontecializa a visão do usuário referente a onde o robô deve atuar. A visão através da tela contribui menos que o uso do capacete, quando se tratado da carga de trabalho exercida pelo usuário. Por fim, a interface via teclado teve melhor resposta na eficiência do trabalho perante o \emph{joystick}. Estudos de representações espaciais devem continuar com o intuíto de elevar mais a experiência do usuário na manipulação do robô~\cite{adamides:2017}.

Muitos artigos na literatura discutem o quão importante são os estudos aplicados a robôs domésticos. Com isso em mente, \citeonline{mcginn:2017} conduziram um estudo para verificar a aptidão dos usuários novatos em controlar o robô pelo ambiente doméstico através de um controle. Os usuários selecionados nunca tiveram contato direto com robôs reais. Para realizar o teste, construiu-se um ambiente virtual e os usuários tinham que realizar a navegação pela casa virtual.

As observações feitas durante o teste demonstraram que os usuários não possuem destreza para executar a tarefa. Houve um número alto de colisões, principalmente durante a transição pela região das portas da residência e corredores estreitos. Os controles demonstram que as técnicas na criação de um interface para controle do robô não apresentam uma usabilidade adequada e promovem uma experiência ruim ao usuário, ponto que será trabalho no futuro~\cite{mcginn:2017}.

Alguns trabalhos fizeram a adaptação da técnica de Personas (vide seção~\ref{sec:personas}) para utilizar na interação humano-robô. A ideia é fazer com que o robô se comporte ou tenha características de uma Persona para realizar tarefas. Além disso, discussões sobre a possibilidade do robô incorporar uma determinada Persona de acordo com o cenário ao qual ele se encontra, também é apresentado nos trabalhos envolvendo o tema. Pouco se fala em utilizar Personas, como método de modelagem e classificação do usuário, durante o desenvolvimento de projetos de robótica social, de serviço e assistiva~\cite{woods:2005, ljungblad:2006, meerbeek:2009, ruckert:2011, duque:2013, ruckert:2013}.

Dado os trabalhos que falam sobre experiência de usuário na literatura, uma discussão ampla é guiada por \citeonline{alenljung:2017}. A investigação na literatura sobre amplicação das técnicas voltadas para trabalhos com experiência de usuário em projetos de interação entre humanos e robôs sociais é feita com algumas críticas. Eles defendem que as técnicas devem ser adaptadas e empregadas de maneira correta ao longo do projeto de criação de robôs sociais, de serviço e assistivos, em principal os autonômos. Apesar de muitos trabalhos serem voltados para que haja uma boa experiência de interação, não existe a formalidade das técnicas aplicadas como em cenários de interação humano-computador. Trabalhar com as técnicas tradicionais de maneira adaptada é essencial e pode otimizar o tempo de aceitação do robô na sociedade, tornando uma área muito importante para estudos e pesquisas dedicadas.
